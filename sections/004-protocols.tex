\section{Protocols and Composition}

\begin{definition}[Base Protocols]
A \emph{base protocol} $\mathcal{P}$ consists of:
\begin{itemize}
\item Systems $P_1, \ldots, P_n$, called \emph{players}
\item A game $F$, called the \emph{ideal functionality}
\end{itemize}

Furthermore, we also impose requirements on the channels and functions
these elements use.

First, we require that the player systems are jointly closed,
with no extra channels that aren't connected to other players:
$$
\bigcup_{i \in [n]} \text{OutChan}(P_i) = \bigcup_{i \in [n]} \text{InChan}(P_i)
$$

Second, we require that the functions the systems depend on are disjoint:
$$
\forall i, j \in [n].\quad \text{In}(P_i) \cap \text{In}(P_j) = \emptyset
$$

We can also define a few convenient notations related to the interface of a base
protocol.

Let $\text{Out}_i(\mathcal{P}) := \text{Out}(P_i)$, and let $\text{In}_i(\mathcal{P}) := \text{In}(P_i) / \text{Out}(F)$.
We then define $\text{Out}(\mathcal{P}) := \bigcup_{i \in [n]} \text{Out}_i(\mathcal{P})$
and $\text{In}(\mathcal{P}) := \bigcup_{i \in [n]} \text{In}_i(\mathcal{P})$.
Finally, we define $\text{IdealIn}(\mathcal{P}) := \text{In}(F)$.

$\square$
\end{definition}
