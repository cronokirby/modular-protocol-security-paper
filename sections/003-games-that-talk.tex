\section{Systems}

The goal of this section is to extend the notion of packages
to that of \emph{systems}.
Intuitively, systems are like packages, except that they can send
messages to other systems via channels.
This is very useful, since it lets us model the kind of interaction
we need to describe protocols.

Continuing with the machine analogy, we can see packages like machines,
arranged into rows, with each row using output functions
provided by the row behind it.
Systems have the same setup, except that now all the machines within
a given row have the ability to communicate with each other via channels.

\subsection{Asynchronous Packages}

Before we get to channels, we first need to define a notion
of packages that have \emph{asynchronous} functions.
This becomes necessary to have channels, since we need to be
able to handle the case where a system is receiving a message along
channel, and is waiting for that message to arrive.
A natural way to model this is an asynchronous process,
where a system can \emph{yield} control back to the caller,
indicating that it isn't able to provide an answer yet,
because it's waiting on something else to happen first.

While the intuition of yielding control is simple, defining
it precisely is a bit more tricky.
Ultimately, the definition we provide isn't very elegant,
but we think it's a very straightforward approach
providing a clear meaning to yield statements.

\begin{definition}[Yield Statements]
\label{def:yield}
We define the semantics of $\textbf{yield}$ by compiling functions with
such statements to functions without them.

Note that we don't define the semantics for functions which still contain references
to oracles.
Like before, we can delay the definition of semantics until all of the
pseudo-code has been inlined.

A first small change is to make it so that the function accepts one
argument, a binary string, and all yield points also accept binary
strings as continuation.
Like with plain packages, we can implement richer types on top by adding
additional checks to the well-formedness of binary strings, aborting
otherwise.

The next step is to make it so that all the local variables of the function $F$
are present in the global state.
So, if a local variable $v$ is present, then every use
of $v$ is replaced with a use of the global variable $F.v$ in the package.
This allows the state of the function to be saved across yields.

The next step is transforming all the control flow of a function to
use $\textbf{ifgoto}$, rather than structured programming constructs like
$\textbf{while}$ or $\textbf{if}$.
The function is broken into lines, each of which contains a single statement.
Each line is given a number, starting at $0$.
The execution of a function $F$ involves a special variable $\texttt{pc}$,
representing the current line being executing.
Excluding $\textbf{yield}$ and $\textbf{return}$ a single line statement has one of the
forms:
$$
\begin{aligned}
&\langle \texttt{var} \rangle \gets \langle \texttt{expr} \rangle\cr
&\langle \texttt{var} \rangle \xleftarrow{\$} \langle \texttt{dist} \rangle\cr
\end{aligned}
$$
which have well defined semantics already.
Additionally, after these statements, we set $\texttt{pc} \gets \texttt{pc} + 1$.

The semantics of $\textbf{ifgoto } \langle \texttt{expr} \rangle i$ is:
$$
\texttt{pc} \gets \textbf{ if } \langle \texttt{expr} \rangle \textbf{ then } i \textbf{ else } \texttt{pc} + 1
$$
This gives us a conditional jump, and by using $\texttt{true}$ as the condition,
we get a standard unconditional jump.

This allows us to define $\textbf{if}$ and $\textbf{while}$ statements
in the natural way.

Finally, we need to augment functions to handle $\textbf{yield}$ and $\textbf{return}$
statements.
To handle this, each function $F$ also has an associated variable
$F.\texttt{pc}$, which stores the program counter for the function.
This is different than the local $\texttt{pc}$ which is while the function is
execution.
$F.\texttt{pc}$ is simply used to remember the program counter after a yield
statement.

The function now starts with:
$$
\begin{aligned}
&\textbf{ifgoto } \texttt{true}\ F.\text{pc} 
\end{aligned}
$$
This has the effect of resuming execution at the saved program counter.

Furthermore, the input variable $x$ to $F$ is replaced with a special
variable $\texttt{input}$, which holds the input supplied to the function.
At the start of the function body, we add:
$$
0: F.x \gets \texttt{input}
$$
to capture the fact that the original input variable needs to get assigned
to the input to the function.

The semantics of $F.m \gets \textbf{yield } v$ are:
$$
\begin{aligned}
(i-1):&\ F.\text{pc} \gets i + 1\cr
i:&\ \textbf{return } (\texttt{yield}, v)\cr
(i + 1):&\ F.m \gets \texttt{input}
\end{aligned}
$$

The semantics of $\textbf{return } v$ become:
$$
\begin{aligned}
&F.\text{pc} \gets 0\cr
&\textbf{return } (\texttt{return}, v)
\end{aligned}
$$
The main difference is that we annotate the return value to be different than
yield statements, but otherwise the semantics are the same.

$\square$
\end{definition}

Note that while calling a function which can yield will notify the
caller as to whether or not the return value was \emph{yielded}
or \emph{returned}, syntactically the caller often ignores this,
simply doing $x \gets F(\ldots)$, meaning that they simply use
return value $x$, discarding the tag.
We describe a bit of syntax for this shorthand.

\begin{syntax}[Empty Yields]
In many cases, no value is yielded, or returned back, which we can write as:
$$
\textbf{yield}
$$
which is shorthand for:
$$
\bullet \gets \textbf{yield } \bullet
$$
i.e. just yielding a dummy value and ignoring the result.

$\square$
\end{syntax}

Unless otherwise specified, we only consider empty yields from now on.
In other contexts, being able to yield intermediate values
can be useful, but for modeling channels, we only need
empty yields.

Very often, a package just wants to run another asynchronous
process to completion.
It's not enough to simply loop until the process completes,
because this might cause an infinite loop, as some external
intervention might be necessary to cause the process to make progress.
Instead, we want to poll the process, and yield ourselves
if the process is not yet ready.
We define these semantics via the $\textbf{await}$ statement.

\begin{syntax}[Await Statements]
\label{syn:await}
We define the semantics of $v \gets \await F(\ldots)$ in a straightforward way:
$$
\begin{aligned}
&(\tx{tag}, v) \gets (\texttt{yield}, \bot)\cr
&\txbf{while } \tx{tag} = \texttt{yield}:\cr
\pind{1}\txbf{if } v \neq \bot:\cr
\pind{2}\txbf{yield}\cr
\pind{1}(\tx{tag}, v) \gets F(\ldots)\cr
\end{aligned}
$$
In other words, we keep calling the function until it actually returns
its final value, but we do yield to our caller whenever our function yield, but we do yield to our caller whenever our function yields.

$\square$
\end{syntax}

In practice, $\txbf{await}$ is the most common way that asynchronous functions
will be called.
Most systems will await other functions directly, and maybe only
adversaries will care about being able to see the underlying polling
process.

However, sometimes we want to await several values at once, returning the first
one which completes. To that end, we define the $\textbf{select}$ statement.

\begin{syntax}[Select Statements]
\label{syn:select}
Select statements generalize await statements in that they allow waiting
for multiple events concurrently.

More formally, we define:
$$
\begin{aligned}
&\txbf{select}:\cr
\pind{1}v_1 \gets \await F_1(\ldots):\cr
\pind{2} \langle \tx{body}_1 \rangle\cr
\pind{1}\vdots\cr
\pind{1}v_n \gets \await F_n(\ldots):\cr
\pind{2} \langle \tx{body}_n \rangle\cr
\end{aligned}
$$
As follows:
$$
\begin{aligned}
&(\tx{tag}_i, v_i) \gets (\texttt{yield}, \bot)\cr
&i \gets 0\cr
&\txbf{while } \nexists i.\ \tx{tag}_i \neq \texttt{yield}:\cr
\pind{1} \txbf{if } i \geq n:\cr
\pind{2} i \gets 0\cr
\pind{2} \txbf{yield}\cr
\pind{1} i \gets i + 1\cr
\pind{1} (\tx{tag}_i, v_i) \gets F_i(\ldots)\cr
&\langle \tx{body}_i \rangle
\end{aligned}
$$
Note that the order in which we call the functions is completely deterministic,
and fair.
It's also important that we yield, like with await statements, but we only
do so after having pinged each of our underlying functions at least once.
This is so that if one of the function is immediately ready, we never yield.

$\square$
\end{syntax}

This kind of situation can arise quite often when defining protocols,
where you might be waiting on a message from any one of several parties.
Using a select statement lets a package wait for the first message
that happens to arrive.

Another variant of waiting occurs when we want to wait for some
\emph{condition} to be true.
For example, we could set up a lock over a shared value,
and we might need to wait for the lock to be free, so we can modify
the value.
We model this kind of situation with a \txbf{wait} statement.

\begin{definition}[Wait Statement]
  \label{syn:wait}
  We define the semantics of $\txbf{wait } \langle \tx{cond} \rangle$
  as equivalent to:
  $$
  \begin{aligned}
  &\pwhile{\neg \langle\tx{cond}\rangle}\cr
  \pind{1} \pyield{}\cr
  \end{aligned}
  $$
  $\square$
\end{definition}

So, we simply keep yielding until the condition is true.
This is simple, but surprisingly useful.

We've defined the various asynchronous gadgets we'll be needing,
so the natural next step is to define a kind of package which
uses these gadgets.

\begin{definition}[Asynchronous Packages]
  An \emph{asynchronous} package $P$ is a package which uses the additional
  syntax from Definition~\ref{def:yield} and Syntax~\ref{syn:await},~\ref{syn:select},~\ref{syn:wait}.

  $\square$
\end{definition}

Note that our syntax sugar definitions means that whenever one of the constructs
such as yield and what not are used, they are immediately replaced with their underlying
semantics. Thus, an asynchronous package \emph{literally} is a package which 
does not use any of those syntactical constructs.
Naturally, the definitions of $\circ$ and $\otimes$ for packages also
generalize directly to asynchronous packages.

\subsection{Defining Systems}

Our next goal is to define systems, by first defining channels,
and then giving them meaning in terms of asynchronous packages.
We'll then define various composition operations for systems,
and show that they satisfy similar properties to those of packages.

Our first task is defining channels.
We start by just defining some syntax for using channels,
and defer defining the precise meaning of this syntax until later. 

\begin{syntax}[Channels]
  Using channels involves two syntactic constructs:
  \begin{enumerate}
    \item $m \Rightarrow P$, for sending a message $m$ on a channel $P$,
    \item $m \Leftarrow P$, for receiving a message $m$ on a channel $P$,
    \item $n \gets \txbf{test } P$, for checking how many messages are on a channel $P$.
  \end{enumerate}

  $\square$
\end{syntax}

Like with functions, channels have distinct names.
The two fundamental operations are sending messages, and receiving messages.
We also add an operation for testing how many messages are waiting on a channel.
This is useful to allow a package to change its behavior based on whether
or not a channel is empty, in which case $\txbf{test } P$ will return $0$.
We consider testing to be a kind of operation that a system
can do on the channels it's allowed to receive on.

Next, we need to give packages the ability to use these channels.
We call these, \emph{systems}.

\begin{definition}[Systems]
A \emph{system} is a package which uses channels.

We denote by $\text{InChan(S)}$ the set of channels the system receives on,
or uses $\txbf{test}$ on,
and $\text{OutChan(S)}$ the set of channels the system sends on,
and define
$$
\text{Chan}(S) := \text{OutChan}(S) \cup \text{InChan}(S)
$$

Additionally we require that $\text{OutChan}(S) \cap \text{InChan}(S) = \emptyset$

$\square$
\end{definition}

We also define shorthands $\tx{Chan}(A, B, \ldots) = \tx{Chan}(A) \cup \tx{Chan}(B) \cup \ldots$,
and similarly for $\tx{InChan}$ and $\tx{OutChan}$.
The set of channels can be seen as another part of the interface
of a system.
A package has input and output functions, while a system additionally
has input and output channels.
Like with packages, this set is often implicit, based on whatever
channels the system happens to use syntactically.
We can also consider a system to be ``using'' channels that don't
actually appear in the body of a package as well.

So far, we've defined what systems are, but we haven't formally
defined what their semantics actually are,
although we might already have some intuition, at this stage.
The simplest way of defining the semantics of a system
is to compile it down into an asynchronous package,
which we developed a well defined meaning for.

\begin{definition}
We can compile systems to not use channels.
We denote by $\text{NoChan}(S)$ the asynchronous package corresponding to
a system $S$, with the use of channels replaced with function calls.

Channels define three new syntactic constructions, for sending and receiving
along a channel, along with testing how many messages are in a channel.
We replace these with external function calls as follows:


Sending, with $m \Rightarrow P$ becomes:
$$
\text{Channels}.\tx{Send}_P(m)
$$

Testing, with $n \gets \txbf{test } P$ becomes
$$
n \gets \tx{Channels}.\tx{Test}_P()
$$

Receiving, with $m \Leftarrow P$ becomes:
$$
m \gets \await \text{Channels}.\tx{Recv}_P()
$$
Receiving is an asynchronous function, because the channel might not have
any available messages for us.

These function calls are parameterized by the channel, meaning
that that we have a separate function for each channel.

$\square$
\end{definition}

This definition makes reference to external functions,
so we need to define a package providing these functions.
We do so in Game~\ref{game:Channels},
via the $\tx{Channels}$ package.

\begin{game}{game:Channels}{Channels}
\package{Channels($\{A_1, \ldots, A_n\}$)}{
&q[A_i] \gets \text{FifoQueue.New()}\cr
\cr
&\underline{\tx{Send}_{A_i}(m)\tx{:}}\cr
\pind{1} q[A_i].\tx{Push}(m)\cr
\cr
&\underline{\tx{Test}_{A_i}()\tx{:}}\cr
\pind{1} \txbf{return } q[A_i].\tx{Length}()\cr
\cr
&\underline{\tx{Recv}_{A_i}()\tx{:}}\cr
\pind{1} \txbf{while } q[A_i].\tx{IsEmpty()}\cr
\pind{2} \txbf{yield}\cr
\pind{1} q[A_i].\tx{Next}()\cr
}
\end{game}

Basically, this game has a queue for each channel, and then provides
the functions need to send, receive, and test that channel.
One consequence of defining separate functions for each channel
is that $\text{Channels}(S) \otimes \text{Channels}(R) = \text{Channels}(S \cup R)$.

Armed with the syntax sugar for channels, and the $\tx{Channels}$ game,
we can convert a system $S$ into a package via:
$$
\text{SysPack}(S) := \text{NoChan}(S) \circ (\text{Channels}(\tx{Chan}(S)) \otimes 1(\tx{In}(S)))
$$
This package will have the same input and output functions as the system $S$,
but with the usage of channels replaced with actual semantics.

At this point, we can also define a notion of efficiency for systems.

\begin{definition}[Efficient Systems]
  A system $S$ is said to be \emph{efficient} if $\tx{NoChan}(S)$
  is an efficient package.

  $\square$
\end{definition}

Note that we use $\tx{NoChan}$ rather than $\tx{SysPack}$,
because this captures the fact that a system only needs to be efficient
provided that sending and receiving on channels responds efficiently.
Unless otherwise specified, we only consider \emph{efficient} systems
from here on.

Our next steps will be defining the basic operations we can use
to compose systems, along with some notions of equality we
can use to compare systems.
The first notion of equality we want to define is the strongest
one, \emph{literal equality}, 
which we'll use to define fundamental properties of our composition
operations, like associativity, commutativity, and so on.

First, we need to define a notion of \emph{shape}, like we did
for packages, since our various equality relations will
require the systems to have the same shape.

\begin{definition}
  Given systems $A, B$, we say that they have the same \emph{shape} if
  \begin{itemize}
    \item $\tx{In}(A) = \tx{In}(B)$,
    \item $\tx{Out}(A) = \tx{Out}(B)$,
    \item $\tx{InChan}(A) = \tx{InChan}(B)$,
    \item $\tx{OutChan}(A) = \tx{OutChan}(B)$.
  \end{itemize}

  $\square$
\end{definition}

This is what you might expect, the functions and channels need to all
match for two systems to be considered to have the same shape.

Next, we can define the most basic notion of equality for systems.

\begin{definition}[Literal System Equality]
  Given systems $A$, $B$ with the same shape, we say that they are \emph{literally} equal, written $A \equiv B$ if
  $$
  \tx{NoChan}(A) = \tx{NoChan}(B)
  $$

  $\square$
\end{definition}

This is a very strong notion of equality, which doesn't take
into account the semantics of channels in practice.
Basically, it requires that regardless of what messages the channels
might start out with, or even what the semantics of channels are,
that the behavior is identical.
This is good enough for fundamental properties of our composition
operations, but we'll move on to a looser notion for standard equality
later, like we did with packages.

\subsection{Composing Systems}

Now, we move on to define the various ways to compose systems together.
Naturally, we can compose systems together like we did for packages
by having one system call the functions provided by another,
or having two systems used together independently,
but we also want to compose systems so that they can communicate
with each other using channels.

It's this kind of composition, allowing for communication across
channels, that we define first, and call \emph{tensoring}.

\begin{definition}[System Tensoring]
Given two systems, $A$ and $B$, with \(\text{Out}(A) \cap \text{Out}(B) = \emptyset\), we can define their tensoring $A * B$,
which is any system $A * B$ satisfying:
\begin{itemize}
  \item $\tx{NoChan}(A * B) = \tx{NoChan}(A) \otimes \tx{NoChan}(B)$,
  \item $\tx{InChan}(A * B) = \tx{InChan}(A) \cup \tx{InChan}(B)$,
  \item $\tx{OutChan}(A * B) = \tx{OutChan}(A) \cup \tx{OutChan}(B)$,
  \item $\tx{In}(A * B) = \tx{In}(A) \cup \tx{In}(B)$.
\end{itemize}

$\square$
\end{definition}

Note that combining the definition above with the definition of $\text{SysPack}$
means that:
$$
\tx{SysPack}(A * B) =
\begin{pmatrix}
\tx{NoChan}(A)\cr
\otimes\cr
\tx{NoChan}(B)\cr
\end{pmatrix}
\circ
\begin{pmatrix}
\text{Channels}(\tx{Chan}(A) \cup \tx{Chan}(B))\cr
\otimes\cr
1(\tx{In}(A) \cup \tx{In}(B))\cr
\end{pmatrix}
$$

The intuition for this definition is that tensoring is like $\otimes$
for packages, except that now the systems can interact by exchanging
messages.
This interaction only happens through the fact that they
share a common $\tx{Channels}$ package, which well then store
the messages sent by one system, so that the other can receive them,
and vice versa.

We can also gain some confidence in the quality of this definition
by proving that it's both associative and commutative.

\begin{lemma}
System tensoring is associative, i.e. $A * (B * C) \equiv (A * B) * C$.
\txbf{Proof:} This follows directly from the associativity
of $\otimes$ for packages and $\cup$.

$\blacksquare$
\end{lemma}

\begin{lemma}
System tensoring is commutative, i.e. $A * B \equiv B * A$

\txbf{Proof:} This follows from the commutativity of $\otimes$ and $\cup$.

$\blacksquare$
\end{lemma}

We've also made our lives quite easy, by defining literal
equality in terms of $\tx{NoChan}$, so we can lean
heavily on the work we did in proving that package tensoring
is associative and commutative.

In many situations, we'll have systems that don't actually share
any channels, and we'll want to compose them as well, while
benefiting from some nicer properties.

We define this situation formally.

\begin{definition}[Overlapping Systems]
Two systems $A$ and $B$ overlap if $\text{Chan}(A) \cap \text{Chan}(B) \neq \emptyset$.

In the case of non-overlapping systems, we write $A \otimes B$ instead of $A * B$,
insisting on the fact that they don't communicate.

$\square$
\end{definition}

One very common way this situation arises is if a system doesn't
use any channels at all.
For example, we might write $A \otimes 1(\ldots)$,
since $1(\ldots)$ can be considered as a system with no use of channels,
and so won't overlap with $A$.
This is why we can see $*$ as the natural generalization of $\otimes$
for systems, because it literally becomes $\otimes$ when used
for systems that do not use channels.

Next, we define the analogue of package composition for systems,
which allows one system to use the functions provided by the other.

\begin{definition}[System Composition]
Given two systems, $A$ and $B$, we can define their (horizontal) composition
$A \circ B$ as any system, provided a few constraints hold:
\begin{itemize}
\item $A$ and $B$ do not overlap ($\text{Chan}(A) \cap \tx{Chan}(B) = \emptyset$)
\item $\tx{In}(A) \subseteq \tx{Out}(B)$
\end{itemize}

With these in place, we define the composition as any system $A \circ B$ such that:
\begin{itemize}
  \item $\tx{NoChan}(A \circ B) = \tx{NoChan}(A) \circ \begin{pmatrix}
    \tx{NoChan}(B)\cr
    \otimes\cr
    1(\tx{Channels}(\tx{Chan}(A)))
  \end{pmatrix}
    $,
  \item $\tx{InChan}(A \circ B) = \tx{InChan}(A) \cup \tx{InChan}(B)$,
  \item $\tx{OutChan}(A \circ B) = \tx{OutChan}(A) \cup \tx{OutChan}(B)$,
  \item $\tx{In}(A \circ B) = \tx{In}(B)$.
\end{itemize}

$\square$
\end{definition}

It's very important that the systems do not overlap.
Our intention with system composition is that the two systems
interact only via the functions that one system provides to the other,
and not via any channels.
This is like the machine analogy we had earlier,
where machines within a row communicate across channels,
but are only connected via functions to the rows behind them.

As one might expect, this definition of composition is also associative. 

\begin{lemma}
System composition is associative, i.e. $A \circ (B \circ C) \equiv (A \circ B) \circ C$.

\txbf{Proof:} This follows from the associativity of $\circ$ for \emph{packages}.

$\blacksquare$
\end{lemma}

We've now defined tensoring and system composition,
and are in the same position as with packages, in that we need
some way of characterizing how these operations behave together,
so that we can do the various manipulations we need inside proofs.

Thankfully, Lemma~\ref{thm:package_interchange} (interchange) generalizes
to systems as well, allowing us to reason in the same way
as we can for packages.

\begin{lemma}[Interchange Lemma]
\label{thm:interchange_system}
Given systems $A, B, C, D$ such that $\tx{In}(A) \cap \tx{Out}(D) = \emptyset$,
and $\tx{In}(C) \cap \tx{Out}(B) = \emptyset$,
and neither $A$ nor $C$ overlap with $B$ or $D$,
the following relation holds:
$$
\begin{pmatrix} 
A\cr
*\cr
C\cr
\end{pmatrix} 
\circ
\begin{pmatrix} 
B\cr
*\cr
D\cr
\end{pmatrix} 
\equiv
\begin{matrix} 
  (A \circ B)\cr
*\cr
  (C \circ D)\cr
\end{matrix} 
$$
provided these expressions are well defined.

\txbf{Proof:} $\tx{InChan}$, $\tx{OutChan}$, and $\tx{In}$ are equal for
both of these systems, by associativity of $\cup$.
We now look at $\tx{NoChan}$.
Starting with the right hand side, we get:
$$
\tx{NoChan}
\begin{pmatrix} 
  (A \circ B)\cr
*\cr
  (C \circ D)\cr
\end{pmatrix} 
= 
\begin{pmatrix}
  \tx{NoChan}(A \circ B)\cr
  \otimes\cr
  \tx{NoChan}(C \circ D)
\end{pmatrix}
=
\begin{pmatrix}
    \tx{NoChan}(A) \circ
    \begin{pmatrix}
      \tx{NoChan}(B)\cr
      \otimes\cr
      1(\tx{Channels}(\tx{Chan}(A)))
    \end{pmatrix}
  \cr
  \otimes\cr
    \tx{NoChan}(C) \circ
    \begin{pmatrix}
      \tx{NoChan}(D)\cr
      \otimes\cr
      1(\tx{Channels}(\tx{Chan}(C)))
    \end{pmatrix}
\end{pmatrix}
$$
Next, apply the interchange lemma for packages, to get:
$$
\begin{pmatrix}
  \tx{NoChan}(A)\cr
  \otimes\cr
  \tx{NoChan}(C)
\end{pmatrix}
\circ
\begin{pmatrix}
  \tx{NoChan}(B)\cr
  \otimes\cr
  1(\tx{Channels}(\tx{Chan}(A)))\cr
  \otimes\cr
  \tx{NoChan}(D)\cr
  \otimes\cr
  1(\tx{Channels}(\tx{Chan}(C)))\cr
\end{pmatrix}
$$
Then, observe that:
$$
\tx{Channels}(S_1 \cup S_2) = \tx{Channels}(S_1) \otimes \tx{Channels}(S_2)
$$
We can use this, along with the commutativity of $\otimes$ to get:
$$
\begin{pmatrix}
  \tx{NoChan}(A)\cr
  \otimes\cr
  \tx{NoChan}(C)
\end{pmatrix}
\circ
\begin{pmatrix}
  \tx{NoChan}(B)\cr
  \otimes\cr
  \tx{NoChan}(D)\cr
  \otimes\cr
  1(\tx{Channels}(\tx{Chan}(A * C)))\cr
\end{pmatrix}
$$
Which is just:
$$
\tx{NoChan}
\left(
\begin{pmatrix} 
A\cr
*\cr
C\cr
\end{pmatrix} 
\circ
\begin{pmatrix} 
B\cr
*\cr
D\cr
\end{pmatrix} 
\right)
$$

$\blacksquare$
\end{lemma}

This lemma plays the same critical role as it did for packages,
and we'll be applying it quite often throughout the rest
of this work, starting even from the next subsection.

\subsection{System Equality and Indistinguishability}

Next, we define some looser notions of equality for systems,
like those we defined for packages,
and then show that the various operations we've defined
respect the notions of equality, with one exception.

First, we define the standard notion of equality we'll be using.

\begin{definition}[System Equality]
  We say that two systems $A$, $B$ with the same shape are equal, written $A = B$,
  if:
  $$
  \tx{SysPack}(A) = \tx{SysPack}(B)
  $$

  $\square$
\end{definition}

This is the natural definition of equality, since $\tx{SysPack}$
tries and capture the actual semantics of a system.
This, comparing two systems using $\tx{SysPack}$ allows
us to compare the behavior of the two systems,
disregarding inessential details, and actually
looking at how the use of channels affects their behavior.

If we can compare systems for equality using $\tx{SysPack}$,
we should also be able to compare them for indistinguishability
in the same way.

\begin{definition}[System Indistinguishability]
  We say that two systems $A$, $B$ with the same shape are indistinguishable up to $\epsilon$,
  written $A \overset{\epsilon}{\approx} B$,
  if:
  $$
  \tx{SysPack}(A) \overset{\epsilon}{\approx} \tx{SysPack}(B)
  $$

  $\square$
\end{definition}

This allows for small differences that a bounded adversary can't notice
to pop up, and this is the notion of equality that we'll
target most often in proofs.

We've seen three notions of equality so far, but we haven't commented
that much on how well behaved they are.
Thankfully, they all satisfy all the properties we'd
expect from an equality relation, including transitivity,
which we prove here explicitly.

\begin{lemma}[Transitivity of System Equality]
  \label{thm:system_trans}
  Given systems $A, B, C$, we have:
  \begin{enumerate}
    \item $A \equiv B, B \equiv C \implies A \equiv C$,
    \item $A = B, B = C \implies A = C$,
    \item $A \overset{\epsilon_1}{\approx} B, B \overset{\epsilon_2}{\approx} C \implies A \overset{\epsilon_1 + \epsilon_2}{\approx} C$.
  \end{enumerate}
  provided these expressions are well-defined.

  \txbf{Proof:} This follows immediately from the fact that equality and indistinguishability
  for \emph{packages} satisfy these relations, and the notions for systems are
  defined in terms of $\tx{NoChan}$ or $\tx{SysPack}$.

  $\blacksquare$
\end{lemma}

Next, we need to prove whether or not our various operations respect
these notions of equality, like we did for packages.
This is very useful, since it allows using the characteristic
modular proofs that we have for packages in the context of systems.
We can break down a large package into smaller components,
and then appeal to the indistinguishability of those small
components alone, in order to make an argument about the system
as a whole.

The first operation we target is composition.

\begin{lemma}[Composition Compatability]
  \label{lemma:system_composition}
  Given systems $A$, $B$, $B'$, we have:
  \begin{enumerate}
    \item $B = B' \implies A \circ B = A \circ B'$,
    \item $B \overset{\epsilon}{\approx} B' \implies A \circ B \overset{\epsilon}{\approx} A \circ B'$.
  \end{enumerate}
  provided these expressions are well-defined.

  \txbf{Proof:} We prove that
  $$
  \tx{SysPack}(A \circ B) = \tx{SysPack}(A) \circ \tx{SysPack}(B)
  $$
  which then clearly implies this lemma by application of the similar properties
  for packages.

  We start with:
  $$
  \tx{SysPack}(A \circ B) = \tx{NoChan}(A) \circ
  \begin{pmatrix}
    \tx{NoChan}(B)\cr
    \otimes\cr
    1(\tx{Channels}(\tx{Chan}(A)))
  \end{pmatrix}
  \circ
  \begin{pmatrix}
    \tx{Channels}(\tx{Chan}(A) \cup \tx{Chan}(B))\cr
    \otimes\cr
    1(\tx{In}(B))
  \end{pmatrix}
  $$
  We then use the fact that $\tx{Channels}(S \cup R) = \tx{Channels}(S) \otimes \tx{Channels}(R)$,
  and the interchange lemma, to get:
  $$
  \tx{NoChan}(A) \circ
  \begin{pmatrix}
    \tx{NoChan}(B)\cr
    \otimes\cr
    \tx{Channels}(\tx{Chan}(A))
  \end{pmatrix}
  \circ
  \begin{pmatrix}
    \tx{Channels}(\tx{Chan}(B))\cr
    \otimes\cr
    1(\tx{In}(B))
  \end{pmatrix}
  $$
  Apply interchange once more, to get:
  $$
  \tx{NoChan}(A) \circ
  \begin{pmatrix}
    1(\tx{In}(A))\cr
    \otimes\cr
    \tx{Channels}(\tx{Chan}(A))
  \end{pmatrix}
  \circ
    \tx{NoChan}(B)
  \circ
  \begin{pmatrix}
    \tx{Channels}(\tx{Chan}(B))\cr
    \otimes\cr
    1(\tx{In}(B))
  \end{pmatrix}
  $$
  Which is none other than:
  $$
  \tx{SysPack}(A) \circ \tx{SysPack}(B)
  $$
  concluding our proof.

  $\blacksquare$
\end{lemma}

This proof is made relatively simple by being able to 
appeal to the work we already did in proving the analogous property
for packages.

Next, we look at strict tensoring, for systems that do not overlap.

\begin{lemma}[Strict Tensoring Compatability]
  Given systems $A$, $B$, $B'$, we have:
  \begin{enumerate}
    \item $B = B' \implies A \otimes B = A \otimes B'$,
    \item $B \overset{\epsilon}{\approx} B' \implies A \otimes B \overset{\epsilon}{\approx} A \otimes B'$.
  \end{enumerate}
  provided these expressions are well-defined.

  \txbf{Proof:} Similar to Lemma~\ref{lemma:system_composition},
  we start by proving:
  $$
  \tx{SysPack}(A \otimes B) = \tx{SysPack}(A) \otimes \tx{SysPack}(B)
  $$
  which then entails our theorem through similar properties for packages.

  Our starting point is:
  $$
  \tx{SysPack}(A \otimes B) =
  \begin{pmatrix}
    \tx{NoChan}(A)\cr
    \otimes\cr
    \tx{NoChan}(B)
  \end{pmatrix}
  \circ
  \begin{pmatrix}
    \tx{Channels}(\tx{Chan}(A) \cup \tx{Chan}(B))\cr
    \otimes\cr
    1(\tx{In}(A), \tx{In}(B))
  \end{pmatrix}
  $$
  We can write this as:
  $$
  \begin{pmatrix}
    \tx{NoChan}(A)\cr
    \otimes\cr
    \tx{NoChan}(B)
  \end{pmatrix}
  \circ
  \begin{pmatrix}
    \tx{Channels}(\tx{Chan}(A))\cr
    \otimes\cr
    1(\tx{In}(A))\cr
    \otimes\cr
    \tx{Channels}(\tx{Chan}(B))\cr
    \otimes\cr
    1(\tx{In}(B))\cr
  \end{pmatrix}
  $$
  Crucially, we can use the fact that $A$ and $B$ do not overlap,
  in order to apply the interchange lemma, giving us:
  $$
  \begin{matrix}
    \tx{NoChan}(A) \circ \begin{pmatrix}
    \tx{Channels}(\tx{Chan}(A))\cr
    \otimes\cr
    1(\tx{In}(A))\cr
    \end{pmatrix}\cr
    \otimes\cr
    \tx{NoChan}(B) \circ \begin{pmatrix}
    \tx{Channels}(\tx{Chan}(B))\cr
    \otimes\cr
    1(\tx{In}(B))\cr
    \end{pmatrix}
  \end{matrix}
  $$
  Which is none other than:
  $$
  \tx{SysPack}(A) \otimes \tx{SysPack}(B)
  $$
  concluding our proof.

  $\blacksquare$
\end{lemma}

The assumption that the systems do not overlap is in fact essential,
because this lemma \emph{does not} hold for tensoring in general.

Here's some intuition for a counter example.
The idea is that you can insert a back door into a package
by having a channel which is never sent on.
The back door is triggered if the package can successfully receive
from this channel.
If the use of this back door allows distinguishing two packages,
then in isolation they will be equal, since it's not
possible to trigger a message being sent to open the back door.
However, when composed with another package,
that package might be able to unlock the door by sending a message,
and thus the composed system can be distinguishable again.
