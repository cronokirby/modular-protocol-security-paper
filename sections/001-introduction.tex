\section{Introduction}

Universally composable (UC) security \cite{EPRINT:Canetti00}
has seen widespread success since its introduction over
two decades ago,
becoming the dominant framework for analyzing the security of cryptographic
protocols.

This success is understandable, because the guarantees
the framework provides are very useful.
In essence, when a protocol is proved secure in the UC framework,
then arbitrary instances of that protocol can be used concurrently in the context
of a larger system, without modifying the behavior
of these protocols.
This allows the study of isolated components to be used
to guarantee the security of a system as a whole;
this modular approach is essential to being able to scale
formal analysis to larger systems.

However, the framework is not ubiquitous.
For many cryptographic schemes, standalone security with games
is a more suitable approach.
Even for some protocols, the use of game-based security remains popular.
The analysis of threshold signatures and messaging protocols
has seen the development of increasingly intricate games,
which describe all the ways in which an adversary can attack a protocol.
This is the main disadvantage of game-based security for protocols:
it's not always clear what the ``right'' game is to analyze
the security of a protocol, and these games often need to provide
explicit capabilities to the adversary.

This problem is alleviated somewhat in UC security.
The security goals of a protocol are described by
an ``ideal functionality'',
which represents how the protocol could function if one had access
to a perfectly trusted third party.
This makes it easier to determine if a given notion of security
reflects the kinds of attacks that need to be analyzed.

The game-based approach is still sometimes preferred because of
its perceived ease of use as compared to UC proofs.
Some of these difficulties are inherent:
because UC security provides stronger guarantees than standalone
security, it's not surprising that proofs would involve more ``work''.

Some of these difficulties don't seem to be inherent though,
which is why a series of works have provided improvements,
simplifications, and variants of UC security.
GNUC \cite{JC:HofSho15} was an early variant,
simplifying many aspects of UC, and also patching several
foundational gaps present in the paper at the time.

One disadvantage of developing a new framework is that
proofs in one framework may not necessarily
or automatically translate to UC proofs.
One approach to addressing this is to develop
a ``higher level'' language for simpler proofs,
which is then compiled down to an actual UC proof.
This was done in \cite{C:CanCohLin15},
which provided a simplified version of UC, suitable
for the common setting of multiparty computation,
but also a way to interpret proofs in this simplified
model as actually being UC proofs.

Another interesting alternative to UC is that presented in
\cite{cramer2015secure}.
This approach defines a kind of UC security in terms of
a calculus for \emph{interactive systems},
and their composition.
This is an interesting departure from the interactive Turing machine
foundations, and does away with many inessential details.
This approach is the most similar one to the framework we develop
in this work.

In practice, UC proofs are often quite informal,
without explicitly mentioning the various details that the formalism
might require.
For example, the framework might specify protocols
in terms of interactive Turing machines, but in practice,
proofs are written with an informal description of what the protocol
does.
We think that this informality actually
makes proofs harder to write and understand,
because it isn't clear what exactly a proof can consist of,
nor what certain informal patterns mean precisely.

In this work, we propose a new framework for analyzing the security
of protocols, which we believe to be both less informal,
but also simpler to understand and use.

\subsection{Our Framework}

Our framework---which we call, ``modular protocol security'',
or ``\shortname''---attempts to provide a simple, high-level language
for reasoning about protocol security, while also having
a well-defined formal model that is close to how proofs
are actually written.

\shortname{} tries to be \emph{modular}, in the sense that
large proofs for complicated protocols can be built up from smaller proofs
for simpler protocols.

The first way we try and achieve this is by allowing
modular specifications of protocols: being able to describe a protocol
as the composition of other protocols.
The two fundamental operations we have are:
\begin{enumerate}
\item tensoring, written $\mathscr{P} \otimes \mathscr{Q}$,
which allows writing a protocol as involving two distinct protocols
running at the same time,
\item and composition, written $\mathscr{P} \lhd \mathscr{Q}$,
which allows the participants in one protocol to use another
as a kind of ``sub-protocol'', in which each player may play
several roles.
\end{enumerate}
These operations can simplify proofs by allowing large protocols
to be decomposed into smaller components,
and then for security to be argued component by component.

The second approach is to have the
notion of simulation be between \emph{protocols},
rather than between a protocol and an ideal functionality,
as in UC security.
The ultimate goal is to prove that a protocol can be simulated
by some clear ideal functionality,
but this often requires writing a complicated simulator
which can achieve this large jump all at once.
By allowing protocols to be simulated by other protocols,
we can transform this jump into several smaller hops,
which are then composed together.
This can help break down a large proof into a series of simpler proofs.

MPS builds on state-separable proofs \cite{AC:BDFKK18},
a recent framework for standalone security with games.
Our work can be seen as an attempt to lift the modular
properties of this framework for games into a modular
framework for protocol security.
Ultimately, the semantics of protocols will be defined in terms
of state-separable games.

This provides an interesting advantage, in that proofs
and techniques using games can be used to reason
about the security of protocols.
This can also help motivate complicated games used in the analysis
of protocols, which can be seen as being related to protocols
written in this framework.

We also take the opportunity to present state-separable proofs
in a more formal way, filling in several proofs left as sketches
in the original paper.
Instead of using interactive Turing machines as our foundational
object, like in UC security,
we instead simply assume the existence of computable randomized functions,
and some pseudo-code to describe them.

\subsection{Overview}

Here we provide a basic overview of the rest of this work.

In Section 2, we develop the framework of state-separable proofs
for standalone security from scratch, filling some gaps
left in the original paper, and proving a few additional properties
that we'll be needing in the rest of this work.

In Section 3, we generalize state-separable games into \emph{systems},
which are a kind of game that has the ability to communicate
with other games by sending and receiving messages along a channel.
We also need to develop a notion of \emph{asynchronous} games
in this section, allowing us to model games which can delay
answers to queries, if they're not yet ready to provide them.
This arises naturally with channels, since a game may be waiting
to receive a message along a channel before being able to respond.

In Section 4, we use the newly developed notion of systems
to define \emph{protocols}.
We then define various ways to compose protocols together,
then notions of equality and simulation for protocols,
before showing that the various ways of composing protocols
behave well with respect to equality and simulation,
allowing us to decompose proofs in a modular fashion.
We also describe how to incorporate global functionalities,
such as a common random oracle, into our framework.

In Section 5, we provide a couple of examples of proofs written
in our framework.
Our first example constructs a private channel from one
that leaks all messages sent over it, by using a public key encryption scheme.
Our second example constructs a protocol for sampling
an unbiased random value, using the common ``commit reveal'' paradigm.


In Section 6, we compare the MPS framework we've developed with
that of UC security, illustrating several differences
between the two approaches.
