\section{State-Separable Proofs}

Our framework for describing protocols is based on 
\emph{state-separable proofs}
\cite{AC:BDFKK18}.
The security notions we develop for protocols ultimately
find meaning in analogous notions of security for \emph{packages},
the main object of study in state-seperable proofs.

This section is intended to be a suitable independent presentation
of this formalism.
In that spirit, we develop state-separable proofs ``from scratch''.
Our starting point is merely that of computable randomized functions.
This is in contrast to other protocol security frameworks like UC,
whose foundational starting point is usually the more concrete notion
of \emph{interactive turing machines}.

We also take the opportunity to solidify the formalism of state-separable
proofs, providing more complete definitions of various objects,
completing several proofs left as mere sketches in the original paper,
and proving a few additional properties we'll need later.
This makes this section of interest to readers who are already familiar
with state-separable proofs.

\subsection{Some Notational Conventions}

We write $[n]$ to denote the set $\{1, \ldots, n\}$.

We write $\txt{01}$ to denote the set $\{0, 1\}$,
and write $\str$ to denote binary strings.
We write $\bullet$ to denote the empty string,
which also serves as a ``dummy'' value in various contexts.

\subsection{Probabilistic Functions}

Our starting point is the notion of \emph{randomized computable functions}.
This is a notion we assume can be defined in a rigorous way, but whose
concrete semantics we don't assign.
We write $f : \randf{\str}{\str}$ to denote such a function (named $f$).
Intuitively, this represents a function described by some algorithm,
which takes in a binary string as an input, and produces a binary string
as output, and is allowed to make randomized decisions to aid its computation.

We mainly consider \emph{families} of functions,
parametrized by a security parameter $\lambda$.
Formally, this is in fact a function $f : \mathbb{N} \to \randf{\str}{\str}$,
and we write $f_{\lambda} : \randf{\str}{\str}$ to denote a particular
function in the family.
In most cases, this security parameter is left \emph{implicit}.
In fact, all of the objects we consider from here on out will \emph{implicitly}
be \emph{families} of objects, parametrized by a security parameter $\lambda$,
and we will invoke this fact only as necessary.

\begin{definition}[Efficient Functions]
    We assume that a function $f$ has a runtime, denoted $T(f, x)$,
    measuring how long the function takes to execute on a given input $x \in \str$.

    We say that a function family is \emph{efficient} if:
    $$
    \forall \lambda.\ \forall x, |x| \in \mathcal{O}(\tx{poly}(\lambda)).\quad T(f_\lambda, x) \in \mathcal{O}(\tx{poly}(\lambda))
    $$
    In other words, the runtime is always polynomial in $\lambda$, regardless
    of the input, or the random choices of the function.

    $\square$
\end{definition}

Functions which are not necessarily efficient are said to be \emph{unbounded}.

Considering efficient functions is essential, because the vast majority
of cryptographic techniques depend on assuming that some problems are ``hard''
for adversaries with bounded computational resources,
and so this notion of efficiency is critical to defining game-based
security.
Ironically, for protocol security, many protocols can be proved secure
without this restriction.

Another crucial notion we need to develop is that of a \emph{distance},
measuring how different two functions behave.
This will underpin our later notion of security for games,
which is based on saying that two different games are difficult
to tell apart.

\begin{definition}[Distance Function]
    Given a function $f : \randf{\bullet}{\bin}$, we assume that the probability
    $P[f \to 1]$ of the function returning $1$ on the input $\bullet$ is well defined.

    Given two functions $f, g$, we define their distance $\varepsilon(f, g)$ as:
    $$
    \varepsilon(f, g) := |P[f \to 1] - P[g \to 1]|
    $$

    $\square$
\end{definition}

In other words, the distance looks at how often one function returns $1$
compared to the other.
If the functions agree most of the time, then their distance will be small,
whereas if they disagree very often, their distance will be large.
This definition is actually quite natural.
Since $P[f \to 1] = (1 - P[f \to 0])$, $\varepsilon$ is actually
just the total variation---or statistical---distance.
This immediately implies that this distance has some nice properties,
in particular that it forms a \emph{metric}.

\begin{lemma}[Distance is a Metric]
    $\varepsilon$ is a valid metric, in particular, it holds
    for any functions $f, g, h$, that:
    \begin{enumerate}
        \item $\varepsilon(f, f) = 0$,
        \item $\varepsilon(f, g) = \varepsilon(g, f)$,
        \item $\varepsilon(f, h) \leq \varepsilon(f, g) + \varepsilon(g, h)$.
    \end{enumerate}

    \txbf{Proof:}

    \txbf{1.} Follows from the fact that $P[f \to 1] = P[f \to 1]$,
    so $\varepsilon(f, f) = 0$.

    \txbf{2.} Follows from the fact that $|a - b| = |b - a|$.

    \txbf{3.} Follows from the triangle inequality for $\mathbb{R}$
    and the fact that:
    $$
    |P[f \to 1] - P[h \to 1]| = |(P[f \to 1] - P[g \to 1]) + (P[g \to 1] - P[H \to 1])|
    $$

    $\blacksquare$
\end{lemma}

We actually skipped one property in our proof that $\varepsilon$ is a valid metric,
which requires that if $f \neq g$, then $\varepsilon(f, g) > 0$.
This is because we haven't yet defined what equality should mean
for functions.
This metric property gives us a very natural definition though.

\begin{definition}[Function Equality]
    Two functions, $f$ and $g$, are \emph{equal}, written $f = g$, when:
    $$
    \varepsilon(f, g) = 0
    $$

    $\square$
\end{definition}

It's easy to see that this is an equality relation, satisfying
reflexivity, symmetry, and transitivity.

While the functions we've considered so far only manipulate binary strings,
it's useful to allow \emph{typed} functions,
with richer input and output types.
This could be defined in several ways, but the end result means
that a typed function $f : \randf{A}{B}$ can be interpreted as a function
over binary strings, using a suitable encoding and decoding mechanism,
as well as perhaps having a special output value that $f$ can return
if it fails to decode its input successfully.

Being able to quantify types is also useful for the formalism itself,
and potentially even for some packages.
This allows us to type functions such as:
$$
\begin{aligned}
    &\tx{id} : \forall s.\ s \to s\cr
    &\tx{id} = x \mapsto x
\end{aligned}
$$
In this example, $s$ is a quantified type variable.
Formally, we can see $\tx{id}$ as a function parametrized by a type,
with $\tx{id}_S$ being a concrete function, after having chosen this type.

\subsection{Packages}

Our next goal is to define the central object of state-separable proofs:
the \emph{package}.
Intuitively, a package has some kind of state, as well as functions
which manipulate this state.
You can interact with a package by calling the various output functions
it provides.
This makes packages a natural fit for security games.
What distinguishes packages from games is that they can have \emph{input}
functions.
A package can depend on another package, with each of its functions
potentially using the functions provided by this other package.
This modularity makes the common proof technique of ``game-hopping''
much more easily usable, and is the core strength of the state-separable
proof formalism.

Before we get to packages, we first need to define a few convenient
notions for functions manipulating a state, and parametrizing
functions with other functions.

Our first definition will be a little bit of shorthand.
\begin{definition}[Stateful Function]
    A \emph{stateful} function is simply a function $f$ of the form:
    $$
    f : \randf{(S, \str)}{(S, \str)}
    $$
    $S$ represents the state being used and modified by the function.
    As a convenient shorthand, we write:
    $$
    f :\ \stateful{S}
    $$

    $\square$
\end{definition}
It's useful to have a bit of typing to separate the state from the rest
of the input and output, since it allows us to avoid defining
inessential padding details inside the formalism itself.

We also need to consider functions parametrized by other functions.
Intuitively, this arises when one function calls another.
For example, consider:
$$
f(x) := g(x) \oplus g(x)
$$
which is well defined regardless of what $g$ is.
Here $f$ is implicitly parametrized by $g$, but we could write this explicitly
as $f(x) := g \mapsto g(x) \oplus g(x)$.
We could write $f : (\randf{\str}{\str}) \to (\randf{\str}{\str})$  
as a potential type in this example.
We write $f[g]$ for the instantiation of a parametrized function $f$
with an input function $g$.
It might also be the case that $g$ is itself parametrized,
in which case $f[g]$ is defined as:
$$
f[g] := h \mapsto f[g[h]]
$$

We've now developed enough tools to define packages.

\begin{definition}[Package]
    A package $A$ consists of:
    \begin{itemize}
        \item a type $S$, for its state,
        \item a set of \emph{input names} $\tx{In}(A)$, of size $m$,
        \item a permutation $\pi_{\tx{in}} : \tx{In}(A) \leftrightarrow [m]$,
        \item a set of \emph{output names} $\tx{In}(A)$, of size $n$,
        \item a permutation $\pi_{\tx{out}} : \tx{Out}(A) \leftrightarrow [m]$,
        \item a set of parametrized functions $f_1, \ldots, f_n : \forall s. \stateful{s}^m \to \stateful{(S, s)}$.
    \end{itemize}

    We also only consider a package to be defined \emph{up to} potentially
    renaming its input and output functions injectively.

    $\blacksquare$
\end{definition}

The basic idea is that a package has an internal state $S$, which gets
manipulated by each of the functions it exports.
These functions, in turn, can depend on other input functions.
If a stateful function $f$ uses a stateful function $g$,
then the result is a stateful function $f[g]$ manipulating 
\emph{both} the state of $f$, and the state of $g$.
Furthermore, $f$ is defined in such way agnostic to what the state manipulated
by $g$ happens to be,
which is why we use a \emph{quantified} type instead, to allow instantiation
with functions manipulating different kinds of state.

In practice, each function in a package is unlikely to use \emph{all}
of the input functions of the package, but it is much simpler
to have each function parametrized by all the possible inputs,
even if some are left unused.
It's also much simpler to define an ordering of the input functions
$\pi$, so be able to use $\stateful{s}^m$ as the input type for
the parametrized functions.

The semantics of a package without inputs are intuitively that of a stateful computer program
or machine you can interact with.
The machine has some kind of state, represented by $S$,
along with various functions you can call, represented by $f_1, \ldots, f_n$.
Each of these will use the input you provide, along with the current state
of the machine, in order to supply you with an output,
potentially modifying the state along the way.
The input functions allow a package to interact with other packages itself.

We describe this kind of interaction using the formal
notion of package \emph{composition}.

\begin{definition}[Package Composition]
    Given two packages $A, B$ with $\tx{In}(A) \subseteq \tx{Out}(B)$,
    we define their composition $A \circ B$ as a package characterized by:

    \begin{itemize}
        \item a state type $(A.S, B.S)$,
        \item input names $\tx{In}(B)$,
        \item output names $\tx{Out}(A)$,
        \item $\pi_{\tx{in}} := B.\pi_{\tx{in}}$,
        \item $\pi_{\tx{out}} := A.\pi_{\tx{out}}$,
        \item output functions $A.f_1[B.f_1, \ldots, B.f_{B.n}], \ldots$
        of type $\forall s.\stateful{s}^{B.m} \to \stateful{(A.S, B.S)}$.
    \end{itemize}

    In more detail, each output function is of the form:
    $$
    (h_1, \ldots, h_{B.m}) \mapsto A.f_i[B.f_1[h_1, \ldots], \ldots, B.f_n[h_1, \ldots]]
    $$
    with the same input functions $h_j$ being given
    to all the functions used by $A.f_i$.

    $\square$
\end{definition}

\subsection{Syntactical Conventions for Packages}