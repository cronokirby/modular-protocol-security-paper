\section{Additional Game Definitions}

In this section, we include explicit definitions of several games we
use throughout the rest of this work.
While we expect these notions to be familiar, we think the precise
details are worth spelling out here.

\subsection{Encryption}
\label{app:encryption}
An encryption scheme consists of types $\txbf{K}, \txbf{M}, \txbf{C}$,
along with probabilistic functions $\tx{Enc} : \txbf{K} \times \txbf{M} \xleftarrow{\$} \txbf{C}$ and $\tx{Dec} : \txbf{K} \times \txbf{C} \to \txbf{M}$.
By $\txbf{M}(|m|)$ we denote the distribution of messages with the same
length as $m$.
We require that $\txbf{K}$ and $\txbf{M}(|m|)$ are efficiently sampleable.

The encryption scheme must satisfy a correctness property:
$$
\forall k \in \txbf{K},\ m \in \txbf{M}.\ P[\tx{Dec}(\tx{Enc}(k, m)) = m] = 1
$$
Encrypting and then decrypting a message should return that same message.

The security of an encryption scheme can be captured by the following game:
\package{$\tx{IND-CPA}_b$}{
\pind{0} \draw{k}{\txbf{K}}\cr
\cr
&\begin{aligned}
    &\pfn{Challenge}{m_0}\cr
    \pind{1} \draw{m_1}{\txbf{M}(|m|)}\cr
    \pind{1} \preturn{\tx{Enc}(k, m_b)}\cr
\end{aligned}
&\begin{aligned}
    &\pfn{Enc}{m}\cr
    \pind{1} \preturn{\tx{Enc}(k, m)}\cr
    &\cr
\end{aligned}
}

In essence, an adversary cannot distinguish between an encryption of a message
of their choice and that of a random message, even if they can encrypt
messages at will.