\section{Differences with UC Security}

In this section, we outline a few differences between
the framework we've developed, and that of UC security
\cite{EPRINT:Canetti00}.
Despite these differences, we think that the frameworks
ultimately remain quite compatible, 
in that proofs in one framework should translate well
to proofs in the other.
This process is not, by any means, automatic,
as is the case for other variants of UC,
such as \cite{C:CanCohLin15}.

\subsubsection*{Foundations without Turing Machines}

One major technical difference is that our framework
doesn't specify a concrete computational model.
Rather than using interactive randomized turing machines,
as most frameworks do, we instead just
assume the existence of computable randomized functions,
and then build everything on top of that foundation.

We believe that this makes the foundations simpler to understand,
since the complicated details of turing machines and various
tapes are never mentioned, but it also
makes proofs closer to the actual formalism.

In principle, UC proofs would need to make reference
to interactive turing machines writing messages on each other's
tapes.
In practice, a much higher level language is used.
The advantage of basing ourselves on state-separable proofs
is that we can give a formal justification for this kind
of high-level language, by providing a precise semantics
for the pseudo-code we use.
Thus, we expect proofs in our framework to be writeable
in a style close to the formalism itself,
while also proving a high level of abstraction.

\subsubsection*{Semi-Honest Security without Randomness}

Another technical difference is that our notion of semi-honest
security does not allow an adversary to see the randomness
sampled by a given party.
Instead, they're allowed to see all function calls
and messages sent by the party.
As explained before, the main reason for this difference
is that we ultimately want two protocols
with equal parties to be consider equal protocols,
under any corruption model, and being able to see the
exact randomness being sampled is often enough to distinguish
otherwise equal parties.

In practice, we don't expect this difference to matter,
because meaningful randomness should affect the output
calls and behavior of the adversary,
and so the difference between these models are likely
to come from more pathological examples.

\subsubsection*{Hybrid Only}

In the usual presentation of UC security, simulation
happens between a protocol in the \emph{hybrid} world,
where parties can potentially interact with an ideal functionality,
and the \emph{ideal} world, where the parties
don't communicate between each other, instead interacting
only with the ideal functionality.

In our framework, only the hybrid world exists.
Protocols aren't simulated by ideal functionalities,
instead, protocols are simulated by other protocols,
and all protocols may make use of ideal functionalities.

The advantage of this approach is that it allows decomposing
a larger simulation proof into multiple smaller proofs,
which can then be stringed together via transitivity.
The larger the gap between the protocols being simulated,
the more complicated the simulator needs to be,
and so this style of proof can be much simpler.

\subsubsection*{Corruption Agnostic Ideal Functionalities}

Another technical difference is that in our framework,
ideal functionalities are not aware of which parties are corrupted,
whereas some UC functionalities make use of this fact.

We don't think this is a necessary feature of the framework itself,
since it can be modeled by having slightly more complicated
protocols on top of an ideal functionality.
For example, one common use this kind of ``corruption aware''
functionality is to describe \emph{endemic} functionalities,
where malicious parties are allowed to choose their own randomness.
This can be written by having the functionality alter its behavior
based on which parties are corrupt, allowing them to choose their
own randomness.
In our framework, we can instead just have a small wrapping
protocol around the functionality, where honest parties
sample a random value before calling the functionality.
Malicious parties are then free to sample a biased value,
deviating from the protocol.

In general, one can always have the functionality
behave differently for certain inputs,
and then restrict honest parties to never trigger this behavior,
thus allowing the functionality to behave differently
for malicious parties.

\subsubsection*{The Lack of Adversaries}

In the traditional presentations of UC,
simulation is a statement of the form ``for all adversaries,
there exists a simulator, such that for all environments...''.
In our framework, we eliminate the notion of adversary
entirely, instead simply considering the environment
to be the adversary.

This is actually a possibility in UC itself.
Subsequent versions of \cite{EPRINT:Canetti00} include
an explicit proof that it suffices to prove security against ``dummy adversary'',
which simply does whatever the environment tells it to do.
We can thus consider our framework to implicitly use such a dummy adversary.

\subsubsection*{The Lack of Session IDs}

Another big difference is that we do away with the use of ``session IDs'',
at least explicitly.
These are most often used to distinguish between multiple
instances of a protocol in a given execution.
These can still be used in our case, but are more implicit.

For example, multiple instances of a protocol would be written
$\mathscr{P} \otimes \mathscr{P}$.
Technically, this is disallowed, but we could fix this
by renaming all of the functions in one instance of the protocol,
so that there's no longer any conflict.
If we use this protocol, in practice it means that we
have a way of distinguishing between the messages belonging
to once instance of the protocol from the other instance.
One way of accomplishing this would be assining session ids,
but these aren't a formal part of our framework.

An exploration of secure composition without session IDs
was also conducted for UC security and other models
in \cite{CCS:KusTue11}.

\subsubsection*{``Timing Side-Channels''}

One unfortunate strength of our framework is that the adversary
is able to observe more timing properties of protocol execution.
Indeed, they are able to observe how many times a given function
yields before returning a result, or simply whether
or not a function can return a result given the current state of
execution.
This is a consequence of the more asynchronous nature
of execution our framework ends up having for protocols.

This is arguably present in some variants of UC already,
depending on the precision of the proofs.
Indeed, if the adversary is able to stall or abort execution,
then this needs to be reflected in the functionality
targeted by the simulation proof.
This is how the notion of ``MPC with abort'' arises.

In some cases though, it seems like the visible
delays are an undesirable consequence of simulation that
is required to be, perhaps, too precise.
We think that further work could develop more relaxed notions of simulation,
which can paper over inessential differences like those of timing and delay.

\subsubsection*{Clearer Connection with Games}

Finally, we believe that a major advantage of our framework is that
it provides a much simpler bridge between standalone security
with games, and the composable security of protocols.
By having ideal functionalities be simply games,
and by having theorems showing that
we can use indistinguishability results for games to produce
simulation results for protocols,
our framework makes it easier to use work done in proving
standalone security for proving protocol security.
Furthermore, simulation arguments ultimately boil down
to an argument about games, and so this can motivate the
intricate games that one might find in the analysis of protocols
such as messaging.
