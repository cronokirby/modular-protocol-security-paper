\documentclass[12pt]{article}
\usepackage[a4paper, margin=1.5in]{geometry}
\usepackage{hyperref}
\usepackage{fancyhdr}
\usepackage[parfill]{parskip}

\usepackage{xcolor}
\definecolor{highlight}{HTML}{2563EB}
\definecolor{gamebg}{HTML}{bfdbfe}

\usepackage[outline]{contour}

\usepackage{newtxtext, newtxmath}
\usepackage[T1]{fontenc}
\usepackage{scalerel}
\let\openbox\relax
\usepackage{amsthm}
\usepackage{etoolbox}

\newtheoremstyle{mystyle}% 〈name〉
{}% 〈Space above〉1
{}% 〈Space below 〉1
{}% 〈Body font〉
{}% 〈Indent amount〉2
{\bfseries}% 〈Theorem head font〉
{.}% 〈Punctuation after theorem head 〉
{0.5em}% 〈Space after theorem head 〉3
{\thmname{#1}\thmnumber{ #2}\thmnote{ (#3)}}
\theoremstyle{mystyle}

\newtheorem{atheorem}{Theorem}[section]
\newtheorem{alemma}[atheorem]{Lemma}
\newtheorem{aclaim}[atheorem]{Claim}
\newtheorem{acorollary}[atheorem]{Corollary}
\newtheorem{adefinition}{Definition}[section]
\newtheorem{asyntax}[adefinition]{Syntax}

\newcounter{theoremcount}[section]
\renewcommand{\thetheoremcount}{\arabic{theoremcount}}

\newcounter{definitioncount}[section]
\renewcommand{\thedefinitioncount}{\arabic{definitioncount}}

\newcommand{\theoremcaption}{}
\newenvironment{theorem}[1][]
{
\refstepcounter{theoremcount}
\renewcommand{\theoremcaption}{#1}
\begin{atheorem}[{#1}]
}
{
\end{atheorem}
}

\newcommand{\lemmacaption}{}
\newenvironment{lemma}[1][]
{
\refstepcounter{theoremcount}
\renewcommand{\lemmacaption}{#1}
\begin{alemma}[{#1}]
}
{
\end{alemma}
}

\newcommand{\claimcaption}{}
\newenvironment{claim}[1][]
{
\refstepcounter{theoremcount}
\renewcommand{\claimcaption}{#1}
\begin{aclaim}[{#1}]
}
{
\end{aclaim}
}

\newcommand{\corollarycaption}{}
\newenvironment{corollary}[1][]
{
\refstepcounter{theoremcount}
\renewcommand{\corollarycaption}{#1}
\begin{acorollary}[{#1}]
}
{
\end{acorollary}
}

\newcommand{\definitioncaption}{}
\newenvironment{definition}[1][]
{
\refstepcounter{definitioncount}
\renewcommand{\definitioncaption}{#1}
\begin{adefinition}[{#1}]
}
{
\end{adefinition}
}

\newcommand{\syntaxcaption}{}
\newenvironment{syntax}[1][]
{
\refstepcounter{definitioncount}
\renewcommand{\syntaxcaption}{#1}
\begin{asyntax}[{#1}]
}
{
\end{asyntax}
}

% For moving the title up
\usepackage{titling}
\setlength{\droptitle}{-4em}
\usepackage{mystyles}
\usepackage{mymacros}
\usepackage{game}

\usepackage{float}
\usepackage{newfloat}
\usepackage{caption}

\usepackage{chngcntr}
\usepackage{stmaryrd}

\DeclareFloatingEnvironment[fileext=game,placement={!h},name=Game,within=section]{gamefloat}
\captionsetup[gamefloat]{labelfont=bf}
\newcommand{\gamelabel}{}
\newcommand{\gamecaption}{}
\newenvironment{game}[2]
{
\renewcommand{\gamelabel}{#1}
\renewcommand{\gamecaption}{#2}
\begin{gamefloat} 
}
{
\caption{\gamecaption}
\label{\gamelabel}
\end{gamefloat}
}

\DeclareFloatingEnvironment[placement={H},name=Protocol,within=section]{protocolfloat}
\captionsetup[protocolfloat]{labelfont=bf}
\newcommand{\protocollabel}{}
\newcommand{\protocolcaption}{}
\newenvironment{protocol}[2]
{
\renewcommand{\protocolcaption}{#2}
\renewcommand{\protocollabel}{#1}
\begin{protocolfloat} 
}
{
\caption{\protocolcaption}
\label{\protocollabel}
\end{protocolfloat}
}

\makeatletter
\let\c@protocolfloat\c@gamefloat
\makeatother


\DeclareFloatingEnvironment[placement={!h},name=Functionality,within=section]{functionalityfloat}
\captionsetup[functionalityfloat]{labelfont=bf}
\newcommand{\functionalitycaption}{}
\newenvironment{functionality}[2]
{
\renewcommand{\functionalitycaption}{#2}
\begin{functionalityfloat} 
\label{#1}
}
{
\caption{\functionalitycaption}
\end{functionalityfloat}
}

\makeatletter
\let\c@functionalityfloat\c@gamefloat
\makeatother
