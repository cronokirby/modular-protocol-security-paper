\documentclass[12pt]{article}
\usepackage[a4paper, margin=1.5in]{geometry}
\usepackage{hyperref}
\usepackage{fancyhdr}
\usepackage[parfill]{parskip}

\usepackage{xcolor}
\definecolor{highlight}{HTML}{2563EB}
\definecolor{gamebg}{HTML}{bfdbfe}

\usepackage[outline]{contour}

\usepackage{newtxtext, newtxmath}
\usepackage[T1]{fontenc}
\usepackage{scalerel}
\let\openbox\relax
\usepackage{amsthm}
\usepackage{etoolbox}

\newtheoremstyle{mystyle}% 〈name〉
{}% 〈Space above〉1
{}% 〈Space below 〉1
{}% 〈Body font〉
{}% 〈Indent amount〉2
{\bfseries}% 〈Theorem head font〉
{.}% 〈Punctuation after theorem head 〉
{0.5em}% 〈Space after theorem head 〉3
{\thmname{#1}\thmnumber{ #2}\thmnote{ (#3)}}
\theoremstyle{mystyle}

\newtheorem{atheorem}{Theorem}[section]
\newtheorem{alemma}[atheorem]{Lemma}
\newtheorem{aclaim}[atheorem]{Claim}
\newtheorem{acorollary}[atheorem]{Corollary}
\newtheorem{adefinition}{Definition}[section]
\newtheorem{asyntax}[adefinition]{Syntax}

\newcounter{theoremcount}[section]
\renewcommand{\thetheoremcount}{\arabic{theoremcount}}

\newcounter{definitioncount}[section]
\renewcommand{\thedefinitioncount}{\arabic{definitioncount}}

\newcommand{\theoremcaption}{}
\newenvironment{theorem}[1][]
{
\refstepcounter{theoremcount}
\renewcommand{\theoremcaption}{#1}
\begin{atheorem}[{#1}]
}
{
\end{atheorem}
}

\newcommand{\lemmacaption}{}
\newenvironment{lemma}[1][]
{
\refstepcounter{theoremcount}
\renewcommand{\lemmacaption}{#1}
\begin{alemma}[{#1}]
}
{
\end{alemma}
}

\newcommand{\claimcaption}{}
\newenvironment{claim}[1][]
{
\refstepcounter{theoremcount}
\renewcommand{\claimcaption}{#1}
\begin{aclaim}[{#1}]
}
{
\end{aclaim}
}

\newcommand{\corollarycaption}{}
\newenvironment{corollary}[1][]
{
\refstepcounter{theoremcount}
\renewcommand{\corollarycaption}{#1}
\begin{acorollary}[{#1}]
}
{
\end{acorollary}
}

\newcommand{\definitioncaption}{}
\newenvironment{definition}[1][]
{
\refstepcounter{definitioncount}
\renewcommand{\definitioncaption}{#1}
\begin{adefinition}[{#1}]
}
{
\end{adefinition}
}

\newcommand{\syntaxcaption}{}
\newenvironment{syntax}[1][]
{
\refstepcounter{definitioncount}
\renewcommand{\syntaxcaption}{#1}
\begin{asyntax}[{#1}]
}
{
\end{asyntax}
}

% For moving the title up
\usepackage{titling}
\setlength{\droptitle}{-4em}
\usepackage{mystyles}
\usepackage{mymacros}
\usepackage{game}

\usepackage{float}
\usepackage{newfloat}
\usepackage{caption}

\usepackage{chngcntr}
\usepackage{stmaryrd}

\DeclareFloatingEnvironment[fileext=game,placement={!h},name=Game,within=section]{gamefloat}
\captionsetup[gamefloat]{labelfont=bf}
\newcommand{\gamelabel}{}
\newcommand{\gamecaption}{}
\newenvironment{game}[2]
{
\renewcommand{\gamelabel}{#1}
\renewcommand{\gamecaption}{#2}
\begin{gamefloat} 
}
{
\caption{\gamecaption}
\label{\gamelabel}
\end{gamefloat}
}

\DeclareFloatingEnvironment[placement={H},name=Protocol,within=section]{protocolfloat}
\captionsetup[protocolfloat]{labelfont=bf}
\newcommand{\protocollabel}{}
\newcommand{\protocolcaption}{}
\newenvironment{protocol}[2]
{
\renewcommand{\protocolcaption}{#2}
\renewcommand{\protocollabel}{#1}
\begin{protocolfloat} 
}
{
\caption{\protocolcaption}
\label{\protocollabel}
\end{protocolfloat}
}

\makeatletter
\let\c@protocolfloat\c@gamefloat
\makeatother


\DeclareFloatingEnvironment[placement={!h},name=Functionality,within=section]{functionalityfloat}
\captionsetup[functionalityfloat]{labelfont=bf}
\newcommand{\functionalitycaption}{}
\newenvironment{functionality}[2]
{
\renewcommand{\functionalitycaption}{#2}
\begin{functionalityfloat} 
\label{#1}
}
{
\caption{\functionalitycaption}
\end{functionalityfloat}
}

\makeatletter
\let\c@functionalityfloat\c@gamefloat
\makeatother


\date{\today}
\title{Towards Modular Foundations for\\ Protocol Security}
\author{Lúcás Críostóir Meier\\\texttt{lucas@cronokirby.com}}

\begin{document}

\maketitle

\begin{abstract}

Universally composable (UC) security \cite{EPRINT:Canetti00}
is the most widely used framework for analyzing the security of cryptographic
protocols.
Many variants and simplifications of the framework have been proposed and developed,
nonetheless, many practitioners find UC proofs to be both difficult
to construct and understand.
We remedy this situation by proposing a new framework for protocol security.
We believe that our framework provides proofs that are both
easier to write, but also more rigorous, and easier to understand.
Our work is based on state-separable proofs
\cite{AC:BDFKK18},
allowing for \emph{modular} proofs, by decomposing
complicated protocols into simple components.
\end{abstract}

\section{Introduction}

Universally composable (UC) security \cite{EPRINT:Canetti00}
has seen widespread success since its introduction over
two decades ago,
becoming the dominant framework for analyzing the security of cryptographic
protocols.

This success is understandable, because the guarantees
the framework provides are very useful.
In essence, when a protocol is proved secure in the UC framework,
then arbitrary instances of that protocol can be used concurrently in the context
of a larger system, without modifying the behavior
of these protocols.
This allows the analysis of isolated components to be used
to guarantee the security of a system as a whole;
this modular analysis is essential to being able to scale
formal analysis to larger systems.

The framework is not ubiquitous however.
For many cryptographic schemes, standalone security with games
is a more suitable approach.
Even for some protocols, the use of game-based security has been popular.
The analysis of threshold signatures and messaging protocols
has seen the development of increasingly intricate games,
which describe all the ways in which an adversary can attack a protocol.
This is the main disadvantage of game-based security for protocols:
it's not always clear what the ``right'' game is to analyze
the security of a protocol, and these games often need to provide
explicit capabilities to the adversary.
This problem is alleviated somewhat in UC security,
where the security of a protocol boils down to describing
an ``ideal functionality'',
which represents how the protocol could function if one had access
to a perfectly trusted third party,
thus making it easier to determine if a given notion of security
reflects the kinds of attacks that need to be analyzed.

The game based approach is still sometimes preferred because of
its perceived ease of use as compared to UC proofs.
Some of these difficulties are inherent:
because UC security provides stronger guarantees than standalone
security, it's not surprising that proofs would involve more ``work''.

Some of these difficulties don't seem to be inherent though,
which is why a series of works have provided improvements,
simplifications, and variants of UC security.
GNUC \cite{JC:HofSho15} was an early variant,
simplifying many aspects of UC, and also patching several
foundational gaps present in the paper at the time.

One disadvantage of developing a new framework is that
proofs in one framework may not necessarily
or automatically translate to UC proofs.
One approach to addressing this is to develop
a ``higher level'' language for simpler proofs,
which is then compiled down to an actual UC proof.
This was done in \cite{C:CanCohLin15},
which provided a simplified version of UC, suitable
for the common setting of multiparty computation,
but also a way to interpret proofs in this simplified
model as actually being UC proofs.

Another series of works ended up taking
a similar approach.
The IITM framework \cite{EPRINT:Kuesters06, JC:KusTueRau20a}
was first introduced as an alternative to UC, with very different foundations.
Recently, this framework was shown to be effectively equivalent to UC
\cite{EC:RauKusChe22}.

Another interesting alternative to UC is that presented in
\cite{cramer2015secure}.
This approach defines a kind of UC security in terms of
a calculus for \emph{interactive systems},
and their composition.
This is an interesting departure from the interactive turing machine
foundations, and does away with many inessential details.
This approach is the most similar to the framework we develop
in this work.

In practice, UC proofs are often quite informal,
without explicitly mentioning the various details that the formalism
might require.
For example, the framework might specify protocols
in terms of interactive turing machines, but in practice,
proofs are written with an informal description of what the protocol
does.
We think that this informality actually
makes proofs harder to write and understand,
because it isn't clear what exactly a proof can consist of,
nor what certain informal patterns mean precisely.

In this work, we propose a new framework for analyzing the security
of protocols, which we believe to be both less informal,
but also simpler to understand and use.

\subsection{Our Framework}

More modular, formal, and simpler.

breaking down protocols.

transitivity of simulation

connection with games.

founded on state separable proofs,
most similar to.

hierarchy of stuff

we also formalize state separable proofs while we're at it, why not.

\subsection{Overview}

as mentioned above, we need to first define packages,
systems, etc.

in section 2

in section 3

in section 4

in section 5

in section 6

\section{State-Separable Proofs}

Our framework for describing protocols is based on 
\emph{state-separable proofs}
\cite{AC:BDFKK18}.
The security notions we develop for protocols ultimately
find meaning in analogous notions of security for \emph{packages},
the main object of study in state-seperable proofs.

This section is intended to be a suitable independent presentation
of this formalism.
In that spirit, we develop state-separable proofs ``from scratch''.
Our starting point is merely that of computable randomized functions.
This is in contrast to other protocol security frameworks like UC,
whose foundational starting point is usually the more concrete notion
of \emph{interactive turing machines}.

We also take the opportunity to solidify the formalism of state-separable
proofs, providing more complete definitions of various objects,
completing several proofs left as mere sketches in the original paper,
and proving a few additional properties we'll need later.
This makes this section of interest to readers who are already familiar
with state-separable proofs.

\subsection{Some Notational Conventions}

We write $[n]$ to denote the set $\{1, \ldots, n\}$.

We write $\txt{01}$ to denote the set $\{0, 1\}$,
and write $\str$ to denote binary strings.
We write $\bullet$ to denote the empty string,
which also serves as a ``dummy'' value in various contexts.

\subsection{Probabilistic Functions}

Our starting point is the notion of \emph{randomized computable functions}.
This is a notion we assume can be defined in a rigorous way, but whose
concrete semantics we don't assign.
We write $f : \randf{\str}{\str}$ to denote such a function (named $f$).
Intuitively, this represents a function described by some algorithm,
which takes in a binary string as an input, and produces a binary string
as output, and is allowed to make randomized decisions to aid its computation.

We mainly consider \emph{families} of functions,
parametrized by a security parameter $\lambda$.
Formally, this is in fact a function $f : \mathbb{N} \to \randf{\str}{\str}$,
and we write $f_{\lambda} : \randf{\str}{\str}$ to denote a particular
function in the family.
In most cases, this security parameter is left \emph{implicit}.
In fact, all of the objects we consider from here on out will \emph{implicitly}
be \emph{families} of objects, parametrized by a security parameter $\lambda$,
and we will invoke this fact only as necessary.

\begin{definition}[Efficient Functions]
    We assume that a function $f$ has a runtime, denoted $T(f, x)$,
    measuring how long the function takes to execute on a given input $x \in \str$.

    We say that a function family is \emph{efficient} if:
    $$
    \forall \lambda.\ \forall x, |x| \in \mathcal{O}(\tx{poly}(\lambda)).\quad T(f_\lambda, x) \in \mathcal{O}(\tx{poly}(\lambda))
    $$
    In other words, the runtime is always polynomial in $\lambda$, regardless
    of the input, or the random choices of the function.

    $\square$
\end{definition}

Functions which are not necessarily efficient are said to be \emph{unbounded}.

Considering efficient functions is essential, because the vast majority
of cryptographic techniques depend on assuming that some problems are ``hard''
for adversaries with bounded computational resources,
and so this notion of efficiency is critical to defining game-based
security.
Ironically, for protocol security, many protocols can be proved secure
without this restriction.

Another crucial notion we need to develop is that of a \emph{distance},
measuring how different two functions behave.
This will underpin our later notion of security for games,
which is based on saying that two different games are difficult
to tell apart.

\begin{definition}[Distance Function]
    Given a function $f : \randf{\bullet}{\bin}$, we assume that the probability
    $P[f \to 1]$ of the function returning $1$ on the input $\bullet$ is well defined.

    Given two functions $f, g$, we define their distance $\varepsilon(f, g)$ as:
    $$
    \varepsilon(f, g) := |P[f \to 1] - P[g \to 1]|
    $$

    $\square$
\end{definition}

In other words, the distance looks at how often one function returns $1$
compared to the other.
If the functions agree most of the time, then their distance will be small,
whereas if they disagree very often, their distance will be large.
This definition is actually quite natural.
Since $P[f \to 1] = (1 - P[f \to 0])$, $\varepsilon$ is actually
just the total variation---or statistical---distance.
This immediately implies that this distance has some nice properties,
in particular that it forms a \emph{metric}.

\begin{lemma}[Distance is a Metric]
    $\varepsilon$ is a valid metric, in particular, it holds
    for any functions $f, g, h$, that:
    \begin{enumerate}
        \item $\varepsilon(f, f) = 0$,
        \item $\varepsilon(f, g) = \varepsilon(g, f)$,
        \item $\varepsilon(f, h) \leq \varepsilon(f, g) + \varepsilon(g, h)$.
    \end{enumerate}

    \txbf{Proof:}

    \txbf{1.} Follows from the fact that $P[f \to 1] = P[f \to 1]$,
    so $\varepsilon(f, f) = 0$.

    \txbf{2.} Follows from the fact that $|a - b| = |b - a|$.

    \txbf{3.} Follows from the triangle inequality for $\mathbb{R}$
    and the fact that:
    $$
    |P[f \to 1] - P[h \to 1]| = |(P[f \to 1] - P[g \to 1]) + (P[g \to 1] - P[H \to 1])|
    $$

    $\blacksquare$
\end{lemma}

We actually skipped one property in our proof that $\varepsilon$ is a valid metric,
which requires that if $f \neq g$, then $\varepsilon(f, g) > 0$.
This is because we haven't yet defined what equality should mean
for functions.
This metric property gives us a very natural definition though.

\begin{definition}[Function Equality]
    Two functions, $f$ and $g$, are \emph{equal}, written $f = g$, when:
    $$
    \varepsilon(f, g) = 0
    $$

    $\square$
\end{definition}

It's easy to see that this is an equality relation, satisfying
reflexivity, symmetry, and transitivity.

We can also generalize this to arbitrary functions, rather than just
$f : \randf{\bullet}{\bin}$, by defining:
$$
\varepsilon(f, g) := \tx{sup}_{x, y \in \str} |P[f(x) \to y] - P[g(x) \to y]|
$$
however, we will not really be needing this general definition, outside
of a technical and very strong notion of equality for packages
used in the following subsection.

While the functions we've considered so far only manipulate binary strings,
it's useful to allow \emph{typed} functions,
with richer input and output types.
This could be defined in several ways, but the end result means
that a typed function $f : \randf{A}{B}$ can be interpreted as a function
over binary strings, using a suitable encoding and decoding mechanism,
as well as perhaps having a special output value that $f$ can return
if it fails to decode its input successfully.

Being able to quantify types is also useful for the formalism itself,
and potentially even for some packages.
This allows us to type functions such as:
$$
\begin{aligned}
    &\tx{id} : \forall s.\ s \to s\cr
    &\tx{id} = x \mapsto x
\end{aligned}
$$
In this example, $s$ is a quantified type variable.
Formally, we can see $\tx{id}$ as a function parametrized by a type,
with $\tx{id}_S$ being a concrete function, after having chosen this type.

\subsection{Packages}

Our next goal is to define the central object of state-separable proofs:
the \emph{package}.
Intuitively, a package has some kind of state, as well as functions
which manipulate this state.
You can interact with a package by calling the various output functions
it provides.
This makes packages a natural fit for security games.
What distinguishes packages from games is that they can have \emph{input}
functions.
A package can depend on another package, with each of its functions
potentially using the functions provided by this other package.
This modularity makes the common proof technique of ``game-hopping''
much more easily usable, and is the core strength of the state-separable
proof formalism.

Before we get to packages, we first need to define a few convenient
notions for functions manipulating a state, and parametrizing
functions with other functions.

Our first definition will be a little bit of shorthand.
\begin{definition}[Stateful Function]
    A \emph{stateful} function is simply a function $f$ of the form:
    $$
    f : \randf{(S, \str)}{(S, \str)}
    $$
    $S$ represents the state being used and modified by the function.
    As a convenient shorthand, we write:
    $$
    f :\ \stateful{S}
    $$

    $\square$
\end{definition}
It's useful to have a bit of typing to separate the state from the rest
of the input and output, since it allows us to avoid defining
inessential padding details inside the formalism itself.

We also need to consider functions parametrized by other functions.
Intuitively, this arises when one function calls another.
For example, consider:
$$
f(x) := g(x) \oplus g(x)
$$
which is well defined regardless of what $g$ is.
Here $f$ is implicitly parametrized by $g$, but we could write this explicitly
as $f(x) := g \mapsto g(x) \oplus g(x)$.
We could write $f : (\randf{\str}{\str}) \to (\randf{\str}{\str})$  
as a potential type in this example.
We write $f[g]$ for the instantiation of a parametrized function $f$
with an input function $g$.
It might also be the case that $g$ is itself parametrized,
in which case $f[g]$ is defined as:
$$
f[g] := h \mapsto f[g[h]]
$$

We can define a natural, albeit very strong, notion of equality for parametrized
functions, saying that:
$$
f = g \iff \forall h_1, \ldots, h_n.\ f[h_1, \ldots] = g[h_1, \ldots]
$$
In other words, the two functions must be equal regardless of how we instantiate
them.

We've now developed enough tools to define packages.

\begin{definition}[Package]
    A package $A$ consists of:
    \begin{itemize}
        \item a type $S$, for its state,
        \item a set of \emph{input names} $\tx{In}(A)$, of size $m$,
        \item a permutation $\pi_{\tx{in}} : \tx{In}(A) \leftrightarrow [m]$,
        \item a set of \emph{output names} $\tx{In}(A)$, of size $n$,
        \item a permutation $\pi_{\tx{out}} : [m] \leftrightarrow \tx{Out}(A)$,
        \item a set of parametrized functions $f_1, \ldots, f_n : \forall s. \stateful{s}^m \to \stateful{(S, s)}$,
        each of which has a distinct name $n_i \in \tx{Out}(A)$.
    \end{itemize}

    We also only consider a package to be defined \emph{up to} potentially
    renaming its input and output functions injectively.

    $\blacksquare$
\end{definition}

The basic idea is that a package has an internal state $S$, which gets
manipulated by each of the functions it exports.
These functions, in turn, can depend on other input functions.
If a stateful function $f$ uses a stateful function $g$,
then the result is a stateful function $f[g]$ manipulating 
\emph{both} the state of $f$, and the state of $g$.
Furthermore, $f$ is defined in such way agnostic to what the state manipulated
by $g$ happens to be,
which is why we use a \emph{quantified} type instead, to allow instantiation
with functions manipulating different kinds of state.

In practice, each function in a package is unlikely to use \emph{all}
of the input functions of the package, but it is much simpler
to have each function parametrized by all the possible inputs,
even if some are left unused.
It's also much simpler to define an ordering of the input functions
$\pi$, so be able to use $\stateful{s}^m$ as the input type for
the parametrized functions.

The semantics of a package without inputs are intuitively that of a stateful computer program
or machine you can interact with.
The machine has some kind of state, represented by $S$,
along with various functions you can call, represented by $f_1, \ldots, f_n$.
Each of these will use the input you provide, along with the current state
of the machine, in order to supply you with an output,
potentially modifying the state along the way.
The input functions allow a package to interact with other packages itself.

We'll often use $\tx{In}(A)$ or $\tx{Out}(A)$ to talk about the input
and output functions of a package.
As a bit of a short hand notation, we write $\tx{In}(A, B, \ldots)$
for the union $\tx{In}(A) \cup \tx{In}(B) \cup \ldots$, and similarly
for $\tx{Out}(\ldots)$.

We describe this kind of interaction using the formal
notion of package \emph{composition}.

\begin{definition}[Package Composition]
    Given two packages $A, B$ with $\tx{In}(A) \subseteq \tx{Out}(B)$,
    we define their composition $A \circ B$ as a package characterized by:

    \begin{itemize}
        \item a state type $(A.S, B.S)$,
        \item input names $\tx{In}(B)$,
        \item output names $\tx{Out}(A)$,
        \item $\pi_{\tx{in}} := B.\pi_{\tx{in}}$,
        \item $\pi_{\tx{out}} := A.\pi_{\tx{out}}$,
        \item output functions $A.f_1[\varphi(B.f_1), \ldots, \varphi(B.f_{B.n})], \ldots$
        of type $\forall s.\stateful{s}^{B.m} \to \stateful{(A.S, B.S)}$.
    \end{itemize}

    In more detail, each output function is of the form:
    $$
    (h_1, \ldots, h_{B.m}) \mapsto A.f_i[\varphi_{A}(B.f_1)[h_1, \ldots], \ldots, \varphi_{A}(B.f_{B.n})[h_1, \ldots]]
    $$
    where $\varphi_{A}$ assigns each function $B.f_i$ to a slot in $[m]$
    using $A.\pi_{\tx{in}}$ on the name of that function, $B.n_i$.
    The same input functions $h_j$ being given
    to all the functions used by $A.f_i$.

    $\square$
\end{definition}

Package composition formally defines the intuitive notion of one package
``using'' the functions provided by another package.
The result is a package providing the functions defined in $A$,
and requiring the functions needed by $B$, but with the functions
inside $B$ itself now effectively inlined inside of $A \circ B$.

Next we'd like to prove that package composition satisfies some nice
properties.
For example $A \circ (B \circ C)$ is the same as $(A \circ B) \circ C$.
There's one problem though, which is that we haven't defined
what it means for two packages to be "the same".

\begin{definition}[Literal Equality]
    We say that two packages $A, B$ are \emph{literally equal},
    written $A \equiv B$, when:
    \begin{itemize}
        \item $A.S \cong B.S$,
        \item $\tx{In}(A) = \tx{In}(B)$,
        \item $\tx{Out}(A) = \tx{Out}(B)$,
        \item There exists a permutation $\pi : [n] \leftrightarrow [n]$ such that
        $$\forall i \in [n].\ A.f_i = B.f_{\pi(i)} \land A.n_i = B.n_{\pi(i)}$$
    \end{itemize}

    $\square$
\end{definition}

We require strict equality for the input and output names,
to avoid spurious comparisons between two packages with completely different names,
although it should be noted that packages are only really defined up to renaming anyways,
so this is essentially an isomorphism constraint.
For the type of state, we consider an isomorphism directly,
mainly so that $(A.S, (B.S, C.S))$ is considered to be the same state
type as $((A.S, B.S), C.S)$, which might already be the case depending on
how one defines equality for sets.
The final condition also implies that $\pi_{\tx{in}}$ is the same
for both packages.

This notion of equality is very strong, especially because of the equality
it imposes on the functions defined in each package.
While it suffices to explore basic properties of composition for packages,
we'll want to abandon it quite quickly for a looser and more easily
used notion of equality.

The first property we prove is the one used as an example above.

\begin{lemma}[Associativity of Composition]
    Given packages $A, B, C$, it holds that:
    $$
    A \circ (B \circ C) \equiv (A \circ B) \circ C
    $$
    provided these expressions are well defined.

    \txbf{Proof:} The input and output names are clearly equal on both sides.
    Furthermore, the state on the left is $(A.S, (B.S, C.S))$,
    $((A.S, B.S), C.S)$ on the right, and so the two states are isomorphic.
    All that's left is the final condition, talking about the equality
    of the functions defined in each package.

    Now, for the equality of functions, we'll expand the functions
    of the package on the left, and then on the right, before comparing
    the results we get.

    The functions in $B \circ C$ are of the form:
    $$
    (h_1, \ldots) \mapsto B.f_i[\varphi_{B}(C.f_1)[h_1, \ldots], \ldots]
    $$
    And then the functions in $A \circ (B \circ C)$ are of the form:
    $$
    (h_1, \ldots) \mapsto A.f_i[\varphi_{A}(B.f_1)[\varphi_{B}(C.f_1)[h_1, \ldots]], \ldots]
    $$

    From the other side, the functions in $A \circ B$ are of the form:
    $$
    (h_1, \ldots) \mapsto A.f_i[\varphi_{A}(B.f_1)[h_1, \ldots], \ldots]
    $$
    This makes the functions in $(A \circ B) \circ C$ of the form:
    $$
    (h_1, \ldots) \mapsto A.f_i[\varphi_{A}(B.f_1)[\varphi_{A \circ B}(C.f_1)[h_1, \ldots]], \ldots]
    $$

    Now, the only difference is that we end up with $\varphi_{A \circ B}$
    as our means of assigning the functions in $C$ to the slots
    of $B$.
    However, $\varphi_{X}$ only depends on $X.\pi_{\tx{in}}$,
    and by definition $(A \circ B).\pi_{\tx{in}} = B.\pi_{\tx{in}}$,
    so $\varphi_{A \circ B} = \varphi_B$.

    So, in both cases, we end up with the same functions, concluding
    our proof.

    $\blacksquare$
\end{lemma}

This property is useful, since it lets us simply write $A \circ B \circ C$,
without worrying about the order in which packages are composed.

Another more technical property we want composition to satisfy is
that if \emph{equality preservation}.
If $B \equiv B'$, then it should be the case that $A \circ B \equiv A \circ B'$,
or that $B \circ C \equiv B' \circ C$.
If that weren't the case, then that would indicate that something is wrong
with our definition of either equality or composition.
The property we want for literal equality is that $A$ and $A'$ are completely
interchangeable, and so once can always be replaced with the other, no matter
the context, to the point that we can think of them as literally being the same
package.

Thankfully, it turns out that composition and literal equality do
in fact get along.

\begin{lemma}[Composition Preserves Equality]
    Given any packages $A, B, B', C$ it holds that:
    \begin{itemize}
        \item $B \equiv B' \implies A \circ B \equiv A \circ B'$,
        \item $B \equiv B' \implies B \circ C \equiv B' \circ C$,
    \end{itemize}
    provided these expressions are well defined.

    \txbf{Proof:} In one case the state type is $(A.S, B.S)$
    or $(A.S, B'.S)$, which are isomorphic if $B.S \cong B'.S$.
    Similarly, in the other case, we have $(B.S, C.S)$ vs $(B'.S, C.S)$,
    and the same observation holds.

    Now, remember that $\tx{In}(X \circ Y) = \tx{In}(Y)$, and $\tx{Out}(X \circ Y) = \tx{Out}(X)$.
    Thus, since both $\tx{In}(B) = \tx{In}(B')$ and $\tx{Out}(B) = \tx{Out}(B')$
    hold, we conclude that $\tx{In}$ and $\tx{Out}$ match up in both cases.

    The trickier part is the 4th condition for equality.

    In the first case, the functions are of the form:
    $$
    A.\tx{f}_i[\varphi_A(B.f_1), \ldots]
    $$
    Now, $\varphi_A$ orders the functions in $B$ based only on their \emph{names}.
    In particular, the ordering does not matter.
    Since the functions in $B'$ are the same as $B$ up to their ordering,
    including their names,
    $\varphi_A$ will order them in the same way.
    Thus, the functions in $A \circ B$ and $A \circ B'$.

    In the second case, the functions are of the form:
    $$
    B.\tx{f}_i[\varphi_B(C.f_1), \ldots]
    $$
    Now, $\pi_{\tx{in}}$ is the same for both $B$ and $B'$, as we've remarked
    before.
    Thus, $\varphi_B$ and $\varphi_{B'}$ are the same.
    Thus, the functions in $B \circ C$ are the same as $B \circ C'$,
    up to reordering, as required.

    Having noted all of these points, we can conclude our proof.

    $\blacksquare$
\end{lemma}

Now, we look at the other kind of composition for packages:
tensoring.
The intuitive idea is that tensoring allows us to run two
packages ``in parallel''.
The result of tensoring two packages is a new package with the functions
in both packages, and we can interact with one package or the other at will.
We'll discuss the semantics a bit more after the formal definition.

\begin{definition}[Package Tensoring]
    Given two packages $A$, $B$, with $\tx{Out}(A) \cap \tx{Out}(B) = \emptyset$,
    we can define their tensoring $A \otimes B$ as a package characterized by:
    \begin{itemize}
        \item a state type $(A.S, B.S)$,
        \item input names $\tx{In}(A) \cup \tx{In}(B)$,
        \item output names $\tx{Out}(A) \cup \tx{Out}(B$),
        \item an output name assignment defined by:
        $$
        \pi_{\tx{out}}(i) := \begin{cases}
            A.\pi_{\tx{out}}(i) & i \leq A.n\cr
            B.\pi_{\tx{out}}(i - A.n) & i > A.n
        \end{cases}
        $$
        \item an input index assignment defined by:
        $$
        \pi_{\tx{in}}(n) := \begin{cases}
            A.\pi_{\tx{in}}(n) &n \in \tx{In}(A)\cr
            \tx{rank}_B(n) + A.m &n \in \tx{In}(B) / \tx{In}(A)\cr
        \end{cases}
        $$
        where $\tx{rank}_B(n)$ returns the index of $n$ in the list
        of names in $\tx{In}(B) / \tx{In}(A)$, sorted by $B.\pi_{\tx{in}}(n)$.
    \end{itemize}

    Then, for the functions, we have two cases.

    For $i \in [1, \ldots, A.n]$, we have:
    $$
    f_i := (h_1, \ldots, h_m) \mapsto A.f_i[h_1, \ldots, h_{A.m}]
    $$
    Then, for $i \in [A.n + 1, \ldots, A.n + B.n]$, we have:
    $$
    f_i := (h_1, \ldots, h_m) \mapsto
    B.f_i[h_{B.\pi_{\tx{in}}(\pi_{\tx{in}}^{-1}(j))}\mid j \in [m], \pi_{\tx{in}}^{-1}(j) \in \tx{In}(B)]
    $$

    $\square$
\end{definition}

\subsection{Syntactical Conventions for Packages}
\section{Games That Talk}

\subsection{Async Functions}

While the intuition of yield statements is simple, defining
them precisely is a bit more tricky.

\begin{definition}[Yield Statements]
We define the semantics of $\textbf{yield}$ by compiling functions with
such statements to functions without them.

Note that we don't define the semantics for functions which still contain references
to oracles.
Like before, we can delay the definition of semantics until all of the
pseudo-code has been inlined.

A first small change is to make it so that the function accepts one
argument, a binary string, and all yield points also accept binary
strings as continuation.
Like with plain packages, we can implement richer types on top by adding
additional checks to the well-formedness of binary strings, aborting
otherwise.

The next step is to make it so that all the local variables of the function $F$
are present in the global state.
So, if a local variable $v$ is present, then every use
of $v$ is replaced with a use of the global variable $F.v$ in the package.
This allows the state of the function to be saved across yields.

The next step is transforming all the control flow of a function to
use $\textbf{ifgoto}$, rather than structured programming constructs like
$\textbf{while}$ or $\textbf{if}$.
The function is broken into lines, each of which contains a single statement.
Each line is given a number, starting at $0$.
The execution of a function $F$ involves a special variable $\texttt{pc}$,
representing the current line being executing.
Excluding $\textbf{yield}$ and $\textbf{return}$ a single line statement has one of the
forms:
$$
\begin{aligned}
&\langle \texttt{var} \rangle \gets \langle \texttt{expr} \rangle\cr
&\langle \texttt{var} \rangle \xleftarrow{\$} \langle \texttt{dist} \rangle\cr
\end{aligned}
$$
which have well defined semantics already.
Additionally, after these statements, we set $\texttt{pc} \gets \texttt{pc} + 1$.

The semantics of $\textbf{ifgoto } \langle \texttt{expr} \rangle i$ is:
$$
\texttt{pc} \gets \textbf{ if } \langle \texttt{expr} \rangle \textbf{ then } i \textbf{ else } \texttt{pc} + 1
$$
This gives us a conditional jump, and by using $\texttt{true}$ as the condition,
we get a standard unconditional jump.

This allows us to define $\textbf{if}$ and $\textbf{while}$ statements
in the natural way.

Finally, we need to augment functions to handle $\textbf{yield}$ and $\textbf{return}$
statements.
To handle this, each function $F$ also has an associated variable
$F.\texttt{pc}$, which stores the program counter for the function.
This is different than the local $\texttt{pc}$ which is while the function is
execution.
$F.\texttt{pc}$ is simply used to remember the program counter after a yield
statement.

The function now starts with:
$$
\begin{aligned}
&\textbf{ifgoto } \texttt{true}\ F.\text{pc} 
\end{aligned}
$$
This has the effect of resuming execution at the saved program counter.

Furthermore, the input variable $x$ to $F$ is replaced with a special
variable $\texttt{input}$, which holds the input supplied to the function.
At the start of the function body, we add:
$$
0: F.x \gets \texttt{input}
$$
to capture the fact that the original input variable needs to get assigned
to the input to the function.

The semantics of $F.m \gets \textbf{yield } v$ are:
$$
\begin{aligned}
(i-1):&\ F.\text{pc} \gets i + 1\cr
i:&\ \textbf{return } (\texttt{yield}, v)\cr
(i + 1):&\ F.m \gets \texttt{input}
\end{aligned}
$$

The semantics of $\textbf{return } v$ become:
$$
\begin{aligned}
&F.\text{pc} \gets 0\cr
&\textbf{return } (\texttt{return}, v)
\end{aligned}
$$
The main difference is that we annotate the return value to be different than
yield statements, but otherwise the semantics are the same.

$\square$
\end{definition}

Note that while calling a function which can yield will notify the
caller as to whether or not the return value was \emph{yielded}
or \emph{returned}, syntactically the caller often ignores this,
simply doing $x \gets F(\ldots)$, meaning that they simply use
return value $x$, discarding the tag.

\begin{syntax}
In many cases, no value is yielded, or returned back, which we can write as:
$$
\textbf{yield}
$$
which is shorthand for:
$$
\bullet \gets \textbf{yield } \bullet
$$
i.e. just yielding a dummy value and ignoring the result.

$\square$
\end{syntax}

In such situations, often we don't particularly care about the intermediate
yields of a function, and want to wait for the final result,
potentially yielding to our own caller.
We define these semantics via the $\textbf{await}$ statement.

\begin{syntax}[Await Statements]
We define the semantics of $v \gets \await F(\ldots)$ in a straightforward way:
$$
\begin{aligned}
&(\tx{tag}, v) \gets (\texttt{yield}, \bot)\cr
&\txbf{while } \tx{tag} = \texttt{yield}:\cr
\pind{1}\txbf{if } v \neq \bot:\cr
\pind{2}\txbf{yield}\cr
\pind{1}(\tx{tag}, v) \gets F(\ldots)\cr
\end{aligned}
$$
In other words, we keep calling the function until it actually returns
its final value, but we do yield to our caller whenever our function yield, but we do yield to our caller whenever our function yields.

$\square$
\end{syntax}

Sometimes we want to await several values at once, returning the first
one which completes. To that end, we define the $\textbf{select}$ statement.

\begin{syntax}[Select Statements]
Select statements generalize await statements in that they allow waiting
for multiple events concurrently.

More formally, we define:
$$
\begin{aligned}
&\txbf{select}:\cr
\pind{1}v_1 \gets \await F_1(\ldots):\cr
\pind{2} \langle \tx{body}_1 \rangle\cr
\pind{1}\vdots\cr
\pind{1}v_n \gets \await F_n(\ldots):\cr
\pind{2} \langle \tx{body}_n \rangle\cr
\end{aligned}
$$
As follows:
$$
\begin{aligned}
&(\tx{tag}_i, v_i) \gets (\texttt{yield}, \bot)\cr
&i \gets 0\cr
&\txbf{while } \nexists i.\ \tx{tag}_i \neq \texttt{yield}:\cr
\pind{1} \txbf{if } i \geq n:\cr
\pind{2} i \gets 0\cr
\pind{2} \txbf{yield}\cr
\pind{1} i \gets i + 1\cr
\pind{1} (\tx{tag}_i, v_i) \gets F_i(\ldots)\cr
&\langle \tx{body}_i \rangle
\end{aligned}
$$
Note that the order in which we call the functions is completely deterministic,
and fair.
It's also important that we yield, like with await statements, but we only
do so after having pinged each of our underlying functions at least once.
This is so that if one of the function is immediately ready, we never yield.

$\square$
\end{syntax}

\subsection{Channels and System Composition}


\begin{definition}[Systems]
A \emph{system} is a package which uses channels.

We denote by $\text{InChan(S)}$ the set of channels the system receives on,
and $\text{OutChan(S)}$ the set of channels the system sends on,
and define
$$
\text{Chan}(S) := \text{OutChan}(S) \cup \text{InChan}(S)
$$
$\square$
\end{definition}

\begin{definition}
We can compile systems to not use channels.
We denote by $\text{NoChan}(S)$ the package corresponding to
a system $S$, with the use of channels replaced with function calls.

Channels define two new syntactic constructions, for sending and receiving
along a channel.
We replace these with function calls as follows:


Sending, with $m \Rightarrow P$ becomes:
$$
\text{Channels}.\tx{Send}_P(m)
$$

Receiving, with $m \Leftarrow P$ becomes:
$$
m \gets \await \text{Channels}.\tx{Recv}_P()
$$
Receiving is an asynchronous function, because the channel might not have
any available messages for us.

These function calls are parameterized by the channel, meaning
that that we have a separate function for each channel.

$\square$
\end{definition}

\begin{game}{game:Channels}{Channels}
\package{Channels($\{A_1, \ldots, A_n\}$)}{
&q[A_i] \gets \text{FifoQueue.New()}\cr
\cr
&\underline{\tx{Send}_{A_i}(m)\tx{:}}\cr
\pind{1} q[A_i].\tx{Push}(m)\cr
\cr
&\underline{\tx{Recv}_{A_i}()\tx{:}}\cr
\pind{1} \txbf{while } q[A_i].\tx{IsEmpty()}\cr
\pind{2} \txbf{yield}\cr
\pind{1} q[A_i].\tx{Next}()\cr
}
\end{game}

One consequence of this definition with separate functions for each channel
is that $\text{Channels}(S) \otimes \text{Channels}(R) = \text{Channels}(S \cup R)$.

Armed with the syntax sugar for channels, and the $\tx{Channels}$ game,
we can convert a system $S$ into a package via:
$$
\text{SysPack}(S) := \text{NoChan}(S) \circ (\text{Channels}(\tx{Chan}(S)) \otimes 1(\tx{In}(S)))
$$
This package will have the same input and output functions as the system $S$,
but with the usage of channels replaced with actual semantics.

This allows us to lift our standard equality relations on packages onto
\emph{systems}.
\begin{definition}
Given some equality relation $\sim$ on packages, we can lift that relation
to systems by definining:
$$
A \sim B \iff \text{SysPack}(A) \sim \text{SysPack}(B)
$$
$\square$
\end{definition}

\begin{definition}[System Tensoring]
Given two systems, $A$ and $B$, with \(\text{Out}(A) \cap \text{Out}(B) = \emptyset\), we can define their tensor product $A * B$,
which is any system satisfying:
$$
\tx{SysPack}(A * B) =
\begin{pmatrix}
&\tx{NoChan}(A)\cr
&\otimes\cr
&\tx{NoChan}(B)\cr
\end{pmatrix}
\circ
\begin{pmatrix}
&\text{Channels}(\tx{Chan}(A) \cup \tx{Chan}(B))\cr
&\otimes\cr
&1(\tx{In}(A) \cup \tx{In}(B))\cr
\end{pmatrix}
$$

$\square$
\end{definition}

Note that combining the definition above with the definition of $\text{SysPack}$
means that:
$$
\begin{aligned}
&\text{NoChan}(A * B) = \text{NoChan}(A) \otimes \text{NoChan}(B)\cr
&\text{Chan}(A * B) = \text{Chan}(A) \cup \text{Chan}(B)\cr
&\text{In}(A * B) = \text{In}(A) \cup \text{In}(B)\cr
\end{aligned}
$$

This implies the following lemma.

\begin{lemma}
System tensoring is associative, i.e. $A * (B * C) = (A * B) * C$.
\txbf{Proof:} Starting from the definition of tensoring, we have:
$$
\tx{SysPack}(A * (B * C)) =
\begin{pmatrix}
&\tx{NoChan}(A)\cr
&\otimes\cr
&\tx{NoChan}(B * C)\cr
\end{pmatrix}
\circ
\begin{pmatrix}
&\text{Channels}(\tx{Chan}(A) \cup \tx{Chan}(B * C))\cr
&\otimes\cr
&1(\tx{In}(A) \cup \tx{In}(B * C))\cr
\end{pmatrix}
$$
We can then apply the corrollaries we've just derived to show that this
is equal to:
$$
\begin{pmatrix}
&\tx{NoChan}(A)\cr
&\otimes\cr
&\tx{NoChan}(B)\cr
&\otimes\cr
&\tx{NoChan}(C)\cr
\end{pmatrix}
\circ
\begin{pmatrix}
  &\text{Channels}(\tx{Chan}(A) \cup \tx{Chan}(B) \cup \tx{Chan}(C))\cr
&\otimes\cr
  &1(\tx{In}(A) \cup \tx{In}(B) \cup \tx{In}(C))\cr
\end{pmatrix}
$$
(Using the associativity of $\otimes$ for \emph{packages} as well).

With the same reasoning, we can derive the same package from $(A * B) * C$,
letting us conclude that $\text{SysPack}(A * (B * C)) = \text{SysPack}((A * B) * C)$,
and thus that $A * (B * C) = (A * B) * C$.

$\blacksquare$
\end{lemma}

\begin{lemma}
System tensoring is commutative, i.e. $A * B = B * A$
\txbf{Proof:} This follows from the commutativity of $\otimes$ and $\cup$.
$\blacksquare$
\end{lemma}

\begin{definition}[Overlapping Systems]
Two systems $A$ and $B$ overlap if $\text{Chan}(A) \cap \text{Chan}(B) \neq \emptyset$.

In the case of non-overlapping systems, we write $A \otimes B$ instead of $A * B$,
insisting on the fact that they don't communicate.
\end{definition}

\begin{definition}[System Composition]
Given two systems, $A$ and $B$, we can define their (horizontal) composition
$A \circ B$ as any system, provided a few constraints hold:
\begin{itemize}
\item $A$ and $B$ do not overlap ($\text{Chan}(A) \cap \tx{Chan}(B) = \emptyset$)
\item $\tx{In}(A) \subseteq \tx{Out}(B)$
\end{itemize}

With these in place, we define the composition as any system such that:
$$
\text{SysPack}(A \circ B) =
\text{SysPack}(A) \circ \text{SysPack}(B)
$$

$\square$
\end{definition}

\begin{lemma}
System composition is associative, i.e. $A \circ (B \circ C) = (A \circ B) \circ C$.
\txbf{Proof:} This follows from the associativity of $\circ$ for \emph{packages}. $\blacksquare$
\end{lemma}

\begin{lemma}[Interchange Lemma]
Given systems $A, B, C, D$ such that $A \circ B$ and $C \circ D$ are well defined,
$A * C$ and $B * D$ are well defined,
and neither $A$ nor $C$ overlap with $B$ or $D$, i.e.
the following relation holds:
$$
\begin{pmatrix} 
A\cr
*\cr
C\cr
\end{pmatrix} 
\circ
\begin{pmatrix} 
B\cr
*\cr
D\cr
\end{pmatrix} 
=
\begin{matrix} 
  (A \circ B)\cr
*\cr
  (C \circ D)\cr
\end{matrix} 
$$

\txbf{Proof:}
First, we need to develop a few general facts about $\tx{SysPack}(A \circ B)$, $\tx{Chan}(A \circ B)$
and $\tx{NoChan}(A \circ B)$, like those we developed for $A * B$.

As a consequence of how $A \circ B$ is defined, by unrolling $\text{SysPack}(A \circ B)$,
we get:
$$
  \tx{SysPack}(A \circ B) =
  \tx{NC}(A) \circ
  \begin{pmatrix}
    \tx{Channels}(\tx{Chan}(A))\cr
    \otimes\cr
    1(\tx{In}(A))
  \end{pmatrix}
  \circ \tx{NC}(B) \circ
  \begin{pmatrix}
    \tx{Channels}(\tx{Chan}(B))\cr
    \otimes\cr
    1(\tx{In}(B))
  \end{pmatrix}
$$
Applying the interchange lemma for packages a couple times, we then get:
$$
  \tx{NC}(A) \circ
  \begin{pmatrix}
    \tx{NC}(B)\cr
    \otimes\cr
    1(\tx{Channels}(\tx{Chan}(A)))
  \end{pmatrix}
  \circ
  \begin{pmatrix}
    \tx{Channels}(\tx{Chan}(A)) \otimes \tx{Channels}(\tx{Chan}(B))\cr
    \otimes\cr
    1(\tx{In}(B))
  \end{pmatrix}
$$

And then, recalling
that $\tx{Channels}(S) \otimes \tx{Channels}(R) = \tx{Channels}(S \cup R)$,
we conclude that:
$$
\begin{aligned}
  &\tx{NoChan}(A \circ B) =
  \tx{NC}(A) \circ
  \begin{pmatrix}
    \tx{NC}(B)\cr
    \otimes\cr
    1(\tx{Channels}(\tx{Chan}(A)))
  \end{pmatrix}
  \cr
  &\tx{Chan}(A \circ B) = \tx{Chan}(A) \cup \tx{Chan}(B)\cr
\end{aligned}
$$

Next we apply these facts, along with those derived for $A * B$ to tackle the main lemma.

Starting from $\tx{SysPack}((A * C) \circ (B * D))$, we can apply the above
results to get:
$$
  \tx{NC}(A * C) \circ
  \begin{pmatrix}
    \tx{NC}(B * D)\cr
    \otimes\cr
    1(\tx{Channels}(\tx{Chan}(A * C)))
  \end{pmatrix}
  \circ
  \begin{pmatrix}
    \tx{Channels}(\tx{Chan}(A * C) \cup \tx{Chan}(B * D))\cr
    \otimes\cr
    1(\tx{In}(B * D))
  \end{pmatrix}
$$
Then, applying what we know about $A * B$ in general, we get:
$$
  \begin{pmatrix}
    \tx{NoChan}(A)\cr
    \otimes\cr
    \tx{NoChan}(C)
  \end{pmatrix}
  \circ
  \begin{pmatrix}
    \tx{NoChan}(B)\cr
    \otimes\cr
    1(\tx{Channels}(\tx{Chan}(A)))\cr
    \otimes\cr
    \tx{NoChan}(D)\cr
    \otimes\cr
    1(\tx{Channels}(\tx{Chan}(C))
  \end{pmatrix}
  \circ
  \begin{pmatrix}
    \tx{Channels}(\tx{Chan}(A, B, C, D))\cr
    \otimes\cr
    1(\tx{In}(B, D))
  \end{pmatrix}
$$
Applying the interchange lemma for packages again, along with 
what we know about $A \circ B$, we get:
$$
\begin{pmatrix}
  \tx{NoChan}(A \circ B)\cr
  \otimes\cr
  \tx{NoChan}(C \circ D)
\end{pmatrix}
  \circ
  \begin{pmatrix}
    \tx{Channels}(\tx{Chan}(A, B, C, D))\cr
    \otimes\cr
    1(\tx{In}(B, D))
  \end{pmatrix}
$$
Noting that $\tx{Chan}(A, B, C, D) = \tx{Chan}(A \circ B, C \circ D)$,
and that $\tx{In}(B, D) = \tx{In}(A \circ B, C \circ D)$, we realize that the
expression above is equal to:
$$
\tx{SysPack}((A \circ B) * (C \circ D))
$$

$\blacksquare$
\end{lemma}

\begin{definition}[System Games]
Analogously to games, we define a \emph{system game} as a system $S$
with $\tx{In}(S) = \emptyset$.
\end{definition}

\begin{definition}[System Game Reductions]
We can also define notions of reductions for system game (pairs).

First, we define:
$$
\epsilon(\mathcal{A} \circ S_b) := \epsilon(\mathcal{A} \circ \tx{SysPack}(S_b))
$$

We then also use the syntax sugar of:
$$
S_b \leq f(G_b^1, G_b^2, \ldots)
$$
as shorthand for, $\forall \mathcal{A}.\ \exists \mathcal{B}_1, \ldots$:
$$
\epsilon(\mathcal{A} \circ S_b) \leq f(\epsilon(\mathcal{B}_1 \circ G_b^1), \epsilon(\mathcal{B}_2 \circ G_b^2), \ldots)
$$

We also sometimes omit explicitly writing $S_b$, instead writing just $S$,
if it's clear that we're talking about a pair of systems.

$\square$
\end{definition}

Similar properties hold for reductions:

\begin{lemma}
$A \circ G_b \leq G_b$.

\textbf{Proof:} $\tx{SysPack}(A \circ G_b) = \tx{SysPack}(A) \circ \tx{SysPack}(G_b) \leq \tx{SysPack}(G_b)$. $\blacksquare$
\end{lemma}

\begin{lemma}
There exists system games $A$, $G_B$ such that $G_B$ is secure
but $A * G_b$ is insecure.

\textbf{Proof:}
Consider:
\package{$G_b$}{
&\pfn{Cheat}{}\cr
\pind{1}\psend{b}{P}\cr
\pind{1}\precv{\hat{b}}{Q}\cr
\pind{1}\preturn{\hat{b}}\cr
}
\package{$A$}{
&\pfn{Run}{}\cr
\pind{1}\precv{b}{P}\cr
\pind{1}\psend{b}{Q}\cr
}
Clearly, $G_b$ is secure in isolation, since no other system is present
to provide a value on $Q$, so $G_b$ will block forever in the cheating function.

However, when linked with $A$, this cheating function will return $b$,
allowing an adversary to break the game with probability $1$.

$\blacksquare$
\end{lemma}

\section{Protocols and Composition}

\begin{definition}[Protocols]
A \emph{protocol} $\mathcal{P}$ consists of:
\begin{itemize}
\item Systems $P_1, \ldots, P_n$, called \emph{players}
\item An asynchronous package $F$, called the \emph{ideal functionality}
\item A set $\tx{Leakage} \subseteq \tx{Out}(F)$, called the leakage
\end{itemize}

Furthermore, we also impose requirements on the channels and functions
these elements use.

First, we require that the player systems are jointly closed,
with no extra channels that aren't connected to other players:
$$
\bigcup_{i \in [n]} \text{OutChan}(P_i) = \bigcup_{i \in [n]} \text{InChan}(P_i)
$$

Second, we require that the functions the systems depend on are disjoint,
outside of the ideal functionality:
$$
\forall i, j \in [n].\quad \text{In}(P_i) \cap \text{In}(P_j) \subseteq \tx{Out}(F)
$$

Third, we require that the functions the systems export on are disjoint:
$$
\forall i, j \in [n].\quad \text{Out}(P_i) \cap \text{Out}(P_j) = \emptyset
$$

We can also define a few convenient notations related to the interface of a base
protocol.

Let $\text{Out}_i(\mathcal{P}) := \text{Out}(P_i)$, and let $\text{In}_i(\mathcal{P}) := \text{In}(P_i) / \text{Out}(F)$.
We then define $\text{Out}(\mathcal{P}) := \bigcup_{i \in [n]} \text{Out}_i(\mathcal{P})$
and $\text{In}(\mathcal{P}) := \bigcup_{i \in [n]} \text{In}_i(\mathcal{P})$.
Let $\tx{IdealIn}_i(\mathcal{P}) := \tx{In}(P_i) \cap \tx{Out}(F)$.

Finally, we define
$$
\begin{aligned}
&\text{IdealIn}(\mathcal{P}) := \text{In}(F)\cr
\end{aligned}
$$

$\square$
\end{definition}

\begin{definition}[Closed Protocol]
  We say that a protocol $\mathcal{P}$ is \emph{closed} if
  $\tx{In}(\mathcal{P}) = \emptyset$ and $\tx{IdealIn}(\mathcal{P}) = \emptyset$.

  $\square$
\end{definition}

\begin{definition}[Literal Equality]
Given two protocols $\mathcal{P}$ and $\mathcal{Q}$, we say that
they are \emph{literally equal}, written as $\mathcal{P} \equiv \mathcal{Q}$
when:
\begin{itemize}
\item $\mathcal{P}.n = \mathcal{Q}.n$
\item There exists a permuation $\pi : [n] \leftrightarrow [n]$ such that
$
{\forall i \in [n].\enspace \mathcal{P}.P_i \equiv \mathcal{Q}.P_{\pi(i)}}
$
\item $\mathcal{P}.F = \mathcal{Q}.F$
\item $\mathcal{P}.\tx{Leakage} = \mathcal{Q}.\tx{Leakage}$
\end{itemize}

$\square$
\end{definition}

\begin{definition}[Vertical Composition]
Given an protocol $\mathcal{P}$ and a package $G$, satisfying
$\text{IdealIn}(\mathcal{P}) \subseteq \text{Out}(G)$,
we can define the protocol $\mathcal{P} \circ G$.

$\mathcal{P} \circ G$ has the same players and leakage as $\mathcal{P}$,
but its ideal functionality $F$ becomes $F \circ G$.

$\square$
\end{definition}

\begin{claim}[Vertical Composition is Associative]
For any protocol $\mathcal{P}$, and packages $G, H$, such that their composition
is well defined, we have
$$
\mathcal{P} \circ (G \circ H) = (\mathcal{P} \circ G) \circ H
$$

\txbf{Proof:} This follows from the definition of vertical composition
and the associativity of $\circ$ for packages.
$\blacksquare$
\end{claim}

\begin{definition}[Horizontal Composition]
Given two protocols $\mathcal{P}, \mathcal{Q}$,
we can define the protocol $\mathcal{P} \lhd \mathcal{Q}$,
provided a few requirements hold.

First, we need: $\text{In}(\mathcal{P}) \subseteq \text{Out}(\mathcal{Q})$.
We also require that the functions exposed by a player in $\mathcal{Q}$
are used by \emph{exactly} one player in $\mathcal{P}$.
We express this as:
\[
  \forall i \in [\mathcal{Q}.n].\ \exists! j \in [\mathcal{P}.n].\quad \text{In}_j \cap \text{Out}_i \neq \emptyset
\]

Second, we require that the players share no channels between the two
protocols.
In other words $\text{Chan}(\mathcal{P}.P_i) \cap \text{Chan}(\mathcal{Q}.P_j) = \emptyset$, for all $P_i, P_j$.

Third, we require that the ideal functionalities of one protocol aren't used in the other.
$$
\begin{aligned}
&\tx{Out}(\mathcal{P}.F) \cap \tx{In}(\mathcal{Q}) = \emptyset\cr
&\tx{Out}(\mathcal{Q}.F) \cap \tx{In}(\mathcal{P}) = \emptyset
\end{aligned}
$$

Finally, we require that the ideal functionalities do not overlap, 
  in the sense that $\text{Out}(\mathcal{P}.F) \cap \text{Out}(\mathcal{Q}.F) = \emptyset$

Our first condition has an interesting consequence: every player $\mathcal{Q}.P_j$
has its functions used by exactly one player $\mathcal{P}.P_i$.
In that case, we say that $\mathcal{P}.P_i$ \emph{uses} $\mathcal{Q}.P_j$.

With this in hand, we can define $\mathcal{P} \lhd \mathcal{Q}$.

The players will consist of:
$$
  \mathcal{P}.P_i \circ
  \begin{pmatrix}
    {\displaystyle \bigast_{\mathcal{Q}.P_j \text{ used by } \mathcal{P}.P_i } \mathcal{Q}.P_j} \cr
    \otimes\cr
    1(\tx{IdealIn}_i)
  \end{pmatrix}
$$
And, because of our assumption, each player in $\mathcal{Q}$ appears
somewhere in this equation.

The ideal functionality is $\mathcal{P}.F \otimes \mathcal{Q}.F$,
and the leakage is $\mathcal{P}.\tx{Leakage} \cup \mathcal{Q}.\tx{Leakage}$.

We can also easily show that this definition is well defined, satisfying
the required properties of an protocol.
Because of the definition of the players, we see that:
$$
  \bigcup_{i \in [(\mathcal{P} \lhd \mathcal{Q}).n]} \tx{OutChan}((\mathcal{P} \lhd \mathcal{Q}).{P_i})
  = \left(\bigcup_{i \in [\mathcal{P}.n]} \tx{OutChan}(\mathcal{P}.P_i)\right) \cup
  \left(\bigcup_{i \in [\mathcal{Q}.n]} \tx{OutChan}(\mathcal{Q}.P_i)\right)
$$
  since $\text{OutChan}(A \circ B) = \text{OutChan}(A \otimes B) = \text{OutChan}(A, B)$.
A similar reasoning applies to $\text{InChan}$, allowing us to conclude that:
$$
  \bigcup_{i \in [(\mathcal{P} \lhd \mathcal{Q}).n]} \tx{OutChan}((\mathcal{P} \lhd \mathcal{Q}).{P_i}) =
  \bigcup_{i \in [(\mathcal{P} \lhd \mathcal{Q}).n]} \tx{InChan}((\mathcal{P} \lhd \mathcal{Q}).{P_i})
$$
as required.

By definition, the dependencies $\text{In}$ of each player in $\mathcal{P} \lhd \mathcal{Q}$
are the union of several players in $\mathcal{Q}$, and
the ideal dependencies of players in $\mathcal{P}$,
both of these are required to be disjoint, so disjointness property
continues to hold.

Finally, since each player is of the form $\mathcal{P}.P_i \circ \ldots$,
the condition on $\text{Out}_i$ is also satisfied in $\mathcal{P} \lhd \mathcal{Q}$,
since $\mathcal{P}$ does.

$\square$

\end{definition}

\begin{lemma}
Horizontal composition is associative, i.e.
${\mathcal{P} \lhd (\mathcal{Q} \lhd \mathcal{R}) \equiv (\mathcal{P} \lhd \mathcal{Q}) \lhd \mathcal{R}}$
for all protocols $\mathcal{P}, \mathcal{Q}, \mathcal{R}$ where this expression is well defined.

$\txbf{Proof:}$
For the ideal functionalities, it's clear that by the associativity
of $\otimes$ for systems, the resulting functionality is the same
in both cases.

The trickier part of the proof is showing that the resulting players
are identical.

It's convenient to define a relation for the players in $\mathcal{R}$
that get used in $\mathcal{P}$ via the players in $\mathcal{Q}$.
To that end, we say that $\mathcal{P}.P_i$ \emph{uses} $\mathcal{R}.P_j$
if there exists $\mathcal{Q}.P_k$ such that $\mathcal{P}.P_i$ uses
$\mathcal{Q}.P_k$, and $\mathcal{Q}.P_k$ uses $\mathcal{R}.P_j$.

The players of $\mathcal{P} \lhd (\mathcal{Q} \lhd \mathcal{R})$ are of the form:
$$
  \mathcal{P}.P_i \circ
  \begin{pmatrix}
  \displaystyle \bigast_{\mathcal{Q}.P_j \text{ used by } \mathcal{P}.P_i } 
  \mathcal{Q}.P_j \circ
  \begin{pmatrix}
  \displaystyle \bigast_{\mathcal{R}.P_k \text{ used by } \mathcal{Q}.P_j } \mathcal{R}.P_k\cr
  \otimes\cr
  1(\mathcal{Q}.\tx{IdealIn}_j)
  \end{pmatrix}\cr
  \otimes\cr
  1(\mathcal{P}.\tx{IdealIn}_i)
  \end{pmatrix}
$$
While those in $(\mathcal{P} \lhd \mathcal{Q}) \mathcal{R}$ are of the form:
$$
  \left( \mathcal{P}.P_i \circ
  \begin{pmatrix}
  \displaystyle \bigast_{\mathcal{Q}.P_j \text{ used by } \mathcal{P}.P_i } 
  \mathcal{Q}.P_j\cr
  \otimes\cr
  1(\mathcal{P}.\tx{IdealIn}_i)
  \end{pmatrix}
  \right)
  \circ
  \begin{pmatrix}
    \displaystyle \bigast_{\mathcal{R}.P_k \text{ used by } \mathcal{P}.P_i } \mathcal{R}.P_k\cr
  \otimes\cr
  1(\mathcal{Q}.\tx{IdealIn}_j)
  \end{pmatrix}
$$
Now, we can apply the associativity of $\circ$ for systems, and also
group the $\mathcal{R}.P_k$ players based on which $\mathcal{Q}.P_j$ uses them:
$$
  \mathcal{P}.P_i \circ
  \begin{pmatrix}
  \displaystyle \bigast_{\mathcal{Q}.P_j \text{ used by } \mathcal{P}.P_i } 
  \mathcal{Q}.P_j\cr
  \otimes\cr
  1(\mathcal{P}.\tx{IdealIn}_i)
  \end{pmatrix}
  \circ
  \left(
    \displaystyle \bigast_{\mathcal{Q}.P_j}
  \begin{pmatrix}
  \displaystyle \bigast_{\mathcal{R}.P_k \text{ used by } \mathcal{Q}.P_j } \mathcal{R}.P_k
  \cr
  \otimes\cr
  1(\mathcal{Q}.\tx{IdealIn}_j)
  \end{pmatrix}
  \right)
$$
Now, the conditions are satisfied for applying the interchange lemma (Lemma~\ref{thm:interchange_system}),
giving us:
$$
  \mathcal{P}.P_i \circ
  \begin{pmatrix}
  \displaystyle \bigast_{\mathcal{Q}.P_j \text{ used by } \mathcal{P}.P_i } 
  \mathcal{Q}.P_j \circ
  \begin{pmatrix}
  \displaystyle \bigast_{\mathcal{R}.P_k \text{ used by } \mathcal{Q}.P_j } \mathcal{R}.P_k\cr
  \otimes\cr
  1(\mathcal{Q}.\tx{IdealIn}_j)
  \end{pmatrix}\cr
  \otimes\cr
  1(\mathcal{P}.\tx{IdealIn}_i)
  \end{pmatrix}
$$
Which is non other than the players in $\mathcal{P} \lhd (\mathcal{Q} \lhd \mathcal{R})$.

$\blacksquare$
\end{lemma}

\begin{definition}[Concurrent Composition]
Given two protocols $\mathcal{P}, \mathcal{Q}$,
we can define their concurrent composition---or tensor product---
$\mathcal{P} \otimes \mathcal{Q}$, provided a few requirements hold.
We require that:
\begin{enumerate}
\item $\tx{In}(\mathcal{P}) \cap \tx{In}(\mathcal{Q}) = \emptyset$.
\item $\tx{Out}(\mathcal{P}) \cap \tx{Out}(\mathcal{Q}) = \emptyset$.
\item $\tx{Out}(\mathcal{P}.F) \cap \tx{Out}(\mathcal{Q}.F) = \emptyset$ \emph{or} $\mathcal{P}.F = \mathcal{Q}.F$.
\item $\tx{Leakage}(\mathcal{P}) \cap \tx{In}(\mathcal{Q}) = \emptyset = \tx{Leakage}(\mathcal{Q}) \cap \tx{In}(\mathcal{P})$
\end{enumerate}

The players of $\mathcal{P} \otimes \mathcal{Q}$ consist of all the players
in $\mathcal{P}$ and $\mathcal{Q}$.
The ideal functionality is $\mathcal{P}.F \otimes \mathcal{Q}.F$, 
unless $\mathcal{P}.F = \mathcal{Q}.F$, in which case the ideal functionality
is simply $\mathcal{P}.F$.
In either case, the leakage is $\mathcal{P}.\tx{Leakage} \cup \mathcal{Q}.\tx{Leakage}$.
This use of $\otimes$ is well defined by assumption.

The resulting protocol is also clearly well defined.

The jointly closed property holds because we've simply taken the union
of both player sets.

Since $\tx{In}(\mathcal{P}) \cap \tx{In}(\mathcal{Q}) = \emptyset$,
it also holds that for every $P_i, P_j$ in $\mathcal{P} \otimes \mathcal{Q}$,
we have $\tx{In}(P_i) \cap \tx{In}(P_j) = \emptyset$,
since each player comes from either $\mathcal{P}$ or $\mathcal{Q}$.
      
Finally, $\tx{Out}(\mathcal{P}) \cap \tx{Out}(\mathcal{Q}) = \emptyset$,
we have that $\tx{Out}(P_i) \cap \tx{Out}(P_j) = \emptyset$,
by the same reasoning.
    
$\square$
\end{definition}

\todo{The reason why we allow for $F = G$ is so that you can have like the same $1$}

\begin{lemma}
Concurrent composition is associative and commutative.
I.e. $\mathcal{P} \otimes (\mathcal{Q} \otimes \mathcal{R}) \equiv (\mathcal{P} \otimes \mathcal{Q}) \otimes \mathcal{R}$,
and $\mathcal{P} \otimes \mathcal{Q} \equiv \mathcal{Q} \otimes \mathcal{P}$ for
all protocols $\mathcal{P}, \mathcal{Q}, \mathcal{R}$ where these expressions
are well defined.

\txbf{Proof:}

By the definition of $\equiv$, all that matter is the \emph{set} of players,
and not their order.
Because $\cup$ is associative, and so is $\otimes$ for systems,
we conclude that concurrent composition is associative as well,
since the resulting set of players and ideal functionality are the same
in both cases.

Similarly, since $\cup$ and $\otimes$ (for systems) are commutative,
we conclude that concurrenty composition is commutative.

$\blacksquare$
\end{lemma}

\subsection{Corruption and Simulation}

\begin{definition}[``Honest'' Corruption]
Given a system $P$,
we define the ``honest'' corruption of $P$
$$
\tx{Corrupt}_H(P) := P
$$

This is clearly equality preserving, by tautology.

$\square$
\end{definition}

\begin{definition}[Semi-Honest Corruption]
Given a system $P$, we can define
the semi-honest corruption $\tx{Corrupt}_{\tx{SH}}(P)$.

This is a transformation of
of $P$, providing access to its ``view''.
More formally, $\tx{Corrupt}_{\tx{SH}}(P)$ is a system which works the same
as $P$, but with an additional public variable $\tx{log}$,
which contains several sub logs:
\begin{enumerate}
  \item $\tx{log}.{A_i}$ for each sending channel $A_i$,
  \item $\tx{log}.{B_i}$ for each receiving channel $B_i$,
  \item $\tx{log}.F$ for each input function $F$.
  \item $\tx{log}.G$ for each output function $G$.
\end{enumerate}
Each of these sub logs is initialized with ${\tx{log}.\bullet \gets \tx{FifoQueue.New()}}$.
Additionally, $\tx{Corrupt}_{\tx{SH}}(P)$ modifies $P$ by pushing events to these
logs at different points in time.
These events are:
\begin{itemize}
\item $(\texttt{call}, (x_1, \ldots, x_n))$ to $\tx{log}.F$ when a function call $F(x_1, \ldots, x_n)$ happens.
\item $(\texttt{ret}, y)$ to $\tx{log}.F$ when the function $F$ returns a value $y$.
\item $(\texttt{input}, (x_1, \ldots, x_n))$ to $\tx{log}.G$ when the function $G$ is called with $(x_i, \ldots)$ as input.
\item $m$ to $\tx{log}.A$ when a value $m$ is sent on channel $A$.
\item $m$ to $\tx{log}.B$ when a value $m$ is received on channel $B$.
\end{itemize}

This transformation is also equality respecting.
First, note that if $P \equiv P'$ as systems, then
then $\tx{NoChan}(P) = \tx{NoChan}(P')$, and so their logs will be the same.

$\square$
\end{definition}

\begin{definition}[Malicious Corruption]
Given a system $P$ with:
$$
\begin{aligned}
  &\tx{In}(P) = \{F_1, \ldots, F_n\}\cr
  &\tx{OutChan}(P) = \{A_1, \ldots, A_m\}\cr
  &\tx{InChan}(P) = \{B_1, \ldots, B_l\}\cr
\end{aligned}
$$
we define the malicious corruption $\tx{Corrupt}_M(P)$ as the following game:
\package{$\tx{Corrupt}_M(P)$}{
&\underline{\tx{Call}_{F_i}((x_1, \ldots, x_n))\tx{:}}\cr
\pind{1} \preturn{F_i(x_1, \ldots, x_n)}\cr
\cr
&\underline{\tx{Send}_{A_i}(m)\tx{:}}\cr
\pind{1} \psend{m}{A_i}\cr
\cr
&\underline{\tx{Test}_{B_i}()\tx{:}}\cr
\pind{1} \preturn{\txbf{test } B_i}\cr
\cr
&\underline{\tx{Recv}_{B_i}()\tx{:}}\cr
\pind{1} \preturn{\precv{m}{B_i}}\cr
}

In other words, malicious corruption provides access to the functions
and channels used by $P$, but no more than that.

This is also equality preserving, since $\tx{Corrupt}_M(P)$ depends
only on the channels used by $P$ and the functions called by $P$,
all of which are the same for any $P' \equiv P$.

$\square$
\end{definition}

\begin{lemma}[Simulating Corruptions]
  \label{thm:simulatingcorruption}
  We can simulate corruptions using strong forms of corruption.
  In particular, there exists systems $S_{\tx{SH}}$ and $S_{\tx{H}}$ such that
  for all systems $P$, we have:
  \[
    \begin{aligned}
      &\tx{Corrupt}_{\tx{SH}}(P) = S_{\tx{SH}} \circ \tx{Corrupt}_M(P)\cr
      &\tx{Corrupt}_{\tx{H}}(P) = S_{\tx{H}} \circ \tx{Corrupt}_{\tx{SH}}(P)
    \end{aligned}
  \]

\txbf{Proof:}
For the simulation of honest corruption, we can simply ignore
  the additional log variable, and set $S_{\tx{H}} := 1(\tx{Out}(P))$.

For semi-honest corruption, $S_{\tx{SH}}$ is formed by first transforming
$\tx{Corrupt}_{\tx{SH}}(P)$, replacing:
\begin{itemize}
  \item every function call with $\tx{Call}_{F_i}(\ldots)$,
  \item every sending of a message $m$ on $A$ with $\tx{Send}_A(m)$,
  \item every length test of $B$ with $\tx{Test}_B()$,
  \item every reception of a message on $B$ with $\tx{Recv}_B()$.
\end{itemize}

The result is clearly a perfect emulation of semi-honest corruption
using malicious corruption.

$\blacksquare$
\end{lemma}

Sometimes, it's useful to be able to talk about corruptions in general,
in which case we write $\text{Corrupt}_\kappa(P)$,
for $\kappa \in \{\tx{H}, \tx{SH}, \tx{M}\}$.

\begin{definition}[Corruption Models]
Given a protocol $\mathcal{P}$ with players $P_1, \ldots, P_n$, a \emph{corruption model} $C$
is a function $C : [\mathcal{P}.n] \to \{\tx{H}, \tx{SH}, \tx{M}\}$.
This provides a corruption $C_i$ associated with each player $P_i$.
We can then define $\text{Corrupt}_C(P_i) := \text{Corrupt}_{C_i}(P_i)$.

Corruption models have a natural partial order associated with them. 
We have:
$$
\tx{H} < \tx{SH} < \tx{M}
$$
  and then we say that $C \geq C'$ if $\forall i \in [n]. \quad C_i \geq C'_i$.

A \emph{class of corruptions} $\mathcal{C}$ is simply a set of corruption models.

$\square$
\end{definition}

Some common classes are:
\begin{itemize}
  \item The class of malicious corruptions, where all but one player is malicious.
  \item The class of malicious corruptions, where all but one player is semi-honest.
\end{itemize}

\begin{definition}[Instantiation]
  Given a protocol $\mathcal{P}$ with $\tx{In}(\mathcal{P}) = \emptyset$, and a corruption model $C$, we can
  define an \emph{instantiation} $\tx{Inst}_C(\mathcal{P})$, which
  is a system defining the semantics of the protocol.

  First, we need to define a transformation of systems to use
  a \emph{router} $\mathcal{R}$, which will be a special system
  allowing an adversary to control the order of delivery of messages.

  Let $\{A_1, \ldots, A_n\} = \tx{Chan}(P_1, \ldots, P_n)$.
  We then define $\mathcal{R}$ as the syten:
\package{$\mathcal{R}$}{
&\underline{\tx{Deliver}_{A_i}()\tx{:}}\cr
\pind{1} \precv{m}{\langle A_i, \mathcal{R} \rangle}\cr
\pind{1} \psend{m}{\langle \mathcal{R}, A_i \rangle}\cr
}

  Next, we define a transformation $\tx{Routed}(S)$ of a system,
  which makes communication pass via the router:
  \begin{itemize}
    \item Whenever $S$ sends $m$ via $A$, $\tx{Routed}(S)$ sends $m$ via $\langle A , \mathcal{R} \rangle$.
    \item Whenever $S$ receives $m$ via $B$, $\tx{Routed}(S)$ recieves $m$ via $\langle \mathcal{R}, B \rangle$.
  \end{itemize}

With this in hand, we define:
$$
\tx{Inst}_C(\mathcal{P}) :=
  \begin{pmatrix}
    {\displaystyle \bigast}_{i \in [n]} \tx{Routed}(\tx{Corrupt}_C(P_i))\cr
    *\cr
    \mathcal{R}\cr
    \otimes\cr
    1(\tx{Leakage})
  \end{pmatrix}
  \circ F
$$


$\square$
\end{definition}

\begin{lemma}[Properties of $\tx{Routed}$]
  \label{thm:routed}
  For any systems $A, B$, we have:
$$
\begin{aligned}
  &\tx{Routed}(A \circ B) = \tx{Routed}(A) \circ \tx{Routed}(B)\cr
  &\tx{Routed}(A * B) = \tx{Routed}(A) * \tx{Routed}(B)\cr
  &\tx{Routed}(A \otimes B) = \tx{Routed}(A) \otimes \tx{Routed}(B)\cr
\end{aligned}
$$
(provided these expressions are well defined)

\txbf{Proof:} The $\tx{Routed}$ transformation simply
renames each sending and receiving channel in a system.
In all the cases above, even $A * B$, all of the channels present
in $A$ and $B$ are present in the composition, and so all
of these equations hold.

$\blacksquare$
\end{lemma}


\begin{definition}[Compatible Corruptions]
  \label{def:compatc}
  Given protocols $\mathcal{P}, \mathcal{Q}$, and a corruption model
  $C$ for $\mathcal{Q}$, we can define a notion of a \emph{compatible}
  corruption model $C'$ for $\mathcal{P} \otimes \mathcal{Q}$ or $\mathcal{P} \circ \mathcal{Q}$,
  provided these expressions are well defined.

  A corruption model $C'$ for $\mathcal{P} \otimes \mathcal{Q}$.
  is compatible with $C$ when every corruption of a player
  in $\mathcal{Q}$ is $\geq$ that of the corresponding corruption in $C$.

  We say that a corruption model $C'$ for $\mathcal{P} \circ \mathcal{Q}$ is compatible with
a corruption model $C$ for $\mathcal{Q}$ if for every
$\mathcal{Q}.P_j$ used by $\mathcal{P}.P_i$, the corruption
level of $\mathcal{Q}.P_j$ in $\mathcal{C}'$ is $\geq$ the corruption level of $\mathcal{P}.P_i$
in $\mathcal{C}$.

  Furthermore, we say that $C'$ is \emph{strictly} compatible
  with $C$ if the above property holds with $=$, and not just $\geq$.

  This extends to corruption \emph{classes} as well.
  A corruption class $\mathcal{C}'$ is (strictly) compatible with a class $\mathcal{C}$,
  if every $C' \in \mathcal{C}'$ is (strictly) compatible with some $C \in \mathcal{C}$.

  $\square$
\end{definition}

\begin{theorem}[Concurrent Breakdown]
  \label{thm:concurrent_breakdown}
  Given protocols $\mathcal{P}, \mathcal{Q}$, and a corruption model $C$
  for $\mathcal{Q}$, then for any corruption model $C'$ for $\mathcal{P} \otimes \mathcal{Q}$ compatible with $C$, we have:
  \[
    \tx{Inst}_{C'}(\mathcal{P} \otimes \mathcal{Q}) = \tx{Inst}_{C'}(\mathcal{P}) \otimes \tx{Inst}_C(\mathcal{Q})
  \]
\txbf{Proof:} If we unroll $\text{Inst}_{C'}(\mathcal{P} \otimes \mathcal{Q})$, we get:
$$
\begin{pmatrix}
\mathcal{R}\cr
*\cr
\left(\bigast_{i \in [\mathcal{P}.n]} \tx{Routed}(\tx{Corrupt}_{C'}(\mathcal{P}.P_i))\right)\cr
*\cr
\left(\bigast_{i \in [\mathcal{Q}.n]} \tx{Routed}(\tx{Corrupt}_{C'}(\mathcal{Q}.P_i))\right)\cr
\otimes\cr
1(\mathcal{P}.\tx{Leakage}, \mathcal{Q}.\tx{Leakage})
\end{pmatrix}
\circ
\begin{pmatrix}
\mathcal{P}.F\cr
\otimes\cr
\mathcal{Q}.F\cr
\end{pmatrix}
$$

We can apply a few observations here:
\begin{enumerate}
  \item Since $\mathcal{C}'$ is compatible with $\mathcal{C}$, then $\mathcal{Q}.P_i$ follows a corruption from $\mathcal{C}$.
  \item $\mathcal{R}$ can be written as $\mathcal{R}_\mathcal{P} \otimes \mathcal{R}_\mathcal{Q}$,
  with one system using channels in $\mathcal{P}$, and the other using channels in $\mathcal{Q}$.
  \item Since protocols are closed, we can use $\otimes$ between the players in $\mathcal{P}$ and $\mathcal{Q}$,
  since they never send messages to each other.
\end{enumerate}
This results in the following:
$$
\begin{pmatrix}
  \mathcal{R}_{\mathcal{P}} * \left(\bigast_{i \in [\mathcal{P}.n]} \tx{Routed}(\tx{Corrupt}_{C'}(\mathcal{P}.P_i))\right) \otimes 1(\mathcal{P}.\tx{Leakage})\cr
\otimes\cr
  \mathcal{R}_{\mathcal{Q}} * \left(\bigast_{i \in [\mathcal{Q}.n]} \tx{Routed}(\tx{Corrupt}_{C}(\mathcal{Q}.P_i))\right) \otimes 1(\mathcal{Q}.\tx{Leakage})
\end{pmatrix}
\circ
\begin{pmatrix}
\mathcal{P}.F\cr
\otimes\cr
\mathcal{Q}.F\cr
\end{pmatrix}
$$
From here, we apply Lemma~\ref{thm:interchange_system} (interchange), to get:
$$
\begin{matrix}
\tx{Inst}_{C'}(\mathcal{P})\cr
\otimes\cr
\tx{Inst}_{C}(\mathcal{Q})\cr
\end{matrix}
$$

$\blacksquare$

\end{theorem}

\begin{theorem}[Horizontal Breakdown]
  \label{thm:horizontal_breakdown}
  Given protocols $\mathcal{P}, \mathcal{Q}$, and a corruption
  model $C$ for $\mathcal{Q}$, then for any compatible corruption
  model $C'$ for $\mathcal{P} \lhd \mathcal{Q}$, there exists
  systems $S_1, \ldots, S_{\mathcal{Q}.n}$ and a set $L_{\mathcal{Q}}$ such that:
  $$
  \tx{Inst}_{C'}(\mathcal{P} \lhd \mathcal{Q}) =
  1(O)\circ
  \begin{pmatrix}
    {\displaystyle \bigast}_{i \in [\mathcal{P}.n]} \tx{Routed}(\tx{Corrupt}'_{C'}(\mathcal{P}.P_i))
    \cr
    *\cr
    \mathcal{R}_{\mathcal{P}}\cr
    \otimes\cr
    1(\tx{Leakage}, L_{\mathcal{Q}})
  \end{pmatrix}
  \circ
  \begin{pmatrix}
    \mathcal{P}.F\cr
    \otimes\cr
    1(\tx{Out}(\mathcal{R}_q))\cr
    \otimes\cr
    1(\mathcal{Q}.\tx{Leakage})\cr
    \otimes\cr
    \bigotimes_{i \in [\mathcal{Q}.n]} S_i\cr
  \end{pmatrix}
  \circ
  \begin{pmatrix}
  \tx{Inst}_C(\mathcal{Q})\cr
  \otimes\cr
  1(\tx{In}(\mathcal{P}.F))
  \end{pmatrix}
  $$
  where $O := \tx{Out}(\tx{Inst}_{C'}(\mathcal{P} \lhd \mathcal{Q}))$,
  $\mathcal{R}_{\mathcal{P}} \circ \mathcal{R}_{\mathcal{Q}} = \mathcal{R}$
  are a decomposition of the router $\mathcal{R}$ for $\mathcal{P} \lhd \mathcal{Q}$,
  and $\tx{Corrupt}'_{C'}(\ldots)$ is the same as $\tx{Corrupt}_{C'}$,
  except that malicious corruption contains no $\tx{Call}_{F_i}$ functions,
  for $F_i \notin \tx{Out}(\mathcal{P}.F)$

  Furthermore, if the models are \emph{strictly} compatible,
  then $S_j = 1(\tx{Out}(\tx{Routed}(\tx{Corrupt}_C(\mathcal{Q}.P_i))))$.

\txbf{Proof:} We start by unrolling $\tx{Inst}_{C'}(\mathcal{P} \lhd \mathcal{Q})$,
to get:
\[
\tx{Inst}_C(\mathcal{P} \lhd \mathcal{Q}) =
  \begin{pmatrix}
    {\displaystyle \bigast}_{i \in [\mathcal{P}.n]} \tx{Routed}\left(\tx{Corrupt}_{C'}\left(\mathcal{P}.P_i \circ 
        
    \begin{pmatrix}
    \bigast_{\mathcal{Q}.P_j \tx{ used by } \mathcal{P}.P_i} \mathcal{Q}.P_j\cr
    \otimes\cr
    1(\tx{IdealIn}_i)
    \end{pmatrix}
    \right)\right)\cr
    *\cr
    \mathcal{R}\cr
    \otimes\cr
    1(\tx{Leakage})
  \end{pmatrix}
  \circ \begin{pmatrix}
    \mathcal{P}.F\cr
    \otimes \cr
    \mathcal{Q}.F
  \end{pmatrix}
\]
Our strategy will be to progressively build up an equivalent system
to this one, starting with $\tx{Corrupt}_C$, then $\tx{Routed}$, etc.

First, some observations about $\tx{Corrupt}_\kappa(P \circ (1(I) \otimes Q_1 * \cdots * Q_m))$,
where ${I \cap \tx{In}(Q_1, \ldots) = \emptyset}$.

In the case of malicious corruption, we have:
$$
\tx{Corrupt}_M(P \circ (1(I) \otimes Q_1 * \cdots)) =
1(O) \circ
\begin{pmatrix}
  \tx{Corrupt}'_M(P)\cr
  \otimes\cr
  1(\tx{Out}(\tx{Corrupt}_M(Q_1)), \ldots)\cr
\end{pmatrix}
\circ
\begin{pmatrix}
  1(I)\cr
  \otimes\cr
  \tx{Corrupt}_M(Q_1)\cr
  *\cr
  \cdots\cr
\end{pmatrix}
$$
for $O = \tx{Out}(\tx{Corrupt}_M(P \circ (Q_1 * \cdots)))$.
This holds by definition, since corruption $P \circ (Q_1 * \cdots)$ precisely allows
sending messages on behalf of $P$ or any $Q_i$, as well as calling
the input functions to the $Q_i$ systems.
Since we can't call the functions that $P$ uses,
we use $\tx{Corrupt}'_{M}$, which modifies malicious corruption to only
contain $\tx{Send}_{A_i}$, $\tx{Test}_{B_i}$, $\tx{Recv}_{B_i}$,
and $\tx{Call}_{F_i}$ for $F_i \in I$.
In particular the $\tx{Call}_{\bullet}$ functions are omitted for the functions
provided by $Q_1, \ldots, Q_m$.
We can write this expression more concisely,
using $1(L^M)$ for $L^M = \tx{Out}(\tx{Corrupt}_M(Q_1)) \cup \cdots$.

Next, we look at semi-honest corruption.
Because the logs are divided into independent sub logs, we can write:
$$
\tx{Corrupt}_{\tx{SH}}(P \circ (1(I) \otimes Q_1 * \cdots)) =
1(O) \circ
\begin{pmatrix}
  \tx{Corrupt}_{\tx{SH}}(P)\cr
  \otimes\cr
  1(\{Q_1.\tx{log}, \ldots\})
\end{pmatrix}
\circ
\begin{pmatrix}
  1(I)\cr
  \otimes\cr
  \tx{Corrupt}_{\tx{SH}}(Q_1)\cr
  *\cr
  \cdots
\end{pmatrix}
$$
where $O = \tx{Out}(\tx{Corrupt}_{\tx{SH}}(P \circ (Q_1 * \cdots)))$

And for honest corruption, we have
$$
\tx{Corrupt}_{\tx{H}}(P \circ (1(I) \otimes Q_1 * \cdots)) = P \circ (1(I) \otimes Q_1 * \cdots)
$$

Now, the compatibility condition of $C'$ relative to $C$
does not guarantee that if $\mathcal{P}.P_i$ uses $\mathcal{Q}.P_j$,
then $\mathcal{Q}.P_j$ has the same level of corruption: 
it only guarantees a level of corruption at least as strong.
By Lemma~\ref{thm:simulatingcorruption}, we can simulate a weaker
form of corruption using a stronger form, via some simulator system $S$,
depending on the levels of corruption.

Using these simulators, we get, slightly different results based
on the level of corruption.

When $C'_i = \tx{M}$:
$$
\tx{Corrupt}_{C'}((\mathcal{P} \lhd \mathcal{Q}).P_i) =
1(O_i) \circ
\begin{pmatrix}
  \tx{Corrupt}'_{C'}(\mathcal{P}.P_i)\cr
  \otimes\cr
  1(L_i)\cr
\end{pmatrix}
\circ
\begin{pmatrix}
\displaystyle \bigast_{\mathcal{Q}.P_j \tx{ used by } \mathcal{P}.P_i}
  \tx{Corrupt}_C(\mathcal{Q}.P_j)\cr
\otimes\cr
1(\tx{IdealIn}_i)
\end{pmatrix}
$$
with $O_i = \tx{Out}(\tx{Corrupt}_{C'}(\mathcal{P} \lhd \mathcal{Q}).P_i)$, $L_i= \bigcup_{\mathcal{Q}.P_j \tx{ used by } \mathcal{P}.P_i} \tx{Out}(\tx{Corrupt}_M(\mathcal{Q}.P_j))$.
No simulation is needed, since the compatibility of $C'$ with $C$
guarantees that all of the players used by $\mathcal{P}.P_i$
are maliciously corrupted.

When $C'_i = \tx{SH}$:
$$
\tx{Corrupt}_{C'}((\mathcal{P} \lhd \mathcal{Q}).P_i) =
1(O_i) \circ
\begin{pmatrix}
  \tx{Corrupt}_{\tx{C'}}(P)\cr
  \otimes\cr
  1(L_i)
\end{pmatrix}
\circ
\begin{pmatrix}
\displaystyle \bigast_{\mathcal{Q}.P_j \tx{ used by } \mathcal{P}.P_i}
  S_j \circ \tx{Corrupt}_C(\mathcal{Q}.P_j)\cr
  \otimes\cr
1(\tx{IdealIn}_i)
\end{pmatrix}
$$
with $O_i = \tx{Out}(\tx{Corrupt}_{C'}(\mathcal{P} \lhd \mathcal{Q}).P_i)$,
$L_i = \{\mathcal{Q}.P_j.\tx{log} \mid \mathcal{Q}.P_j \tx{ used by } \mathcal{P}.P_i \}$,
and $S_j$ depending on the level of corruption for $\mathcal{Q}.P_j$ in $C$:
\begin{itemize}
  \item $S_j = S_{\tx{SH}}$ if $C_j = \tx{M}$
  \item $S_j = 1$ if $C_j = \tx{SH}$
\end{itemize}

When $C'_i = \tx{H}$:
$$
\tx{Corrupt}_{C'}((\mathcal{P} \lhd \mathcal{Q}).P_i) =
  \tx{Corrupt}_{\tx{C'}}(P)
\circ
\begin{pmatrix}
\displaystyle \bigast_{\mathcal{Q}.P_j \tx{ used by } \mathcal{P}.P_i}
  S_j \circ \tx{Corrupt}_C(\mathcal{Q}.P_j)\cr
\otimes\cr
1(\tx{IdealIn}_i)
\end{pmatrix}
$$
with $S_j$ depending on the level of corruption for $\mathcal{Q}.P_j$ in $C$:
\begin{itemize}
  \item $S_j = S_{\tx{H}} \circ S_{\tx{SH}}$ if $C_j = \tx{M}$
  \item $S_j = S_{\tx{H}}$ if $C_j = \tx{SH}$
  \item $S_j = 1$ if $C_j = \tx{H}$
\end{itemize}

We can unify these three cases, writing:
$$
\tx{Corrupt}'_{C'}((\mathcal{P} \lhd \mathcal{Q}).P_i) =
1(O_i) \circ
\begin{pmatrix}
  \tx{Corrupt}_{\tx{C'}}(P)\cr
  \otimes\cr
  1(L_i)
\end{pmatrix}
\circ
\begin{pmatrix}
\bigast_{\mathcal{Q}.P_j \tx{ used by } \mathcal{P}.P_i}
  S_j \circ \tx{Corrupt}_C(\mathcal{Q}.P_j)\cr
  \otimes\cr
1(\tx{IdealIn}_i)
\end{pmatrix}
$$
with $O_i$ and $L_i$ depending on the corruption level of $\mathcal{P}.P_i$,
and $S_j$ depending on the corruption levels of both $\mathcal{P}.P_i$
and $\mathcal{Q}.P_j$.

By the properties of $\tx{Routed}$ (Lemma~\ref{thm:routed}), we have:
$$
\begin{aligned}
&\tx{Routed}(\tx{Corrupt}'_{C'}((\mathcal{P} \lhd \mathcal{Q}).P_i)) =\cr
&1(O_i) \circ
\begin{pmatrix}
  \tx{Routed}(\tx{Corrupt}'_{\tx{C'}}(P))\cr
  \otimes\cr
  1(L_i)
\end{pmatrix}
\circ
\begin{pmatrix}
\displaystyle\bigast_{\mathcal{Q}.P_j \tx{ used by } \mathcal{P}.P_i}
  S_j \circ \tx{Routed}(\tx{Corrupt}_C(\mathcal{Q}.P_j))\cr
  \otimes\cr
1(\tx{IdealIn}_i)
\end{pmatrix}
\end{aligned}
$$

Next, we need to add the router $\mathcal{R}$.
We note that since $\mathcal{P}$ and $\mathcal{Q}$ have separate channels,
we can write $\mathcal{R} = \mathcal{R}_{\mathcal{P}} \circ \mathcal{R}_{\mathcal{Q}}$,
where the latter contains only the channels in $\mathcal{Q}$,
and the former contains the channels in $\mathcal{P}$,
and provides access to those in $\mathcal{Q}$ via its function dependencies.
Combing this with the interchange lemma, we get:
$$
\begin{aligned}
&\mathcal{R} * \bigast_{i \in [\mathcal{P}.n]}\tx{Routed}(\tx{Corrupt}'_{C'}((\mathcal{P} \lhd \mathcal{Q}).P_i)) * \mathcal{R} =\cr
&1(\tx{Out}(\mathcal{R}), O_1, \ldots, O_{\mathcal{P}.n}) \circ
\begin{pmatrix}
  \tx{Routed}(\tx{Corrupt}_{\tx{C'}}(P))\cr
  *\cr
  \mathcal{R}_{\mathcal{P}}\cr
  \otimes\cr
  1(L_1, \ldots, L_{\mathcal{P}.n})
\end{pmatrix}
\circ
\begin{pmatrix}
\bigast_{j \in [\mathcal{Q}.n]}
  S_j \circ \tx{Routed}(\tx{Corrupt}_C(\mathcal{Q}.P_j))
  \cr
  *\cr
  \mathcal{R}_{\mathcal{Q}}\cr
  \otimes\cr
  1(\tx{Out}(F))
\end{pmatrix}
\end{aligned}
$$

All that remains is to add the ideal functionalities, giving us,
after application of the interchange lemma:
$$
\begin{aligned}
  &\tx{Inst}_{C'}(\mathcal{P} \lhd \mathcal{Q}) =\cr
&1(O) \circ
\begin{pmatrix}
  \tx{Routed}(\tx{Corrupt}'_{\tx{C'}}(P))\cr
  *\cr
  \mathcal{R}_{\mathcal{P}}\cr
  \otimes\cr
  1(\tx{Leakage}, L_{\mathcal{Q}})
\end{pmatrix}
\circ
\begin{pmatrix}
\bigast_{j \in [\mathcal{Q}.n]}
  S_j \circ \tx{Routed}(\tx{Corrupt}_C(\mathcal{Q}.P_j))
  \cr
  *\cr
  \mathcal{R}_{\mathcal{Q}}\cr
  \otimes\cr
  1(\tx{Leakage}, \tx{Out}(F))
\end{pmatrix}
\circ
\begin{pmatrix}
  \mathcal{P}.F\cr
  \otimes\cr
  \mathcal{Q}.F
\end{pmatrix}
\end{aligned}
$$
with $O := \tx{Out}(\tx{Inst}_{C'}(\mathcal{P} \lhd \mathcal{Q}))$,
and $L_{\mathcal{Q}} := \bigcup_{i \in [\mathcal{P}.n]} L_i$.

Now, because $\mathcal{Q}$ does not use any of the functions
in $\mathcal{P}.F$, and because each simulator $S_j$
does not use any channels, we can rewrite this as:
$$
\small
1(O) \circ
\begin{pmatrix}
  \tx{Routed}(\tx{Corrupt}'_{\tx{C'}}(P))\cr
  *\cr
  \mathcal{R}_{\mathcal{P}}\cr
  \otimes\cr
  1(\tx{Leakage}, L_{\mathcal{Q}})
\end{pmatrix}
\circ
\begin{pmatrix}
  \mathcal{P}.F\cr
  \otimes\cr
  1(\tx{Out}(\mathcal{R}_{\mathcal{Q}}))\cr
  \otimes\cr
  1(\mathcal{Q}.\tx{Leakage})\cr
  \otimes\cr
  \bigotimes_{j \in [\mathcal{Q}.n]} S_j
\end{pmatrix}
\circ
\begin{pmatrix}
\begin{pmatrix}
\bigast_{j \in [\mathcal{Q}.n]}
  \tx{Routed}(\tx{Corrupt}_C(\mathcal{Q}.P_j))
  \cr
  *\cr
  \mathcal{R}_{\mathcal{Q}}\cr
  \otimes\cr
  1(\mathcal{Q}.\tx{Leakage})
\end{pmatrix}
\circ
  \mathcal{Q}.F
  \cr
  \otimes\cr
  1(\tx{In}(\mathcal{P}.F))
\end{pmatrix}
$$

We can then notice that the right hand side of this equation
is simply $\tx{Inst}_C(\mathcal{Q})$,
concluding our proof.

$\blacksquare$

\end{theorem}

\subsection{Equality and Simulation}

\begin{definition}[Shape]
  \label{def:shape}
  We say that two protocols $\mathcal{P}, \mathcal{Q}$ have the same \emph{shape}
  if there exists a protocol $\mathcal{Q}' \equiv \mathcal{Q}$ such that:
  \begin{itemize}
    \item $\mathcal{P}.n = \mathcal{Q}'.n$,
    \item $\forall i \in [n].\quad \tx{In}(\mathcal{P}.P_i) = \tx{In}(\mathcal{Q}'.Q_i)$,
    \item $\forall i \in [n].\quad \tx{Out}(\mathcal{P}.P_i) = \tx{Out}(\mathcal{Q}'.Q_i)$,
    \item $\tx{Leakage}(\mathcal{P}) = \tx{Leakage}(\mathcal{Q}')$,
    \item $\tx{IdealIn}(\mathcal{P}) = \tx{IdealIn}(\mathcal{Q}')$.
  \end{itemize}

  $\square$
\end{definition}

\begin{definition}[Semantic Equality]
  We say that two closed protocols $\mathcal{P}$ and $\mathcal{Q}$,
  with the same shape,
  are equal under a class of corruptions $\mathcal{C}$,
  written as $\mathcal{P} =_{\mathcal{C}} \mathcal{Q}$, when we have:
  $$
  \forall C \in \mathcal{C}.\quad \tx{Inst}_C(\mathcal{P}) = \tx{Inst}_C(\mathcal{Q'})
  $$
  as systems, with $\mathcal{Q}' \equiv \mathcal{Q}$ as per 
  Definition~\ref{def:shape}.

  $\square$

\end{definition}

\begin{definition}[Indistinguishability]
  We say that two closed protocols $\mathcal{P}$ and $\mathcal{Q}$,
  with the same shape,
  are \emph{indistinguishable} up to $\epsilon$ under a class of corruptions $\mathcal{C}$,
  written as $\mathcal{P} \overset{\epsilon}{\approx}_{\mathcal{C}} \mathcal{Q}$, when we have:
  $$
  \forall C \in \mathcal{C}.\quad \tx{Inst}_C(\mathcal{P}) \overset{\epsilon}{\approx} \tx{Inst}_C(\mathcal{Q'})
  $$
  as systems, with $\mathcal{Q}' \equiv \mathcal{Q}$ as per 
  Definition~\ref{def:shape}.

  $\square$

\end{definition}

\begin{definition}[Simulated Instantiation]
  A simulator $S$ for a closed protocol $\mathcal{P}$ under a corruption
  model $C$ is a system satisfying:
  \begin{itemize}
    \item $\tx{InChan}(S), \tx{OutChan}(S) = \emptyset$,
    \item $\tx{In}(S) = \tx{Leakage} \cup \left(\bigcup_{C_i = \tx{M}} \tx{Out}(\tx{Corrupt}_{\tx{M}}(P_i))\right) \cup \left(\bigcup_{C_i = \tx{SH}} P_i.\tx{log}\right)$,
    \item $\tx{Out}(S) = \tx{In}(S)$,
  \end{itemize}

  Given such a simulator, we can define the simulated instantiation
  of $\mathcal{P}$ under $C$ with $S$ as:
  $$
  \tx{SimInst}_{S, C}(\mathcal{P}) := 
  \begin{pmatrix}
    S\cr
    \otimes\cr
    1(\tx{Out}(\tx{Inst}_C(\mathcal{P})) / \tx{Out}(S))
  \end{pmatrix}
  \circ \tx{Inst}_C(\mathcal{P})
  $$

  $\square$
\end{definition}

\begin{definition}[Simulatability]
  Given closed protocols $\mathcal{P}, \mathcal{Q}$ with the same shape,
  we say that $\mathcal{P}$ is \emph{simulatable} up to $\epsilon$ by $\mathcal{Q}$
  under a class of corruptions $\mathcal{C}$,
  written as $\mathcal{P} \overset{\epsilon}{\leadsto}_{\mathcal{C}} \mathcal{Q}$,
  when:
  $$
  \forall C \in \mathcal{C}.\exists S.\quad \tx{Inst}_C(\mathcal{P}) \overset{\epsilon}{\approx} \tx{SimInst}_{S, C}(\mathcal{Q}')
  $$
  as systems, with $\mathcal{Q}' \equiv \mathcal{Q}$ as per 
  Definition~\ref{def:shape}.

  $\square$
\end{definition}

\begin{theorem}[Equality Hierarchy]
  \label{thm:equality_hierarchy}
For any corruption class $\mathcal{C}$, we have:
\begin{enumerate}
\item $\mathcal{P} \equiv \mathcal{Q} \implies \mathcal{P} =_\mathcal{C} \mathcal{Q}$.
\item $\mathcal{P} =_{\mathcal{C}} \mathcal{Q} \implies \mathcal{P} \overset{0}{\approx}_\mathcal{C} \mathcal{Q}$.
\item $\mathcal{P} \overset{\epsilon}{\approx}_{\mathcal{C}} \mathcal{Q} \implies \mathcal{P} \overset{\epsilon}{\leadsto}_\mathcal{C} \mathcal{Q}$.
\end{enumerate}

\txbf{Proof:}

\txbf{1.} 
For any $C \in \mathcal{C}$, $\tx{Corrupt}_{C}$ and $\tx{Routed}$ are equality respecting,
so we have:
$$
\forall i \in [n].\quad \tx{Routed}(\tx{Corrupt}_C(\mathcal{P}.P_i)) = 
\tx{Routed}(\tx{Corrupt}_C(\mathcal{Q}.P_i))
$$

Furthermore, the equality of players between $\mathcal{P}$ and $\mathcal{Q}$
makes $\mathcal{P}.\mathcal{R} = \mathcal{Q}.\mathcal{R}$.

And then, the fact that $\mathcal{P}.F = \mathcal{Q}.F$ forces $\tx{Leakage}$
to be the same as well.

Finally, since $\circ, *, \otimes$ are respect $\equiv$, we
can clearly see that $\tx{Inst}_C(\mathcal{P}) = \tx{Inst}_C(\mathcal{Q})$,
since all the sub-components are literally equal.

\txbf{2.} For any systems $A, B$, we have $A = B \implies A \overset{0}{\approx} B$.
Applying this to $\tx{Inst}_C(\mathcal{P})$ and $\tx{Inst}_C(\mathcal{Q})$
gives us our result.

\txbf{3.} It suffices to define a simulator $S$ such that
$\tx{SimInst}_{S, C}(\mathcal{Q}) = \tx{Inst}_C(\mathcal{Q})$,
which will then show our result.
We can simply take $S = 1(\ldots)$ for the right set.

$\blacksquare$
\end{theorem}

\begin{theorem}[Transitivity of Equality]
  \label{thm:prot_trans}
  For any closed protocols $\mathcal{L}, \mathcal{P}, \mathcal{Q}$ with the same shape,
  and any class of corruptions $\mathcal{C}$, we have:
  \begin{enumerate}
    \item $\mathcal{L} =_{\mathcal{C}} \mathcal{P}, \mathcal{P} =_{\mathcal{C}} \mathcal{Q} \implies \mathcal{L} =_{\mathcal{C}}\mathcal{Q}$,
    \item $\mathcal{L} \overset{\epsilon_1}{\approx}_{\mathcal{C}} \mathcal{P}, \mathcal{P} \overset{\epsilon_2}{\approx}_{\mathcal{C}} \mathcal{Q} \implies \mathcal{L} \overset{\epsilon_1 + \epsilon_2}{\approx}_{\mathcal{C}} \mathcal{Q}$,
    \item $\mathcal{L} \overset{\epsilon_1}{\leadsto}_{\mathcal{C}} \mathcal{P}, \mathcal{P} \overset{\epsilon_2}{\leadsto}_{\mathcal{C}} \mathcal{Q} \implies \mathcal{L} \overset{\epsilon_1 + \epsilon_2}{\leadsto}_{\mathcal{C}} \mathcal{Q}$.
  \end{enumerate}

  \txbf{Proof:} The first two parts follow directly from Lemma~\ref{thm:system_trans} (
    transitivity for system equality
  ).
  Indeed, we just look at $\tx{Inst}_C(\mathcal{L})$, $\tx{Inst}_C(\mathcal{P})$, and $\tx{Inst}_C(\mathcal{Q})$
  as systems, for any corruption model $C$.

  For part 3, by assumption we have, for any $C \in \mathcal{C}$:
  \begin{itemize}
  \item $\tx{Inst}_C(\mathcal{L}) \overset{\epsilon_1}{\approx} \begin{pmatrix} S_1\cr \otimes\cr 1(O)\end{pmatrix} \tx{Inst}_C(\mathcal{P})$,
  \item $\tx{Inst}_C(\mathcal{P}) \overset{\epsilon_1}{\approx} \begin{pmatrix} S_2\cr \otimes\cr 1(O)\end{pmatrix} \tx{Inst}_C(\mathcal{Q})$.
  \end{itemize}
  This means that:
  $$
  \tx{Inst}_C(\mathcal{L}) \overset{\epsilon_1 + \epsilon_2}{\approx}
  \begin{pmatrix}
    S_1 \cr
    \otimes\cr
    1(O)
  \end{pmatrix}
  \circ
  \begin{pmatrix}
    S_2 \cr
    \otimes\cr
    1(O)
  \end{pmatrix}
  \circ
  \tx{Inst}_C(\mathcal{Q})
  $$
  applying the properties we have for systems.

  Then, we can apply interchange to write this as:
  $$
  \begin{pmatrix}
    S_1 \circ S_2 \cr
    \otimes\cr
    1(O)
  \end{pmatrix}
  \circ
  \tx{Inst}_C(\mathcal{Q})
  $$
  which concludes our proof, since $S_1 \circ S_2$ will be a valid simulator.

  $\blacksquare$
\end{theorem}

\begin{theorem}[Malicious Completeness]
  \label{thm:mal_complete}
  Let $\mathcal{P}$ and $\mathcal{Q}$ closed protocols with the same shape.
  Given any class of corruptions $\mathcal{C}$, let $\mathcal{C}'$ be a related class, containing
  models in $\mathcal{C}$ with some
  malicious corruptions replaced with semi-honest corruptions.
  We then have:
  \begin{enumerate}
    \item $\mathcal{P} =_{C} \mathcal{Q} \implies \mathcal{P} =_{C'} \mathcal{Q}$,
    \item $\mathcal{P} \overset{\epsilon}{\approx}_{C} \mathcal{Q} \implies \mathcal{P} \overset{\epsilon}{\approx}_{C'} \mathcal{Q}$,
  \end{enumerate}

  \txbf{Proof:} Lemma~\label{thm:simulatingcorruption} (simulating corruptions) is the crux of our proof.
  It implies that there existts a system $S_{\tx{SH}}$ such that
  $$
  \tx{Corrupt}_{\tx{SH}}(P) = S_{\tx{SH}} \circ \tx{Corrupt}_M(P)
  $$

  As a consequence, for any $C' \in \mathcal{C}'$ and the $C \in \mathcal{C}$ it's related to,
  there exists a \emph{simulator} $S_{\tx{SH}}$ such that:
  $$
  \tx{Inst}_{C'}(\mathcal{P}) =
  \begin{pmatrix}
    S_{\tx{SH}}\cr
    \otimes\cr
    1(O)
  \end{pmatrix}
  \circ \tx{Inst}_{C}(\mathcal{P})
  $$
  which simulates all of the semi-honest corruptions in $C'$ from the malicious ones in $C$.

  This immediately implies parts 1 and 2, by the fact that $\circ$ for systems
  respects equality and indistinguishability.

  $\blacksquare$
\end{theorem}

\begin{theorem}[Vertical Composition Theorem]
  \label{thm:vertical_composition_theorem}
  For any protocol $\mathcal{P}$ and game $G$, such that $\mathcal{P} \circ G$
  is well defined and closed, and for any corruption class $\mathcal{C}$, we have:
  \begin{enumerate}
    \item $G = G' \implies \mathcal{P} \circ G =_{\mathcal{C}} \mathcal{P} \circ G'$
    \item $G \overset{\epsilon}{\approx} G' \implies \mathcal{P} \circ G \overset{\epsilon}{\approx}_{\mathcal{C}} \mathcal{P} \circ G'$
  \end{enumerate}
  
\txbf{Proof:} We start by noting that $\tx{Inst}_C(\mathcal{P} \circ G) = A \circ F \circ G$,
for some system $A$.
Part 1 follows immediately from this, since $\circ$ is equality respecting.

Part 2 follows by applying Lemma~\ref{thm:systemreduction},
which entails that for any system $S$, we have $S \circ G \overset{\epsilon}{\approx} S \circ G'$.

$\blacksquare$
\end{theorem}

\begin{theorem}[Concurrent Composition Theorem]
  Let $\mathcal{P}, \mathcal{Q}$ be protocols, with $\mathcal{P} \otimes \mathcal{Q}$
  well defined and closed. For any compatible corruption classes $\mathcal{C}, \mathcal{C}'$
  it holds that:
  \begin{enumerate}
    \item $\mathcal{Q} =_{\mathcal{C}} \mathcal{Q}' \implies \mathcal{P} \otimes \mathcal{Q} =_{\mathcal{C}'} \mathcal{P} \otimes \mathcal{Q}'$
    \item $\mathcal{Q} \overset{\epsilon}{\approx}_{\mathcal{C}} \mathcal{Q}' \implies \mathcal{P} \otimes \mathcal{Q} \overset{\epsilon}{\approx}_{\mathcal{C}'} \mathcal{P} \otimes \mathcal{Q}'$
    \item $\mathcal{Q} \overset{\epsilon}{\leadsto}_{\mathcal{C}} \mathcal{Q}' \implies \mathcal{P} \otimes \mathcal{Q} \overset{\epsilon}{\leadsto}_{\mathcal{C}'} \mathcal{P} \otimes \mathcal{Q}'$
  \end{enumerate}

  \txbf{Proof:} Theorem~\ref{thm:concurrent_breakdown} (concurrent breakdown)
  will be essential to our proof.
  This implies that $\forall C \in \mathcal{C}$, then for any compatible $C' \in \mathcal{C}'$
  we have:
  $$
  \tx{Inst}_{C'}(\mathcal{P} \otimes \mathcal{Q}) = \tx{Inst}_{C'}(\mathcal{P}) \otimes \tx{Inst}_{C}(\mathcal{Q})
  $$

\txbf{1.}
Since $\mathcal{Q} =_{\mathcal{C}} \mathcal{Q}'$, we have $\forall C \in \mathcal{C}.\ \tx{Inst}_C(\mathcal{Q}) = \tx{Inst}_C(\mathcal{Q}')$.
Now, consider any $C' \in \mathcal{C}'$.
By our assumption that $\mathcal{C}'$ is compatible with $\mathcal{C}$,
there exists a $C \in \mathcal{C}$ that $C'$ is compatible with.
Using concurrent breakdown, we then have:
$$
\tx{Inst}_{C'}(\mathcal{P} \otimes \mathcal{Q}) =
\tx{Inst}_{C'}(\mathcal{P}) \otimes \tx{Inst}_C(\mathcal{Q})
$$
Then, since $\mathcal{Q} =_{\mathcal{C}} \mathcal{Q}'$, we have:
$$
\tx{Inst}_{C'}(\mathcal{P}) \otimes \tx{Inst}_C(\mathcal{Q}) =
\tx{Inst}_{C'}(\mathcal{P}) \otimes \tx{Inst}_C(\mathcal{Q}') =
\tx{Inst}_{C'}(\mathcal{P} \otimes \mathcal{Q}')
$$
concluding our proof.

\txbf{2.}
The proof here is similar to part 1.
For any $C' \in \mathcal{C}'$, there exists a compatible $C \in \mathcal{C}$,
and then we get:
$$
\tx{Inst}_{C'}(\mathcal{P} \otimes \mathcal{Q}) =
\tx{Inst}_{C'}(\mathcal{P}) \otimes \tx{Inst}_C(\mathcal{Q})
$$
Since $\mathcal{Q} \overset{\epsilon}{\approx}_{\mathcal{C}} \mathcal{Q}'$,
we have:
$$
\tx{Inst}_{C'}(\mathcal{P}) \otimes \tx{Inst}_C(\mathcal{Q})
\overset{\epsilon}{\approx} 
\tx{Inst}_{C'}(\mathcal{P}) \otimes \tx{Inst}_C(\mathcal{Q}')
$$
since $\otimes$ for systems respects this operation.
We can then conclude with
$$
\tx{Inst}_{C'}(\mathcal{P}) \otimes \tx{Inst}_C(\mathcal{Q}') =
\tx{Inst}_{C'}(\mathcal{P} \otimes \mathcal{Q}')
$$

\txbf{3.} Once more, for any $C' \in \mathcal{C}'$, there exists a compatible
$C \in \mathcal{C}$ giving us:
$$
\tx{Inst}_{C'}(\mathcal{P} \otimes \mathcal{Q}) =
\tx{Inst}_{C'}(\mathcal{P}) \otimes \tx{Inst}_C(\mathcal{Q})
$$
We then apply our assumption that $\mathcal{Q} \overset{\epsilon}{\leadsto}_{\mathcal{C}} \mathcal{Q}'$
to get:
$$
\tx{Inst}_{C'}(\mathcal{P}) \otimes \tx{Inst}_C(\mathcal{Q})
\overset{\epsilon}{\approx}
\tx{Inst}_{C'}(\mathcal{P}) \otimes ((S \otimes 1(\ldots)) \circ \tx{Inst}_C(\mathcal{Q}'))
$$
Next, we apply interchange to get:
$$
\begin{matrix}
1(\tx{Out}(\tx{Inst}_{C'}(\mathcal{P}))) \circ \tx{Inst}_{C'}(\mathcal{P})\cr
\otimes\cr
((S \otimes 1(\ldots)) \circ \tx{Inst}_C(\mathcal{Q}'))
\end{matrix}
=
\begin{pmatrix}
1(\tx{Out}(\tx{Inst}_{C'}(\mathcal{P})))\cr
\otimes\cr
S\cr
\otimes\cr
1(\tx{Out}(\tx{Inst}_C(\mathcal{Q})) / \tx{Out}(S))
\end{pmatrix}
\circ
\begin{pmatrix}
  \tx{Inst}_{C'}(\mathcal{P})\cr
  \otimes\cr
  \tx{Inst}_{C}(\mathcal{Q}')
\end{pmatrix}
$$
Applying concurrent breakdown in reverse, we get that the right hand
side is $\tx{Inst}_{C'}(\mathcal{P} \otimes \mathcal{Q})$,
and that the left hand side is the simulator showing
that $\mathcal{P} \otimes \mathcal{Q} \overset{\epsilon}{\leadsto}_{\mathcal{C}'} \mathcal{P} \otimes \mathcal{Q}'$.
The left hand side is a valid simulator because
$\tx{Out}(\tx{Inst}_C(\mathcal{Q})) = \tx{Out}(\tx{Inst}_{C'}(\mathcal{Q}))$,
and all of the honest parts of $\mathcal{P}$ are left untouched,
since all of it is.

$\blacksquare$
\end{theorem}

\begin{theorem}[Horizontal Composition Theorem]
  \label{thm:horizontal_composition_theorem}
  For any protocols $\mathcal{P}, \mathcal{Q}$ with $\mathcal{P} \lhd \mathcal{Q}$
  well defined and closed, and for any compatible corruption classes $\mathcal{C}, \mathcal{C'}$, we have:
  \begin{enumerate}
    \item $\mathcal{Q} =_{\mathcal{C}} \mathcal{Q}' \implies \mathcal{P} \lhd \mathcal{Q} =_{\mathcal{C}'} \mathcal{P} \lhd \mathcal{Q}'$
    \item $\mathcal{Q} \overset{\epsilon}{\approx}_{\mathcal{C}} \mathcal{Q}' \implies \mathcal{P} \lhd \mathcal{Q} \overset{\epsilon}{\approx}_{\mathcal{C}'} \mathcal{P} \lhd \mathcal{Q}'$
  \end{enumerate}

  Furthermore, if $\mathcal{C}'$ is \emph{strictly} compatible with $\mathcal{C}$,
  we have:
  \begin{enumerate}
    \setcounter{enumi}{2}
    \item $\mathcal{Q} \overset{\epsilon}{\leadsto}_{\mathcal{C}} \mathcal{Q}' \implies \mathcal{P} \lhd \mathcal{Q} \overset{\epsilon}{\leadsto}_{\mathcal{C}'} \mathcal{P} \lhd \mathcal{Q}'$
  \end{enumerate}

  \txbf{Proof:} As one might expect,
  Theorem~\ref{thm:horizontal_breakdown}(horizontal breakdown)
  will be critical to proving each of these statements.

  One crude summary of the theorem, in the case
  that the protocols are closed, is that given compatible
  corruption models $C, C'$, there's a system $\tx{Stuff}$ such that
  $$
  \tx{Inst}_{C'}(\mathcal{P} \lhd \mathcal{Q}) = \tx{Stuff} \circ \tx{Inst}_C(\mathcal{Q})
  $$
  This summary suffices to prove a couple statements already.

  \txbf{1.} By assumption, for any $C' \in \mathcal{C}'$,
  there exists a compatible $C \in \mathcal{C}$.
  In this case, we have:
  $$
  \tx{Inst}_{C'}(\mathcal{P} \lhd \mathcal{Q}) = \tx{Sutff} \circ \tx{Inst}_C(\mathcal{Q})
  $$
  If we then apply $\mathcal{Q} =_{\mathcal{C}} \mathcal{Q}'$,
  we get:
  $$
  \tx{Stuff} \circ \tx{Inst}_C(\mathcal{Q}) = \tx{Stuff} \circ \tx{Inst}_C(\mathcal{Q}')
  $$
  and then, applying breakdown in reverse, we end up with $\tx{Inst}_{C'}(\mathcal{P} \lhd \mathcal{Q}')$,
  completing our proof.

  \txbf{2.} We apply the same reasoning, with the difference that:
  $$
  \tx{Stuff} \circ \tx{Inst}_C(\mathcal{Q}) \overset{\epsilon}{\approx} \tx{Stuff} \circ \tx{Inst}_C(\mathcal{Q}')
  $$
  rather than being strictly equal.

  \txbf{3.} At this point our crude summary of the breakdown theorem is not
  sufficient anymore.
  We start with the same reasoning.
  For any $C' \in \mathcal{C}'$, there exists a \emph{strictly}
  compatible $C \in \mathcal{C}$, and we have:
  $$
  \tx{Inst}_{C'}(\mathcal{P} \lhd \mathcal{Q}) = \tx{Stuff} \circ \tx{Inst}_C(\mathcal{Q})
  $$
  then, we apply our assumption that $\mathcal{Q} \overset{\epsilon}{\leadsto}_{\mathcal{C}} \mathcal{Q}'$,
  giving us:
  $$
  \tx{Stuff} \circ \tx{Inst}_C(\mathcal{Q}) \overset{\epsilon}{\approx} \tx{Stuff} \circ (S \otimes 1(\ldots)) \circ \tx{Inst}_C(\mathcal{Q})
  $$
  Our strategy will be to rearrange the right hand side to get
  $$
  {(S' \otimes 1(\ldots)) \circ \tx{Inst}_{C'}(\mathcal{P} \lhd \mathcal{Q}')}
  $$
  We start by unrolling $\tx{Stuff}$, using strict compatability, to get:
  $$
  1(O)\circ
  \begin{pmatrix}
    {\displaystyle \bigast}_{i \in [\mathcal{P}.n]} \tx{Routed}(\tx{Corrupt}'_{C'}(\mathcal{P}.P_i))
    \cr
    *\cr
    \mathcal{R}_{\mathcal{P}}\cr
    \otimes\cr
    1(\tx{Leakage}, L_{\mathcal{Q}'})
  \end{pmatrix}
  \circ
  \begin{pmatrix}
    \mathcal{P}.F\cr
    \otimes\cr
    1(\tx{Out}(\mathcal{R}_q))\cr
    \otimes\cr
    1(\mathcal{Q}'.\tx{Leakage})\cr
    \otimes\cr
    \bigotimes_{i \in [\mathcal{Q}'.n]} 1_i\cr
  \end{pmatrix}
  \circ
  \begin{pmatrix}
    S\cr
    \otimes\cr
    1(O_{\bar{S}})
  \end{pmatrix}
  \circ
  \tx{Inst}_C(\mathcal{Q}')
  $$
  with $O_{\bar{S}} := \tx{Out}(\tx{Inst}_C(\mathcal{Q}')) / \tx{Out}(S)$,
  and with each $1_i := 1(\tx{Out}(\tx{Inst}_C(\mathcal{Q}'.P_i)))$.
  we can apply interchange a few times to get:
  $$
  \footnotesize
  1(O) \circ
  \begin{pmatrix}
    \begin{pmatrix}
    {\displaystyle \bigast_{C'_i \neq \tx{H}}}
    \begin{pmatrix}
      \tx{Routed}(\tx{Corrupt}'_{C'}(\mathcal{P}.P_i))\cr
      \otimes\cr
      1(L_i)
    \end{pmatrix}\cr
    \otimes\cr
    1(\tx{Leakage})
    \end{pmatrix}
    \circ
    \begin{pmatrix}
      S\cr
      \otimes\cr
      1(O_S)
    \end{pmatrix}
    \circ
    \begin{pmatrix}
    {\displaystyle \bigast_{C_i \neq \tx{H}}}
    \tx{Routed}(\tx{Corrupt}_C(\mathcal{Q}'.P_i))
    \cr
    \otimes\cr
    1(\tx{Out}(\mathcal{P}.F), \tx{Out}(\mathcal{Q}.F))
    \end{pmatrix}
    \cr
    *\cr
    {\displaystyle \bigast_{C'_i = \tx{H}}}
    \tx{Routed}(\tx{Corrupt}_{C'}((\mathcal{P} \lhd \mathcal{Q}').P_i))\cr
    *\cr
    \mathcal{R}_{\mathcal{P}} \circ \mathcal{R}_{\mathcal{Q}'}\cr
  \end{pmatrix}
  \circ
  \begin{pmatrix}
    \mathcal{P}.F\cr
    \otimes\cr
    \mathcal{Q}'.F
  \end{pmatrix}
  $$
  with $O_S := O_{\bar{S}} \cup \tx{Out}(\mathcal{P}.F)$ and $L_i$ as per the horizontal breakdown theorem.
  The only functions that $S$ masks are the leakage, the malicious corruption
  functions, and the logs from semi-honest corruption.
  Semi-honest corruption does not use any outputs of $S$,
  instead relying on the $\mathcal{Q}'.P_i$, accessible via $1(O_S)$.
  In the case of malicious corruption, since $\tx{Corrupt}'_{C'}(\mathcal{P}.P_i)$
  omits the $\tx{Call}_{F_i}$ functions, the system also has no dependencies
  on the output of $S$.
  Since none of these corrupted players depend on $S$,
  we can slide it forward, using interchange, to get:
  $$
  \footnotesize
  1(O) \circ
  \begin{pmatrix}
    \begin{pmatrix}
      S\cr
      \otimes\cr
      1(\ldots)
    \end{pmatrix}
    \circ
    \begin{pmatrix}
    {\displaystyle \bigast_{C'_i \neq \tx{H}}}
    \begin{pmatrix}
      \tx{Routed}(\tx{Corrupt}'_{C'}(\mathcal{P}.P_i))\cr
      \otimes\cr
      1(L_i)
    \end{pmatrix}\cr
    \otimes\cr
    1(\tx{Leakage})
    \end{pmatrix}
    \circ
    \begin{pmatrix}
    {\displaystyle \bigast_{C_i \neq \tx{H}}}
    \tx{Routed}(\tx{Corrupt}_C(\mathcal{Q}'.P_i))
    \cr
    \otimes\cr
    1(\tx{Out}(\mathcal{P}.F), \tx{Out}(\mathcal{Q}.F))
    \end{pmatrix}
    \cr
    *\cr
    {\displaystyle \bigast_{C'_i = \tx{H}}}
    \tx{Routed}(\tx{Corrupt}_{C'}((\mathcal{P} \lhd \mathcal{Q}').P_i))\cr
    *\cr
    \mathcal{R}_{\mathcal{P}} \circ \mathcal{R}_{\mathcal{Q}'}\cr
  \end{pmatrix}
  \circ
  \begin{pmatrix}
    \mathcal{P}.F\cr
    \otimes\cr
    \mathcal{Q}'.F
  \end{pmatrix}
  $$
  which becomes:
  $$
  \begin{pmatrix}
    S\cr
    \otimes\cr
    1(\tx{Out}(\tx{Inst}_{C'}(\mathcal{P} \lhd \mathcal{Q}')) / \tx{Out}(S))
  \end{pmatrix}
  \circ
  \tx{Inst}_{C'}(\mathcal{P} \lhd \mathcal{Q}')
  $$

  From this chain of equalities we conclude that
  $\mathcal{P} \lhd \mathcal{Q}' \overset{\epsilon}{\leadsto} \mathcal{P} \lhd \mathcal{Q}'$

  $\blacksquare$
\end{theorem}

\subsection{Global Functionalities}

\begin{definition}[Relatively Closed Protocols]
  A protocol $\mathcal{P}$ is \emph{closed relative to} a game $G$
  if:
  \begin{itemize}
    \item $\tx{In}(\mathcal{P}) = \emptyset$
    \item $\tx{IdealIn}(\mathcal{P}) \subseteq \tx{Out}(G)$
  \end{itemize} 

  $\square$
\end{definition}
\begin{definition}[Relative Instantiation]
  Given a closed protocol $\mathcal{P}$ relative to $G$, we can define,
  for any corruption model $C$,
  the relative instantiation:
  $$
  \tx{Inst}_{C}^G(\mathcal{P}) :=
  \begin{pmatrix}
    \tx{Inst}_C(\mathcal{P})\cr
    \otimes\cr
    1(\tx{Out}(G))
  \end{pmatrix}
  \circ G
  $$

  We can also extend this to the case of simulated instantiation,
  defining, for any simulator $S$:
  $$
  \tx{SimInst}_{S, C}^G(\mathcal{P}) :=
  \begin{pmatrix}
    \tx{SimInst}_{S, C}(\mathcal{P})\cr
    \otimes\cr
    1(\tx{Out}(G))
  \end{pmatrix}
  \circ G
  $$

  $\square$
\end{definition}

\begin{definition}[Relative Notions of Equality]
  Given closed protocols $\mathcal{P}, \mathcal{Q}$ relative to $G$,
  with the same shape, and a corruption class $\mathcal{C}$
  for these protocols, we define:
  \begin{itemize}
    \item $\mathcal{P} =^G_{\mathcal{C}} \mathcal{Q} \iff \forall C \in \mathcal{C}.\ \tx{Inst}^G_C(\mathcal{P}) = \tx{Inst}^G_C(\mathcal{Q})$
    \item $\mathcal{P} \overset{\epsilon}{\approx}^G_{\mathcal{C}} \mathcal{Q}\iff \forall C \in \mathcal{C}.\ \tx{Inst}^G_C(\mathcal{P}) \overset{\epsilon}{\approx} \tx{Inst}^G_C(\mathcal{Q})$
    \item $\mathcal{P} \overset{\epsilon}{\leadsto}^G_{\mathcal{C}} \mathcal{Q}\iff \forall C \in \mathcal{C}.\ \exists S.\ \tx{Inst}^G_C(\mathcal{P}) \overset{\epsilon}{\approx} \tx{SimInst}^G_{S,C}(\mathcal{Q})$
  \end{itemize}

  $\square$
\end{definition}

\begin{theorem}[Relative Equality Hierarchy]
  For any corruption class $\mathcal{C}$ and game $G$, we have:
\begin{enumerate}
\item $\mathcal{P} =^G_{\mathcal{C}} \mathcal{Q} \implies \mathcal{P} \overset{0}{\approx}^G_\mathcal{C} \mathcal{Q}$.
\item $\mathcal{P} \overset{\epsilon}{\approx}^G_{\mathcal{C}} \mathcal{Q} \implies \mathcal{P} \overset{\epsilon}{\leadsto}^G_\mathcal{C} \mathcal{Q}$.
\end{enumerate}
\txbf{Proof:}

\txbf{1.} This follows from the fact that $A = B \implies A \overset{0}{\approx} B$
for any systems $A, B$.

\txbf{2.} In the proof of Theorem~\ref{thm:equality_hierarchy},
we used the existence of a simulator $S$ such that $\tx{SimInst}_{S, C}(\mathcal{P}) = \tx{Inst}_C(\mathcal{P})$.
This simulator will also satisfy $\tx{SimInst}^G_{S, C}(\mathcal{P}) = \tx{Inst}^G_C(\mathcal{P})$,
and can thus be used directly to prove this relation.

$\blacksquare$
\end{theorem}

\begin{theorem}[Transitivity of Relative Equality]
  \label{thm:trans_relative_equality}
  For any protocols $\mathcal{L}$, $\mathcal{P}$, $\mathcal{Q}$
  closed relative to a game $G$, and for any corruption class, we have:
  \begin{enumerate}
    \item $\mathcal{L} =_{\mathcal{C}}^G \mathcal{P}, \mathcal{P} =_{\mathcal{C}}^G \mathcal{Q} \implies \mathcal{L} =_{\mathcal{C}}^G \mathcal{Q}$,
    \item $\mathcal{L} \overset{\epsilon_1}{\approx}_{\mathcal{C}}^G \mathcal{P}, \mathcal{P} \overset{\epsilon_2}{\approx}_{\mathcal{C}}^G \mathcal{Q} \implies \mathcal{L} \overset{\epsilon_1 + \epsilon_2}{\approx}_{\mathcal{C}}^G \mathcal{Q}$,
    \item $\mathcal{L} \overset{\epsilon_1}{\leadsto}_{\mathcal{C}}^G \mathcal{P}, \mathcal{P} \overset{\epsilon_2}{\leadsto}_{\mathcal{C}}^G \mathcal{Q} \implies \mathcal{L} \overset{\epsilon_1 + \epsilon_2}{\leadsto}_{\mathcal{C}}^G \mathcal{Q}$.
  \end{enumerate}

  \txbf{Proof:} Once again, the first two parts follow directly from Lemma~\ref{thm:system_trans},
  by considering the systems $\tx{Inst}^G_C(\mathcal{L})$, $\tx{Inst}^G_C(\mathcal{P})$,
  $\tx{Inst}^G_C(\mathcal{Q})$ for any $C \in \mathcal{C}$.

  For part 3, given any $C \in \mathcal{C}$, there exists $S_1, S_2$ such that:
  \begin{itemize}
    \item $\tx{Inst}^G_C(\mathcal{L}) \overset{\epsilon_1}{\approx} \tx{SimInst}^G_{S_1, C}(\mathcal{P})$,
    \item $\tx{Inst}^G_C(\mathcal{P}) \overset{\epsilon_2}{\approx} \tx{SimInst}^G_{S_2, C}(\mathcal{Q})$.
  \end{itemize}
  Next, observe that for any protocol $\mathcal{P}$, we can write:
  $$
  \tx{SimInst}^G_C =
  \begin{pmatrix}
    S\cr
    \otimes\cr
    1(O)
  \end{pmatrix}
  \circ
  \tx{Inst}^G_C(\mathcal{P})
  $$
  where $O = \tx{Out}(\tx{Inst}_C(\mathcal{P})) / \tx{Out}(S) \cup \tx{Out}(G)$.

  We then apply transitivity for systems and interchange get:
  $$
  \tx{Inst}^G_C(\mathcal{L}) \overset{\epsilon_1 + \epsilon_2}{\approx}
  \begin{pmatrix}
    S_1 \circ S_2\cr
    \otimes\cr
    1(O)
  \end{pmatrix}
  \circ
  \tx{Inst}^G_C(\mathcal{Q})
  $$
  And the left side is simply $\tx{SimInst}^G_{(S_1 \circ S_2), C}(\mathcal{Q})$,
  concluding our proof.

  $\blacksquare$
\end{theorem}

\begin{theorem}[Global Malicious Completeness]
  Let $\mathcal{P}$ and $\mathcal{Q}$ closed protocols relative to $G$ with the same shape.
  Given any class of corruptions $\mathcal{C}$, let $\mathcal{C}'$ be a related class, containing
  models in $\mathcal{C}$ with some
  malicious corruptions replaced with semi-honest corruptions.
  We then have:
  \begin{enumerate}
    \item $\mathcal{P} =^G_{C} \mathcal{Q} \implies \mathcal{P} =^G_{C'} \mathcal{Q}$,
    \item $\mathcal{P} \overset{\epsilon}{\approx}^G_{C} \mathcal{Q} \implies \mathcal{P} \overset{\epsilon}{\approx}^G_{C'} \mathcal{Q}$,
  \end{enumerate}

  \txbf{Proof:} We proceed similarly to Theorem~\ref{thm:mal_complete} (malicious completeness).
  In that theorem, the key observation was that for any $C' \in \mathcal{C}'$
  and the related $C \in \mathcal{C}$, it holds that:
  $$
  \tx{Inst}_{C'}(\mathcal{P}) = \tx{SimInst}_{S_{\tx{SH}}, C}(\mathcal{P})
  $$
  (this observation also doesn't depend on $\mathcal{P}$ being fully closed,
  allowing us to use it here).

  Now, this clearly implies that:
  $$
  \tx{Inst}_{C'}^G(\mathcal{P}) = \tx{SimInst}^G_{S_{\tx{SH}}, C}(\mathcal{P})
  $$
  And then, using our observation from Theorem~\ref{thm:trans_relative_equality},
  we can write this as:
  $$
  \tx{Inst}_{C'}^G(\mathcal{P}) =
  \begin{pmatrix}
    S_{\tx{SH}}\cr
    \otimes\cr
    1(O)
  \end{pmatrix}
  \circ
  \tx{Inst}^G_C(\mathcal{P})
  $$
  where $O = \tx{Out}(\tx{Inst}_C(\mathcal{P})) / \tx{Out}(S) \cup \tx{Out}(G)$.

  This immediately implies parts 1 and 2.

  $\blacksquare$
\end{theorem}

\begin{theorem}[Global Vertical Composition Theorem]
  For any protocol $\mathcal{P}$ and game $F$, such that $\mathcal{P} \circ F$
  is well defined and closed relative to $G$, and for any corruption class $\mathcal{C}$, we have:
  \begin{enumerate}
    \item $F = F' \implies \mathcal{P} \circ F =^G_{\mathcal{C}} \mathcal{P} \circ F'$
    \item $F \overset{\epsilon}{\approx} F' \implies \mathcal{P} \circ F \overset{\epsilon}{\approx}^G_{\mathcal{C}} \mathcal{P} \circ F'$
  \end{enumerate}

  \txbf{Proof:} The proof of Theorem~\ref{thm:vertical_composition_theorem} will be the basis
  for what we do here.
  Using it, we can write:
  $$
  \tx{Inst}^G_C(\mathcal{P} \circ F) =
  \begin{pmatrix}
    A \circ F\cr
    \otimes\cr
    1(\tx{Out}(G))
  \end{pmatrix}
  \circ G
  $$
  for some system $A$.
  At this point, the theorem immediately holds, since $\circ$ and $\otimes$ (for systems)
  respect both $=$ and $\approx$.

  $\blacksquare$
\end{theorem}

\begin{theorem}[Global Concurrent Composition Theorem]
  Let $\mathcal{P}, \mathcal{Q}$ be closed protocols relative to $G$, with $\mathcal{P} \otimes \mathcal{Q}$
  well defined. For any compatible corruption classes $\mathcal{C}, \mathcal{C}'$
  it holds that:
  \begin{enumerate}
    \item $\mathcal{Q} =^G_{\mathcal{C}} \mathcal{Q}' \implies \mathcal{P} \otimes \mathcal{Q} =^G_{\mathcal{C}'} \mathcal{P} \otimes \mathcal{Q}'$
    \item $\mathcal{Q} \overset{\epsilon}{\approx}^G_{\mathcal{C}} \mathcal{Q}' \implies \mathcal{P} \otimes \mathcal{Q} \overset{\epsilon}{\approx}^G_{\mathcal{C}'} \mathcal{P} \otimes \mathcal{Q}'$
    \item $\mathcal{Q} \overset{\epsilon}{\leadsto}^G_{\mathcal{C}} \mathcal{Q}' \implies \mathcal{P} \otimes \mathcal{Q} \overset{\epsilon}{\leadsto}^G_{\mathcal{C}'} \mathcal{P} \otimes \mathcal{Q}'$
  \end{enumerate}

  \txbf{Proof:} We start by using Theorem~\ref{thm:concurrent_breakdown}, giving us:
  $$
  \tx{Inst}^G_{C'}(\mathcal{P} \otimes \mathcal{Q})
  =
  \begin{pmatrix}
    \tx{Inst}_{C'}(\mathcal{P})\cr
    \otimes\cr
    \tx{Inst}_C(\mathcal{Q})\cr
    \otimes\cr
    1(\tx{Out}(G))
  \end{pmatrix}
  \circ G
  =
  \begin{pmatrix}
    \tx{Inst}_{C'}(\mathcal{P})\cr
    \otimes\cr
    1(\tx{Out}(\tx{Inst}_C(\mathcal{Q})))\cr
    \otimes\cr
    1(\tx{Out}(G))
  \end{pmatrix}
  \circ
  \tx{Inst}^G_C(\mathcal{Q})
  $$
  We can then immediately derive parts 1 and 2.

  For part 3, we apply the hypothesis to the last part of the above relation, to get:
  $$
  \tx{Inst}^G_{C'} \overset{\epsilon}{\approx}
  \begin{pmatrix}
    \tx{Inst}_{C'}(\mathcal{P})\cr
    \otimes\cr
    1(\tx{Out}(\tx{Inst}_C(\mathcal{Q})))\cr
    \otimes\cr
    1(\tx{Out}(G))
  \end{pmatrix}
  \circ
  \tx{SimInst}^G_{S, C}(\mathcal{Q})
  $$
  Then, we unroll $\tx{SimInst}^G_{S, C}(\mathcal{Q})$, to get:
  $$
  \begin{pmatrix}
    \tx{Inst}_{C'}(\mathcal{P})\cr
    \otimes\cr
    1(\tx{Out}(\tx{Inst}_C(\mathcal{Q})))\cr
    \otimes\cr
    1(\tx{Out}(G))
  \end{pmatrix}
  \circ
  \begin{pmatrix}
    \begin{pmatrix}
      S\cr
      \otimes\cr
      1(\ldots)\cr
    \end{pmatrix}
    \circ \tx{Inst}_C(\mathcal{Q})\cr
    \otimes\cr
    1(\tx{Out}(G))
  \end{pmatrix}
  \circ G
  $$
  Then, we apply interchange to get:
  $$
  \begin{pmatrix}
    \begin{pmatrix}
      1(\ldots)\cr
      \otimes\cr
      S\cr
      \otimes\cr
      1(\ldots)\cr
    \end{pmatrix}
    \circ
    \begin{pmatrix}
    \tx{Inst}_{C'}(\mathcal{P})\cr
    \otimes\cr
    \tx{Inst}_C(\mathcal{Q})\cr
    \end{pmatrix}\cr
    \otimes\cr
    1(\tx{Out}(G))
  \end{pmatrix}
  \circ G
  $$
  But this is just $\tx{SimInst}^G_{S', C'}(\mathcal{P} \otimes \mathcal{Q})$,
  for some simulator $S'$,
  applying concurrent breakdown in reverse.

  $\blacksquare$
\end{theorem}

\begin{theorem}[Global Horizontal Composition Theorem]
  For any protocols $\mathcal{P}, \mathcal{Q}$ closed relative to $G$, with $\mathcal{P} \lhd \mathcal{Q}$
  well defined, and for any compatible corruption classes $\mathcal{C}, \mathcal{C'}$, we have:
  \begin{enumerate}
    \item $\mathcal{Q} =^G_{\mathcal{C}} \mathcal{Q}' \implies \mathcal{P} \lhd \mathcal{Q} =^G_{\mathcal{C}'} \mathcal{P} \lhd \mathcal{Q}'$
    \item $\mathcal{Q} \overset{\epsilon}{\approx}^G_{\mathcal{C}} \mathcal{Q}' \implies \mathcal{P} \lhd \mathcal{Q} \overset{\epsilon}{\approx}^G_{\mathcal{C}'} \mathcal{P} \lhd \mathcal{Q}'$
  \end{enumerate}

  Furthermore, if $\mathcal{C}'$ is \emph{strictly} compatible with $\mathcal{C}$,
  we have:
  \begin{enumerate}
    \setcounter{enumi}{2}
    \item $\mathcal{Q} \overset{\epsilon}{\leadsto}^G_{\mathcal{C}} \mathcal{Q}' \implies \mathcal{P} \lhd \mathcal{Q} \overset{\epsilon}{\leadsto}^G_{\mathcal{C}'} \mathcal{P} \lhd \mathcal{Q}'$
  \end{enumerate}

  \txbf{Proof:} This proof is similar to that of Theorem~\ref{thm:horizontal_composition_theorem}.
  By compatability, for any $C' \in \mathcal{C}'$, we have a compatible $C \in \mathcal{C}$.

  A crude summary of the horizontal breakdown theorem is that:
  $$
  \tx{Inst}_{C'}(\mathcal{P} \lhd \mathcal{Q})
  = \tx{Stuff} \circ \begin{pmatrix}
    \tx{Inst}_C(\mathcal{Q})\cr
    \otimes\cr
    1(\tx{In}(\mathcal{P}.F))
  \end{pmatrix}
  $$
  Using the fact that being closed relative to $G$ means $\tx{In}(\mathcal{P}.F) \subseteq \tx{Out}(G)$,
  we get:
  $$
  \tx{Inst}^G_{C'}(\mathcal{P} \lhd \mathcal{Q}) =
  \begin{pmatrix}
    \tx{Stuff}\cr
    \otimes\cr
    1(\tx{Out}(G))
  \end{pmatrix}
  \circ \tx{Inst}^G_C(\mathcal{Q})
  $$
  Part 1 and 2 both follow immediately from this decomposition.

  For part 3, we dig a bit deeper into the proof of Theorem~\ref{thm:horizontal_composition_theorem}.
  In that proof, it was actually shown that:
  $$
  \tx{Stuff} \circ \tx{SimInst}_{S, C}(\mathcal{Q}') = \tx{SimInst}_{S', C'}(\mathcal{P} \lhd \mathcal{Q}')
  $$
  for some appropriate simulator $S'$.

  We can start to apply this, first by using our hypothesis:
  $$
  \tx{Inst}^G_{C'}(\mathcal{P} \lhd \mathcal{Q}) =
  \begin{pmatrix}
    \tx{Stuff}\cr
    \otimes\cr
    1(\tx{Out}(G))
  \end{pmatrix}
  \circ \tx{Inst}^G_C(\mathcal{Q})
  \overset{\epsilon}{\approx}
  \begin{pmatrix}
    \tx{Stuff}\cr
    \otimes\cr
    1(\tx{Out}(G))
  \end{pmatrix}
  \circ \tx{SimInst}^G_C(\mathcal{Q}')
  $$
  Next, we unroll the right side, to get:
  $$
  \begin{pmatrix}
    \tx{Stuff}\cr
    \otimes\cr
    1(\tx{Out}(G))
  \end{pmatrix}
  \circ
  \begin{pmatrix}
    \tx{SimInst}_{S, C}(\mathcal{Q}')\cr
    \otimes\cr
    1(\tx{Out}(G))
  \end{pmatrix}
  \circ
  G
  $$
  Then, apply interchange, to get:
  $$
  \begin{pmatrix}
    \tx{Stuff} \circ \tx{SimInst}_{S, C}(\mathcal{Q}')\cr
    \otimes\cr
    1(\tx{Out}(G))
  \end{pmatrix}
  \circ
  G
  $$
  And finally, apply the fact we dug up above, to get:
  $$
  \begin{pmatrix}
    \tx{SimInst}_{S', C'}(\mathcal{P} \lhd \mathcal{Q})\cr
    \otimes\cr
    1(\tx{Out}(G))
  \end{pmatrix}
  \circ
  G
  $$
  which is none other than $\tx{SimInst}^G_{S', C'}(\mathcal{P} \lhd \mathcal{Q})$.

  $\blacksquare$
\end{theorem}

\subsection{Hopping Ideal Functionalities}

\begin{lemma}[Deidealization Lemma]
  Given a closed protocol $\mathcal{P}$ with an ideal functionality $F \otimes G$,
  there exists protocols $\mathcal{P}'$ and $\mathcal{G}$
  such that:
  $$
  \mathcal{P} \equiv \mathcal{P}' \lhd \mathcal{G}
  $$
  and $\mathcal{P}'$ has ideal functionality $F$.


  \txbf{Proof:} The players of $\mathcal{P}'$ are those of $\mathcal{P}$,
  except that each $P_i$'s call to a function $g \in \tx{Out}(G)$ is replaced with
  a renamed function $g_i$.
  $\mathcal{G}$ will have one player for each player in $\mathcal{P}'$.
  Each player $\mathcal{G}.P_i$ exports a function $g_i$ for each input
  $g_i$ of $\mathcal{P}'.P_i$, which immediately calls $g \in \tx{Out}(G)$,
  and returns the result.
  From this definition, it's clear that $\mathcal{P}$ is literally equal
  to $\mathcal{P}' \lhd \mathcal{G}$, as when the players in the latter
  are formed, the calls to the intermediate $g_i$ disappear,
  with each player calling $g \in \tx{Out}(G)$ directly

  $\blacksquare$
\end{lemma}

\begin{lemma}[Embedding Lemma]
  Given a protocol $\mathcal{P}$ closed relative to a game $G$,
  there exists a protocol $\tx{Embed}_G(\mathcal{P})$ such that for any
  corruption model $C$, we have:
  $$
  \tx{Inst}^G_C(\mathcal{P}) = \tx{Inst}_C(\tx{Embed}_G(\mathcal{P}))
  $$

  \txbf{Proof:}
  This one is quite simple. $\tx{Embed}_G(\mathcal{P})$ has the same players
  as $\mathcal{P}$, with the ideal functionality becoming:
  $$
  \begin{pmatrix}
    \mathcal{P}.F\cr
    \otimes\cr
    1(\tx{Out}(G))
  \end{pmatrix}
  \circ G
  $$
  and the leakage being $\mathcal{P}.\tx{Leakage} \cup \tx{Out}(G)$.
  The two instantiations will then clearly be equal under any corruption model.

  $\blacksquare$
\end{lemma}

\subsection{Some Syntactical Conventions}

\section{Examples}

In this section, we provide a couple example proofs in the framework,
to illustrate how it works, and some of the advantages it provides.
The two examples we provide are that of constructing a private
channel from one that leaks all messages sent on it,
and that of sampling an unbiased random value using
the ubiquitous paradigm of ``commit reveal''.

\subsection{Constructing Private Channels}

In this subsection, we consider the problem of constructing a \emph{private}
channel from a \emph{public} channel.
A public channel leaks all messages sent over it to an adversary,
whereas a private channel leaks a minimal amount of information:
in our case, essentially just the length of messages sent over the channel.
This example was also used in \cite{cramer2015secure}.

We'll be constructing a two-party private channel from a public channel
using an encryption scheme, and will also show that this construction is secure,
even if one of the two parties using the channel is corrupted.

Let's start with the ideal functionality representing a public channel,
as Game~\ref{game:pubchan}.

A few clarifications on the notation in this game:
\begin{itemize}
    \item For $i \in \{1, 2\}$, we let $\noti$ denote either $2$ or $1$, respectively.
    \item There are two versions of $\tx{Send}_i$ and $\tx{Recv}_i$, for $i \in \{1, 2\}$.
    \item The $\tx{pop}$ function on queues is asynchronous, meaning that we wait until the queue is not empty
    to remove the oldest element from it.
    \item The queues are public in an \emph{immutable} fashion: they can be read but not modified outside the package.
\end{itemize}

\begin{game}{game:pubchan}{Public Channel Functionality}
\package{$F[\tx{PubChan}]$}{
    &\txbf{view } m_{1\to2}, m_{2\to1} \gets \tx{FifoQueue}.\tx{new}()\cr
    \cr
    &\begin{aligned}
        &\pfn{$\tx{Send}_{i\to \noti}$}{m}\cr
        \pind{1}m_{i\to \noti}.\tx{push}(m)\cr
    \end{aligned}
    \quad
    \begin{aligned}
        &\pfn{$\tx{Recv}_{i\to \noti}$}{}\cr
        &\preturn{\await{m_{i\to \noti}.\tx{pop}()}}
    \end{aligned}
}
\end{game}

The idea behind this functionality is that each party can send messages to,
or receive messages from the other party.
However, at any point, the currently stored messages are readable by
the adversary.
Note that this assignment of which functions
are usable by which entities is not defined by the functionality \emph{itself},
but rather merely suggested by its syntax, and enforced only by how
protocols will eventually use the functionality.

Next, we look at a functionality for \emph{private} channels,
captured by Game~\ref{game:privchan}.

\begin{game}{game:privchan}{Private Channel Functionality}
\package{$F[\tx{PrivChan}]$}{
    &\txbf{view } m_{1\to2}, m_{2\to1} \gets \tx{FifoQueue}.\tx{new}()\cr
    &\txbf{pub } l_{1\to2}, l_{2\to1} \gets \tx{FifoQueue}.\tx{new}()\cr
    \cr
    &\begin{aligned}
        &\pfn{$\tx{Send}_{i\to \noti}$}{m}\cr
        \pind{1}m_{i\to \noti}.\tx{push}(m)\cr
        \pind{1}l_{i\to \noti}.\tx{push}((\txt{push}, |m|))\cr
        \cr
    \end{aligned}
    \quad
    \begin{aligned}
        &\pfn{$\tx{Recv}_{i\to \noti}$}{}\cr
        \pind{1} \set{m}{\await{m_{i\to \noti}.\tx{pop}()}}\cr
        \pind{1}l_{i\to \noti}.\tx{push}(\txt{pop})\cr
        \pind{1} \preturn{m}\cr
    \end{aligned}
}
\end{game}

The crucial difference is the nature of the leakage.
Now, rather than being able to see the current state of either message queue,
including the messages themselves, now the adversary can only
see a historical log for each queue, describing only the \emph{length}
of the messages inserted into the queue.
The reason for having a historical log, rather than just a snapshot
of the lengths of the current messages,
is to make the simulator's job easier
in the eventual proof of security.
For technical reasons, it's simpler to allow the log to be mutated,
so that the simulator can ``remember'' which parts of the log they've
already seen, by popping elements from the queue.

Now, we need to define the protocols.
One protocol will use the private channel to send messages,
and the other will try and implement the same behavior,
but using only the public channel, aided by an encryption scheme.

Let's start with the simpler private channel protocol, which we'll
call $\mathscr{Q}$,
and defined via Protocol~\ref{prot:privchan}

\begin{protocol}{prot:privchan}{Private Channel Protocol}
    $\mathscr{Q}$ is characterized by:
    \begin{itemize}
        \item $\tx{Leakage} := \{l_{1\to 2}, l_{2\to1}\}$,
        \item $F := \tx{PrivChan}$,
        \item And two players defined via the following system (for $i \in \{1, 2\}$):
\package{$P_i$}{
    &\begin{aligned}
        &\pfn{$\tx{Send}_{i}$}{m}\cr
        \pind{1} \tx{Send}_{i \to \noti}(m)\cr
    \end{aligned}
    \begin{aligned}
        &\pfn{$\tx{Recv}_{i}$}{}\cr
        \pind{1} \preturn{\await \tx{Recv}_{\noti \to i}()}\cr
    \end{aligned}
}
    \end{itemize}

\end{protocol}

This protocol basically just provides each player access with their corresponding
functions in the functionality, and leaks the parts of the functionality
that the adversary should have access to, as expected.

Next, we need to define a protocol providing an encrypted channel.
We'll call this one $\mathscr{P}$.
The basic idea is that $\mathscr{P}$ will encrypt messages before sending
them over the public channel.
We'll be using public-key encryption, as defined in Appendix~\ref{app:encryption}.
For the sake of simplicity, we'll be relying on an additional functionality,
$\tx{Keys}$, which will be used to setup each party's key pair, and allow
each party to agree on the other's public key.

This functionality is defined in Game~\ref{game:keys}.
The basic idea is that a key pair is generated for each party,
and that party can see their secret key, along with the public key for the other party.
Furthermore, we allow the adversary to see both public keys.

\begin{game}{game:keys}{Keys Functionality}
\package{Keys}{
&\draw{(\tx{sk}_1, \tx{pk}_1)}{\tx{Gen}()}\cr
&\draw{(\tx{sk}_2, \tx{pk}_2)}{\tx{Gen}()}\cr
\cr
&\pfn{$\tx{Keys}_i$}{}\cr
\pind{1} \preturn{(\tx{sk}_i, \tx{pk}_{\noti})}\cr
\cr
&\pfn{PKs}{}\cr
\pind{1} \preturn{(\tx{pk}_1, \tx{pk}_2)}\cr
}
\end{game}

With this in hand, we can define $\mathscr{P}$ itself, in Protocol~\ref{prot:encchan}.

\begin{protocol}{prot:encchan}{Encrypted Channel Protocol}
    $\mathscr{P}$ is characterized by:
    \begin{itemize}
        \item $\tx{Leakage} := \{m_{1\to 2}, m_{2\to1}, \tx{PKs}\}$,
        \item $F := \tx{Keys} \otimes \tx{PrivChan}$,
        \item and two players defined via the following system (for $i \in \{1, 2\}$):
\package{$P_i$}{
    &(\tx{sk}_i, \tx{pk}_{\noti}) \gets \tx{Keys}_i()\cr
    \cr
    &\begin{aligned}
        &\pfn{$\tx{Send}_{i}$}{m}\cr
        \pind{1} \tx{Send}_{i \to \noti}(\tx{Enc}(\tx{pk}_{\noti}, m))\cr
        \cr
    \end{aligned}
    \begin{aligned}
        &\pfn{$\tx{Recv}_{i}$}{}\cr
        \pind{1} \set{c}{\await \tx{Recv}_{\noti \to i}()}\cr
        \pind{1} \preturn{\tx{Dec}(\tx{sk}_i, c)}\cr
    \end{aligned}
}
    \end{itemize}

\end{protocol}

Each player will encrypt their message for the other player before sending it,
and then decrypt it using their secret key after receiving it.

At this point we can state and prove the crux of this example:

\begin{claim}
Let $\mathscr{C}$ be the class of corruptions where up to 1 of 2 parties
is either maliciously corrupt or semi-honestly corrupt.
Then we have:
$$
\mathscr{P} \overset{2 \cdot \tx{IND}}{\leadsto}_{\mathscr{C}} \mathscr{Q}
$$

\txbf{Proof:} We consider the cases where all the parties are honest
and some of the parties are corrupted separately.
Furthermore, we only need to consider malicious corruption,
since the parties in $\mathscr{Q}$ just directly call functions from
the ideal functionality, and so we can simulate malicious corruption
from semi-honest corruption, and can thus apply part 3
of Theorem~\ref{thm:mal_complete}.

\txbf{Honest Case:} Let $\tx{H}$ be a corruption model where both parties are honest.
We prove that $\mathscr{P} \overset{2 \cdot \tx{IND}}{\leadsto}_{\{\tx{H}\}} \mathscr{Q}$.

The high level idea is that since ciphertexts should be indistinguishable from random
encryptions, the information in the log we get as a simulator for $\mathscr{Q}$
is enough to fake all the ciphertexts the environment expects to see in $\mathscr{P}$.

We start by unrolling $\tx{Inst}_{\tx{H}}(\mathscr{P})$, obtaining:
$$
\tx{Inst}_{\tx{H}}(\mathscr{P}) =
\inlinepackage{$\Gamma^0$}{
    &\txbf{view } c_{1\to2}, c_{2\to1} \gets \tx{FifoQueue}.\tx{new}()\cr
    &(\tx{sk}_i, \tx{pk}_{\noti}) \gets \tx{Keys}_i()\cr
    \cr
    &\pfn{PKs}{}\cr
    \pind{1} \preturn (\tx{pk}_1, \tx{pk}_2)\cr
    &\begin{aligned}
        &\pfn{$\tx{Send}_{i}$}{m}\cr
        \pind{1} \set{c}{\tx{Enc}(\tx{pk}_{\noti}, m)}\cr
        \pind{1} c_{i\to\noti}.\tx{push}(c)\cr
    \end{aligned}
    \begin{aligned}
        &\pfn{$\tx{Recv}_{i}$}{}\cr
        \pind{1} \set{c}{\await \tx{c}_{\noti \to i}.\tx{pop}()}\cr
        \pind{1} \preturn{\tx{Dec}(\tx{sk}_i, c)}\cr
    \end{aligned}
}
\circ \tx{Keys}
$$
Note that we can ignore all parts of the instantiation related to channels,
including the router, because the parties don't use any channels.
We also took the liberty of renaming $m_{i \to \noti}$ to $c_{i \to \noti}$,
to emphasize the fact that these queues contain ciphertexts, instead of messages.

Next, we pull a bit of a trick.
It turns out that since both parties are honest, we don't need to actually
decrypt the ciphertext.
Instead, one party can simply send the plaintext via a separate channel
to the other.
Applying this gives us:
$$
\Gamma^0 \circ \tx{Keys} = 
\inlinepackage{$\Gamma^1$}{
    &\txbf{view } c_{1\to2}, c_{2\to1} \gets \tx{FifoQueue}.\tx{new}()\cr
    &\txbf{view } m_{1\to2}, m_{2\to1} \gets \tx{FifoQueue}.\tx{new}()\cr
    &(\bullet, \tx{pk}_{\noti}) \gets \tx{Keys}_i()\cr
    \cr
    &\pfn{PKs}{}\cr
    \pind{1} \preturn (\tx{pk}_1, \tx{pk}_2)\cr
    &\begin{aligned}
        &\pfn{$\tx{Send}_{i}$}{m}\cr
        \pind{1} \set{c}{\tx{Enc}(\tx{pk}_{\noti}, m)}\cr
        \pind{1} c_{i\to\noti}.\tx{push}(c)\cr
        \pind{1} m_{i\to\noti}.\tx{push}(m)\cr
    \end{aligned}
    \begin{aligned}
        &\pfn{$\tx{Recv}_{i}$}{}\cr
        \pind{1} \set{c}{\await \tx{c}_{\noti \to i}.\tx{pop}()}\cr
        \pind{1} \set{m}{\await \tx{m}_{\noti \to i}.\tx{pop}()}\cr
        \pind{1} \preturn{m}\cr
    \end{aligned}
}
\circ \tx{Keys}
$$
This is equal because of the correctness property for encryption,
which guarantees that $m = \tx{Dec}(\tx{Enc}(\tx{pk}, m))$.
Furthermore, the timing properties are the same,
since the size of both the $c_{i\to\noti}$ and $m_{i\to\noti}$ queues
are always the same.

At this point, we can offload the decryption to the $\tx{IND}$ game, giving us:
$$
\Gamma^1 \circ \tx{Keys} = 
\inlinepackage{$\Gamma^2$}{
    &\txbf{view } c_{1\to2}, c_{2\to1} \gets \tx{FifoQueue}.\tx{new}()\cr
    &\txbf{view } m_{1\to2}, m_{2\to1} \gets \tx{FifoQueue}.\tx{new}()\cr
    \cr
    &\pfn{PKs}{}\cr
    \pind{1} \preturn (\txbf{super}.\tx{pk}_1, \super.\tx{pk}_2)\cr
    &\begin{aligned}
        &\pfn{$\tx{Send}_{i}$}{m}\cr
        \pind{1} \set{c}{\tx{Challenge}_{\noti}(m)}\cr
        \pind{1} c_{i\to\noti}.\tx{push}(c)\cr
        \pind{1} m_{i\to\noti}.\tx{push}(m)\cr
    \end{aligned}
    \begin{aligned}
        &\pfn{$\tx{Recv}_{i}$}{}\cr
        \pind{1} \set{c}{\await \tx{c}_{\noti \to i}.\tx{pop}()}\cr
        \pind{1} \set{m}{\await \tx{m}_{\noti \to i}.\tx{pop}()}\cr
        \pind{1} \preturn{m}\cr
    \end{aligned}
}
\circ
\begin{pmatrix}
\tx{IND}_0\cr
\otimes\cr
\tx{IND}_0
\end{pmatrix}
$$
We use two instances of $\tx{IND}$, and we disambiguate the functions in
each instance by attaching $1$ or $2$ to each function.

Next, we can hop to $\tx{IND}_1$, since:
$$
\Gamma^2 \circ
\begin{pmatrix}
\tx{IND}_0\cr
\otimes\cr
\tx{IND}_0
\end{pmatrix}
\overset{\epsilon}{\approx}
\Gamma^2 \circ
\begin{pmatrix}
\tx{IND}_1\cr
\otimes\cr
\tx{IND}_1
\end{pmatrix}
$$
with $\epsilon = 2\cdot \tx{IND}$.

If we unroll this last game, we get:
$$
\Gamma^1 \circ
\begin{pmatrix}
\tx{IND}_1\cr
\otimes\cr
\tx{IND}_1
\end{pmatrix}
=
\inlinepackage{$\Gamma^3$}{
    &\txbf{view } c_{1\to2}, c_{2\to1} \gets \tx{FifoQueue}.\tx{new}()\cr
    &\txbf{view } m_{1\to2}, m_{2\to1} \gets \tx{FifoQueue}.\tx{new}()\cr
    &(\tx{sk}_i, \tx{pk}_i) \xleftarrow{\$} \tx{Gen}()\cr
    \cr
    &\pfn{PKs}{}\cr
    \pind{1} \preturn (\tx{pk}_1, \tx{pk}_2)\cr
    &\begin{aligned}
        &\pfn{$\tx{Send}_{i}$}{m}\cr
        \pind{1} \draw{r}{\txbf{M}(|m|)}\cr
        \pind{1} c_{i\to\noti}.\tx{push}(\tx{Enc}(\tx{pk}_{\noti}, r))\cr
        \pind{1} m_{i\to\noti}.\tx{push}(m)\cr
    \end{aligned}
    \begin{aligned}
        &\pfn{$\tx{Recv}_{i}$}{}\cr
        \pind{1} \set{c}{\await \tx{c}_{\noti \to i}.\tx{pop}()}\cr
        \pind{1} \set{m}{\await \tx{m}_{\noti \to i}.\tx{pop}()}\cr
        \pind{1} \preturn{m}\cr
    \end{aligned}
}
$$

Our next step will be to ``defer'' the creation of the fake ciphertexts,
generating them on demand when the ciphertext queue is viewed by
the adversary.
To do this, we maintain a log which saves the length of messages
being sent, and also lets us know when to remove ciphertexts from the log.
This gives us:
$$
\Gamma^4 =
\inlinepackage{$\Gamma^5$}{
    &\txbf{pub } l_{1\to2}, l_{2\to1} \gets \tx{FifoQueue}.\tx{new}()\cr
    &\txbf{view } c_{1\to2}, c_{2\to1} \gets \tx{FifoQueue}.\tx{new}()\cr
    &\txbf{view } m_{1\to2}, m_{2\to1} \gets \tx{FifoQueue}.\tx{new}()\cr
    &(\tx{sk}_i, \tx{pk}_i) \xleftarrow{\$} \tx{Gen}()\cr
    \cr
    &\begin{aligned}
        &\pfn{PKs}{}\cr
        \pind{1} \preturn (\tx{pk}_1, \tx{pk}_2)\cr
        \cr\cr\cr\cr\cr\cr
        &\pfn{$\tx{Send}_{i}$}{m}\cr
        \pind{1} l_{i\to\noti}.\tx{push}((\txt{push}, |m|))\cr
        \pind{1} m_{i\to\noti}.\tx{push}(m)\cr
        \cr
    \end{aligned}
    \begin{aligned}
        &\pfn{$c_{i\to\noti}$}{}\cr
        \pind{1} \pwhile{\tx{cmd} \gets \tx{l}_{i\to\noti}.\tx{pop}() \neq \bot}\cr
        \pind{2} \pif{\tx{cmd} = \txt{pop}}\cr
        \pind{3} c_{i\to\noti}.\tx{pop}()\cr
        \pind{2} \pif{\tx{cmd} = (\txt{push}, |m|)}\cr
        \pind{3} \draw{r}{\txbf{M}(|m|)}\cr
        \pind{3} c_{i\to\noti}.\tx{push}(\tx{Enc}(\tx{pk}_{\noti}, r))\cr
        \pind{1} \preturn{c_{i \to \noti}}\cr
        \cr
        &\pfn{$\tx{Recv}_{i}$}{}\cr
        \pind{1} \set{m}{\await \tx{m}_{\noti \to i}.\tx{pop}()}\cr
        \pind{1} l_{i\to\noti}.\tx{push}((\txt{pop}, |m|))\cr
        \pind{1} \preturn{m}\cr
    \end{aligned}
}
$$

But, at this point the behavior of $\tx{Send}_i$ and $\tx{Recv}_i$ is identical
to that in $\mathscr{Q}$, allowing us to write:
$$
\Gamma^5 = 
\begin{matrix}
\inlinepackage{$S$}{
    &\txbf{view } c_{1\to2}, c_{2\to1} \gets \tx{FifoQueue}.\tx{new}()\cr
    &(\tx{sk}_i, \tx{pk}_i) \xleftarrow{\$} \tx{Gen}()\cr
    \cr
    &\begin{aligned}
        &\pfn{PKs}{}\cr
        \pind{1} \preturn (\tx{pk}_1, \tx{pk}_2)\cr
        \cr\cr\cr\cr\cr\cr
    \end{aligned}
    \begin{aligned}
        &\pfn{$c_{i\to\noti}$}{}\cr
        \pind{1} \pwhile{\tx{cmd} \gets \tx{l}_{i\to\noti}.\tx{pop}() \neq \bot}\cr
        \pind{2} \pif{\tx{cmd} = \txt{pop}}\cr
        \pind{3} c_{i\to\noti}.\tx{pop}()\cr
        \pind{2} \pif{\tx{cmd} = (\txt{push}, |m|)}\cr
        \pind{3} \draw{r}{\txbf{M}(|m|)}\cr
        \pind{3} c_{i\to\noti}.\tx{push}(\tx{Enc}(\tx{pk}_{\noti}, r))\cr
    \end{aligned}
}\cr
\otimes\cr
1(\tx{Send}_i, \tx{Recv}_i)\cr
\end{matrix}
\circ \tx{Inst}_{\tx{H}}(\mathscr{Q})
$$
which concludes this part of our proof, having written out our simulator,
and proven that $\tx{Inst}_{\tx{H}}(\mathscr{P}) \overset{\epsilon}{\approx} \tx{SimInst}_{S, \tx{H}}(\mathscr{Q})$.


\txbf{Malicious Case:}
Without loss of generality, we can consider the case where $P_1$ is malicious.
This is because the difference between the parties is just a matter of renaming
variables, so the case where $P_2$ is malicious would be the same.
Let $\tx{M}$ denote this corruption model.
We prove that $\mathscr{P} \overset{0}{\leadsto}_{\{\tx{M}\}} \mathscr{Q}$,
which naturally implies the slightly higher upper bound of $2 \cdot \tx{IND}$.

We start by unrolling $\tx{Inst}_{\tx{M}}(\mathscr{P})$, to get:
$$
\tx{Inst}_{\tx{M}}(\mathscr{P}) =
\inlinepackage{$\Gamma^1$}{
    &\txbf{view } c_{1\to2}, c_{2\to1} \gets \tx{FifoQueue}.\tx{new}()\cr
    &(\tx{sk}_2, \tx{pk}_1) \gets \tx{Keys}_2()\cr
    \cr
    &\begin{aligned}
        &\pfn{PKs}{}\cr
        \pind{1} \preturn{(\tx{pk}_1, \tx{pk}_2)}\cr
        \cr
        &\pfn{$\tx{Send}_{1}$}{c}\cr
        \pind{1} c_{1 \to 2}.\tx{push}(c)\cr
        \cr
        &\pfn{$\tx{Send}_{2}$}{m}\cr
        \pind{1} \set{c}{\tx{Enc}(\tx{pk}_1, m)}\cr
        \pind{1} c_{2 \to 1}.\tx{push}(c)\cr
    \end{aligned}
    \begin{aligned}
        &\pfn{$\tx{Keys}_1$}{}\cr
        \pind{1} \preturn{\super.\tx{Keys}_1()}\cr
        \cr
        &\pfn{$\tx{Recv}_{1}$}{}\cr
        \pind{1} \preturn{\await c_{2 \to 1}.\tx{pop}()}\cr
        \cr
        &\pfn{$\tx{Recv}_{2}$}{m}\cr
        \pind{1} \set{c}{\await c_{1 \to 2}.\tx{pop}()}\cr
        \pind{1} \preturn{\tx{Dec}(\tx{sk}_2, c)}\cr
    \end{aligned}
}
\circ \tx{Keys}
$$
The key affordances for malicious corruption are that the adversary
can now see the output of $\tx{Keys}_1$, including their secret key,
and the public key of the other party,
and that they have direct access to $c_{1 \to 2}$.
This allows them to send potentially ``fake'' ciphertexts to the other
party, rather than going through the decryption process.

Next, we explicitly include the code of $\tx{Keys}$, and also include
an additional key pair, used in $\tx{Recv}_2$, this key pair encrypts
and then immediately decrypts the message being received, and thus
has no effect by the correctness property of encryption.
Writing this out, we get:
$$
\Gamma^1 \circ \tx{Keys} =
\inlinepackage{$\Gamma^2$}{
    &\txbf{view } c_{1\to2}, c_{2\to1} \gets \tx{FifoQueue}.\tx{new}()\cr
    &(\tx{sk}_1, \tx{pk}_1),\
    (\tx{sk}_2, \tx{pk}_2),\
    (\tx{sk}'_2, \tx{pk}'_2) \gets \tx{Gen}()\cr
    \cr
    &\begin{aligned}
        &\pfn{PKs}{}\cr
        \pind{1} \preturn{(\tx{pk}_1, \tx{pk}_2)}\cr
        \cr
        &\pfn{$\tx{Send}_{1}$}{c}\cr
        \pind{1} c_{1 \to 2}.\tx{push}(c)\cr
        \cr
        &\pfn{$\tx{Send}_{2}$}{m}\cr
        \pind{1} \set{c}{\tx{Enc}(\tx{pk}_1, m)}\cr
        \pind{1} c_{2 \to 1}.\tx{push}(c)\cr
        \cr
        \cr
        \cr
    \end{aligned}
    \begin{aligned}
        &\pfn{$\tx{Keys}_1$}{}\cr
        \pind{1} \preturn{(\tx{sk}_1, \tx{pk}_2)}\cr
        \cr
        &\pfn{$\tx{Recv}_{1}$}{}\cr
        \pind{1} \preturn{\await c_{2 \to 1}.\tx{pop}()}\cr
        \cr
        &\pfn{$\tx{Recv}_{2}$}{m}\cr
        \pind{1} \set{c}{\await c_{1 \to 2}.\tx{pop}()}\cr
        \pind{1} \set{m}{\tx{Dec}(\tx{sk}_2, c)}\cr
        \pind{1} \set{c'}{\tx{Enc}(\tx{pk}'_2, m)}\cr
        \pind{1} \set{m}{\tx{Dec}(\tx{sk}'_2, c')}\cr
        \pind{1} \preturn{m}\cr
    \end{aligned}
}
$$

The next step we perform is a bit of a trick.
We swap the names of $\tx{sk}_2$ and $\tx{sk}'_2$, as well as $\tx{pk}_2$ and $\tx{pk}'_2$,
after all, renaming has no effect on a system.
We also create a separate message queue $m_{1 \to 2}$ which will be used
to send messages directly.
This gives us:
$$
\Gamma^2 =
\inlinepackage{$\Gamma^3$}{
    &\txbf{view } m_{1\to 2}, c_{1\to2}, c_{2\to1} \gets \tx{FifoQueue}.\tx{new}()\cr
    &(\tx{sk}_1, \tx{pk}_1),\
    (\tx{sk}_2, \tx{pk}_2),\
    (\tx{sk}'_2, \tx{pk}'_2) \gets \tx{Gen}()\cr
    \cr
    &\begin{aligned}
        &\pfn{PKs}{}\cr
        \pind{1} \preturn{(\tx{pk}_1, \tx{pk}'_2)}\cr
        \cr
        &\pfn{$\tx{Send}_{1}$}{c}\cr
        \pind{1} c_{1 \to 2}.\tx{push}(c)\cr
        \pind{1} m \gets \tx{Dec}(\tx{sk}'_2, c)\cr
        \pind{1} m_{1 \to 2}.\tx{push}(m)\cr
        \cr
        &\pfn{$\tx{Send}_{2}$}{m}\cr
        \pind{1} \set{c}{\tx{Enc}(\tx{pk}_1, m)}\cr
        \pind{1} c_{2 \to 1}.\tx{push}(c)\cr
        \cr
    \end{aligned}
    \begin{aligned}
        &\pfn{$\tx{Keys}_1$}{}\cr
        \pind{1} \preturn{(\tx{sk}_1, \tx{pk}'_2)}\cr
        \cr
        &\pfn{$\tx{Recv}_{1}$}{}\cr
        \pind{1} \preturn{\await c_{2 \to 1}.\tx{pop}()}\cr
        \cr
        &\pfn{$\tx{Recv}_{2}$}{m}\cr
        \pind{1} \set{c}{\await c_{1 \to 2}.\tx{pop}()}\cr
        \pind{1} \set{m}{\await m_{1 \to 2}.\tx{pop}()}\cr
        \pind{1} \set{c'}{\tx{Enc}(\tx{pk}_2, m)}\cr
        \pind{1} \set{m}{\tx{Dec}(\tx{sk}_2, c')}\cr
        \pind{1} \preturn{m}\cr
    \end{aligned}
}
$$
Notice that at this point $\tx{sk}_2$ and $\tx{pk}_2$ now don't actually
do anything, since they don't actually modify the message in $\tx{Recv}_2$.
The main remaining barrier to writing this as a simulator over $\mathscr{Q}$
is that the ciphertext queues $c_{i \to \noti}$ are modified both in functions
we control $\tx{Send}_1$ and $\tx{Recv}_1$, but also in the two functions
which we don't control $\tx{Send}_2$, and $\tx{Recv}_2$, and will eventually
need to become pass through functions for $\mathscr{Q}$.

For $\tx{Recv}_2$, it modifies $c_{1 \to 2}$ by popping elements off of it.
We can emulate this behavior by reading the access log of $l_{1 \to 2}$ we
get from $\mathscr{Q}$, and using the pop commands inside to modify $c_{1 \to 2}$
when necessary.

For $\tx{Send}_2$, our task is a bit harder, since we need to create an encryption
of $m$, and the log will only contain $|m|$.
However, our simulator over $\mathscr{Q}$ will be able to receive messages
on behalf of the first party, allowing us to retrieve the message,
and then create a simulate ciphertext by encrypting it.

Putting these ideas together, we write:
$$
\Gamma^3 =
\begin{matrix}
\inlinepackage{$S$}{
    &c_{1\to2}, c_{2\to1} \gets \tx{FifoQueue}.\tx{new}()\cr
    &(\tx{sk}_1, \tx{pk}_1),\
    (\tx{sk}'_2, \tx{pk}'_2) \gets \tx{Gen}()\cr
    \cr
    &\begin{aligned}
        &\pfn{PKs}{}\cr
        \pind{1} \preturn{(\tx{pk}_1, \tx{pk}'_2)}\cr
        \cr
        &\pfn{$\tx{Keys}_1$}{}\cr
        \pind{1} \preturn{(\tx{sk}_1, \tx{pk}'_2)}\cr
        \cr
        &\pfn{$c_{i \to \noti}$}{}\cr
        \pind{1} \tx{Update}_{i \to \noti}()\cr
        \pind{1} \preturn{c_{i \to \noti}}\cr
        \cr
        &\pfn{$\tx{Send}_{1}$}{c}\cr
        \pind{1} \tx{Update}_{1 \to 2}()\cr
        \pind{1} c_{1 \to 2}.\tx{push}(c)\cr
        \pind{1} m \gets \tx{Dec}(\tx{sk}'_2, c)\cr
        \pind{1} \super.\tx{Send}_1(m)\cr
        \cr
    \end{aligned}
    \begin{aligned}
        &\pfn{$\tx{Update}_{1 \to 2}$}{}\cr
        \pind{1} \pwhile{\tx{cmd} \gets l_{1 \to 2}.\tx{pop}() \neq \bot}\cr
        \pind{2} \pif{\tx{cmd} = \txt{pop}}\cr
        \pind{3} c_{1 \to 2}.\tx{pop}()\cr
        \cr
        &\pfn{$\tx{Update}_{2 \to 1}$}{}\cr
        \pind{1} \pwhile{\tx{cmd} \gets l_{2 \to 1}.\tx{pop}() \neq \bot}\cr
        \pind{2} \pif{\tx{cmd} = (\txt{push}, \bullet)}\cr
        \pind{3} \set{m}{\await \super.\tx{Recv}_1()}\cr
        \pind{3} c_{2 \to 1}.\tx{push}(\tx{Enc}(\tx{pk}_1, m))\cr
        \cr
        &\pfn{$\tx{Recv}_{1}$}{}\cr
        \pind{1} \tx{Update}_{2 \to 1}()\cr
        \pind{1} \preturn{\await c_{2 \to 1}.\tx{pop}()}
        \cr
    \end{aligned}
}\cr
\otimes\cr
1(\{\tx{Send}_2, \tx{Recv}_2\})
\end{matrix}
\circ
\tx{Inst}_{\tx{M}}(\mathscr{Q})
$$
We make sure to update both queues whenever necessary.
This includes when they're viewed by the adversary, but also
whenever we modify the queues ourselves, so that we've popped or pushed
everything that we need to before using the queue.

This simulator is effectively creating a man-in-the-middle attack on the adversary,
by providing them with the wrong public key, allowing them to decrypt the ciphertexts
they see.
On the other side, the simulator can receive messages on behalf of the adversary,
and then re-encrypt them to create the fake ciphertext queue.

Having now proved the upper bound for all the corruption models in $\mathscr{C}$,
we conclude that our claim holds.

$\blacksquare$
\end{claim}

\subsection{Drawing a Random Value}

The basic goal of this subsection is to develop a protocol for securely
choosing a common random value.
This process should such that no party can bias the resulting value.
We will follow the common paradigm of ``commit-reveal'', where the parties
first commit to their random values, then wait for all these commitments to have been made,
before finally opening the random values and mixing them together.
This ensures that no party can bias the result, since they have to choose
their contribution before learning any information about the result.

We start by defining the ideal protocol for drawing a random value.
We'll be working over an additive group $\mathbb{G}$,
and assuming that we have parties numbered $1, \ldots, n$.
The core functionality we use allows each party to set a random value,
and then have the functionality add them together.
This is contained in Game~\ref{game:add}.

\begin{game}{game:add}{Addition Functionality}
\package{$F[\tx{Add}]$}{
    &x_1, \ldots, x_n \gets \bot\cr
    \cr
    &\begin{aligned}
        &\pfn{$(1)\tx{Add}_{i}$}{x}\cr
        \pind{1} \set{x_i}{x}\cr
        \pind{1} \pwait{\forall i.\ x_i \neq \bot}\cr
        \pind{1} \preturn{\textstyle \sum_i x_i}\cr
    \end{aligned}
    \quad
    \begin{aligned}
        &\pfn{Leak}{}\cr
        \pind{1} \pif{\exists i.\ x_i = \bot}\cr
        \pind{2} \preturn{(\txt{waiting}, \{i \mid x_i = \bot\})}\cr
        \pind{1} \preturn{(\txt{done}, \textstyle \sum_i x_i)}\cr
    \end{aligned}
}
\end{game}

This game works by first collecting a contribution from each party,
and then adding them together.
At any point after all contributions have been gathered,
the adversary can also see their sum through the $\tx{Leak}$ function.
Note that we only allow a contribution to be provided once,
as marked by the $(1)$ in front of the function.
This will be the case for the random sampling as well.

Using this functionality, we create an ideal protocol for sampling
a random value, defined in Protocol~\ref{prot:rand}

\begin{protocol}{prot:rand}{Ideal Random Protocol}
$\mathscr{P}[\tx{IdealRand}]$ is characterized by:
\begin{itemize}
    \item $F := 1(\tx{Add})$,
    \item $\tx{Leakage} = \{\tx{Leak}\}$,
    \item And $n$ players defined via the following system, for $i \in [n]$:
        \package{$P_i$}{
            &\pfn{$(1)\tx{Rand}_i$}{}\cr
            \pind{1} \draw{x}{\mathbb{G}}\cr
            \pind{1} \preturn{\await \tx{Add}_i(x)}\cr
        }
\end{itemize}
\end{protocol}

The idea is that each party samples a random value, and then submits
that to the addition functionality.
If at least one of the values was sampled randomly, 
then the final result is also random.
Technically, this is an \emph{endemic} random functionality,
in the sense that malicious parties are allowed to choose their own randomness.
We also don't embed the $F[\tx{Add}]$ functionality into the protocol itself,
which makes the ideal protocol technically $\mathscr{P}[\tx{IdealRand}] \circ F[\tx{Add}]$.
We do this to allow considering a slightly modified variant of the protocol,
which uses a version of the addition functionality leaking more information,
defined in Game~\ref{game:add'}.

\begin{game}{game:add'}{Addition Functionality}
\package{$F[\tx{Add}']$}{
    &x_1, \ldots, x_n \gets \bot\cr
    \cr
    &\begin{aligned}
        &\pfn{$(1)\tx{Add}_{i}$}{x}\cr
        \pind{1} \set{x_i}{x}\cr
        \pind{1} \pwait{\forall i.\ x_i \neq \bot}\cr
        \pind{1} \preturn{\textstyle \sum_i x_i}\cr
    \end{aligned}
    \quad
    \begin{aligned}
        &\pfn{Leak}{}\cr
        \pind{1} \pif{\exists i.\ x_i = \bot}\cr
        \pind{2} \preturn{(\txt{waiting}, \{i \mid x_i = \bot\})}\cr
        \pind{1} \preturn{(\txt{done}, [x_i \mid i \in [n]])}\cr
    \end{aligned}
}
\end{game}

The difference in $F[\tx{Add}']$ is simply that the entire list of contributions
is leaked, rather than just their sum.
We introduce this functionality because it will be simpler
to show that our concrete protocol is simulated by this slightly stronger functionality.
Thankfully, the difference doesn't matter in the end, because we can simulate
the stronger functionality from the weaker one.

\begin{claim}
    Let $\mathscr{C}$ be the corruption class where all up to $n - 1$ parties
    are corrupted.
    It then holds that:
    $$
    \mathscr{P}[\tx{IdealRand}] \circ F[\tx{Add}'] \overset{0}{\leadsto}_{\mathscr{C}}
    \mathscr{P}[\tx{IdealRand}] \circ F[\tx{Add}]
    $$
    \txbf{Proof:} The crux of the proof is that we can simply invent random
    shares for the honest parties, subject to the constraint that the sum
    of all shares is the same.

    Now, onto the more formal proof.
    We assume, without loss of generality,
    that $1, \ldots, h$ are the indices of the honest parties,
    and $h + 1, \ldots, m$ the semi-honest parties.
    Another convention we use is that $j$ is used as a subscript
    for semi-honest parties, and $k$ for malicious parties.

    The only difference between the instantiation of both protocols lies
    in $\tx{Leak}$.
    Otherwise, the behavior of all the functions is identical.
    Thus, we simply need to write a simulator for that function.
    The basic idea is to intercept calls to the corrupted parties to
    learn their contributions, and then simply invent some fake but plausible
    contributions for the honest parties.

    This gives us:

    $$
    \begin{matrix}
    \inlinepackage{$S$}{
        &\set{\tx{faked}}{\txt{false}}\cr
        &\set{x'_1, \ldots, x'_n}{\bot}\cr
        \cr
        &\begin{aligned}
            &\pfn{$(1)\tx{Add}_k$}{x}\cr
            \pind{1} \set{x'_k}{x}\cr
            \pind{1} \preturn{\tx{Add}_k(x)}\cr
            \cr
            &\pfn{$\tx{Contribution}_j$}{}\cr
            \pind{1} \txbf{assert} (\txt{call}, x) \in \tx{log}.\tx{Add}_j\cr
            \pind{1} \preturn{x}\cr
        \end{aligned}
        \begin{aligned}
        &\pfn{Leak}{}\cr
        \pind{1} \tx{out} \gets \super.\tx{Leak}()\cr
        \pind{1} \pif{\tx{out} = (\txt{waiting}, \tx{on})}\cr
        \pind{2} \preturn{(\txt{waiting}, \tx{on})}\cr
        \pind{1} \pif{\tx{faked} = \txt{false}}\cr
        \pind{2} \set{\tx{faked}}{\txt{true}}\cr
        \pind{2} \pfor{j \in h + 1, \ldots, m}\cr
        \pind{3} x'_j \gets \tx{Contribution}_j()\cr
        \pind{2} \draw{x'_2, \ldots, x'_h}{\mathbb{G}}\cr
        \pind{2} \set{x'_1}{\textstyle \tx{out} - \sum_{i \in [2, \ldots, n]} x_i}\cr
        \pind{1} \preturn{(\txt{done}, [x'_i \mid i \in [n]])}\cr
        \end{aligned}
    }\cr
    \otimes\cr
    1(\ldots)
    \end{matrix}
    $$

    The shares of the malicious parties are obtained by catching them
    when the call to $\tx{Add}_k$ is made, whereas for the semi-honest party
    we instead fetch them from the log.
    Note that because the leakage is only made available once all the parties
    have contributed, we're guaranteed to have already seen the shares
    from the corrupted parties by the time we fake the other shares.

    It should be clear that:
    $$
    \tx{Inst}_C(\mathscr{P}[\tx{IdealRand}] \circ F[\tx{Add}']) = 
    \tx{SimInst}_{S, C}(\mathscr{P}[\tx{IdealRand}] \circ F[\tx{Add}])
    $$
    concluding our proof.

    $\blacksquare$
\end{claim}

The next task on our hands is to write down the concrete protocol
for sampling randomness via the commit-reveal paradigm.
To do that, we first need to define an appropriate commitment functionality,
which we do in Game~\ref{game:com}

\begin{game}{game:com}{Commitment Functionality}
\package{$F[\tx{Com}]$}{
    &\set{c_1, \ldots, c_n}{\bot}\cr
    &\set{o_1, \ldots, o_n}{\txt{false}}\cr
    \cr
    &\begin{aligned}
        &\pfn{$(1)\tx{Commit}_i$}{x}\cr
        \pind{1} \set{c_i}{x}\cr
        \cr
        &\pfn{$(1)\tx{Open}_i$}{}\cr
        \pind{1} \txbf{assert } c_i \neq \bot\cr
        \pind{1} \set{o_i}{\txt{true}}\cr
        \cr
        \cr
    \end{aligned}
    \begin{aligned}
        &\pfn{$\tx{View}_i$}{x}\cr
        \pind{1} \pif{c_i = \bot}\cr
        \pind{2} \preturn{\txt{empty}}\cr
        \pind{1} \pif{\neg o_i}\cr
        \pind{2} \preturn{\txt{set}}\cr
        \pind{1} \pelse\cr
        \pind{2} \preturn{(\txt{open}, c_i)}\cr
        \cr
    \end{aligned}
}
\end{game}

This functionality acts as a one shot commitment for each participant.
Each party can commit to a value, and then open it at a later point in time.
At any time, each participant can view the state of another participant's
commitment.
This view tells us what stage of the commitment the participant is at,
along with their committed value, once opened.

We can now define a protocol sampling randomness, thanks to this commitment
scheme, in Protocol~\ref{prot:realrand}.

\begin{protocol}{prot:realrand}{Random Protocol}
$\mathscr{P}[\tx{Rand}]$ is characterized by:
\begin{itemize}
    \item $F := F[\tx{Com}]$,
    \item $\tx{Leakage} = \{\tx{View}_1, \ldots, \tx{View}_n\}$,
    \item And $n$ players defined via the following system, for $i \in [n]$:
        \package{$P_i$}{
            &\pfn{$(1)\tx{Rand}_i$}{}\cr
            \pind{1} \draw{x}{\mathbb{G}}\cr
            \pind{1} \tx{Commit}_i(x)\cr
            \pind{1} \pwait{\forall i. \tx{View}_i() \neq \txt{empty}}\cr
            \pind{1} \tx{Open}_i()\cr
            \pind{1} \pwait{\forall i. \tx{View}_i() = (\txt{open}, x_i)}\cr
            \pind{1} \preturn{\textstyle \sum_i x_i}\cr
        }
\end{itemize}
\end{protocol}

The idea is quite simple, everybody generates a random value,
commits to it, and then once everybody has committed, they open the value,
and sum up all the contributions.
The result is, as we'll prove, a random value that no participant
can bias.

Unfortunately, it's not quite the case that $\mathscr{P}[\tx{Rand}]$
is simulated by $\mathscr{P}[\tx{IdealRand}]$.
The reason is a consequence of the timing properties
of the protocols.
Indeed, in $\mathscr{P}[\tx{IdealRand}]$, it suffices to activate
each participant once in order to learn the result,
whereas in $\mathscr{P}[\tx{Rand}]$, two activations are needed,
once to commit, and another time to open.

Instead we introduce a separate protocol, making use of
a ``synchronization'' functionality, defined in Game~\ref{game:sync}.

\begin{game}{game:sync}{Synchronization Game}
\package{$F[\tx{Sync}]$}{
    &\set{\txbf{view } \tx{done}_1, \ldots, \tx{done}_n}{\txt{false}}\cr
    \cr
    &\pfn{$(1)\tx{Sync}_i$}{}\cr
    \pind{1} \tx{done}_i \gets \txt{true}\cr
    \pind{1} \pwait{\forall i.\ \tx{done}_i = \txt{true}}\cr
}
\end{game}

This functionality allows the parties to first ``synchronize'',
by waiting for each party to contribute, before being able to continue.

The protocol using this functionality is then called $\mathscr{Q}$,
and defined in Protocol~\ref{prot:simrand}

\begin{protocol}{prot:simrand}{Synchronized Random Protocol}
    $\mathscr{Q}$ is characterized by:
    \begin{itemize}
        \item $F = F[\tx{Sync}]$,
        \item $\tx{Leakage} := \{\tx{done}_1, \ldots, \tx{done}_n\}$,
        \item And $n$ players defined by the following system, for $i \in [n]$:
        \package{$P_i$}{
            &\pfn{$(1)\tx{Rand}_i$}{}\cr
            \pind{1} \set{\tx{out}}{\await \super.\tx{Rand}_i()}\cr
            \pind{1} \await \tx{Sync}_i()\cr
            \pind{1} \preturn{\tx{out}}\cr
        }
    \end{itemize}
\end{protocol}

The full protocol we consider is $\mathscr{Q} \lhd (\mathscr{P}[\tx{IdealRand}] \circ F[\tx{Add}])$,
which can perfectly simulate $\mathscr{P}[\tx{Rand}]$, as we now prove.

\begin{claim}
    Let $\mathscr{C}$ be the class of corruptions where up to $n - 1$ parties
    are corrupt.
    Then it holds that:
    $$
    \mathscr{P}[\tx{Rand}] \overset{0}{\leadsto}_{\mathscr{C}} \mathscr{Q} \lhd (\mathscr{P}[\tx{IdealRand}] \circ F[\tx{Add}])
    $$
    \txbf{Proof:} Thanks to the composition properties of protocols, it suffices to prove the above claim using $F[\tx{Add}']$ instead,
    since we already proved that:
    $$
    \mathscr{P}[\tx{IdealRand}] \circ F[\tx{Add}'] \overset{0}{\leadsto}_{\mathscr{C}}
    \mathscr{P}[\tx{IdealRand}] \circ F[\tx{Add}]
    $$

    As before, we let $1, \ldots, h$ be the indices of honest parties,
    $h + 1, \ldots, m$ the indices of semi-honest parties,
    and use $i, j, k$ for denoting indices of honest, semi-honest, and malicious
    parties, respectively.
    We start by unrolling $\tx{Inst}_C(\mathscr{P}[\tx{Rand}])$, to get:
    
    \package{$\Gamma^0$}{
        &\set{x_1, \ldots, x_n, \tx{rush}_{m + 1}, \ldots, \tx{rush}_n}{\bot}\cr
        &\set{\tx{o}_1, \ldots, \tx{o}_n}{\txt{false}}\cr
        &\set{\tx{log}_j}{\tx{NewLog}()}\cr
        \cr
        &\begin{aligned}
            &\pfn{$(1)\tx{Rand}_i$}{}\cr
            \pind{1} \draw{x_i}{\mathbb{G}}\cr
            \pind{1} \pwait{\forall i.\ \tx{View}_i \neq \txt{empty}}\cr
            \pind{1} \set{o_i}{\txt{true}}\cr
            \pind{1} \pwait{\forall i.\ \tx{View}_i = (\txt{open}, x_i)}\cr
            \pind{1} \preturn{\textstyle \sum_i x_i}\cr
            \cr
            &\pfn{$\tx{View}_i$}{}\cr
            \pind{1} \pif{x_i = \bot}\cr
            \pind{2} \preturn{\txt{empty}}\cr
            \pind{1} \pif{\neg o_i}\cr
            \pind{2} \preturn{\txt{set}}\cr
            \pind{1} \pelse\cr
            \pind{2} \preturn{(\txt{open}, c_i)}\cr
            \cr
        \end{aligned}
        \begin{aligned}
            &\pfn{$(1)\tx{Rand}_j$}{}\cr
            \pind{1} \tx{log}_j.\tx{Rand}_j.\tx{push}(\txt{input})\cr
            \pind{1} \draw{x_i}{\mathbb{G}}\cr
            \pind{1} \tx{log}_j.\tx{Commit}_j.\tx{push}((\txt{call}, x_i))\cr
            \pind{1} \pwait{\forall i.\ \tx{View}_i \neq \txt{empty}}\cr
            \pind{1} \tx{log}_j.\tx{Open}_j.\tx{push}(\txt{call})\cr
            \pind{1} \set{o_i}{\txt{true}}\cr
            \pind{1} \pwait{\forall i.\ \tx{View}_i = (\txt{open}, x_i)}\cr
            \pind{1} \preturn{\textstyle \sum_i x_i}\cr
            \cr
            &\pfn{$(1)\tx{Commit}_k$}{x}\cr
            \pind{1} \set{x_k}{x}\cr
            \cr
            &\pfn{$(1)\tx{Open}_k$}{}\cr
            \pind{1} \txbf{assert } x_k \neq \bot\cr
            \pind{1} \set{o_k}{\txt{true}}\cr
        \end{aligned}\cr
    }

    Here we've just inlined the main elements of the game.
    The key difference for the semi-honest parties is that we're able
    to see the randomness they used, since they commit to it.
    For the malicious parties, they can commit to any value they want,
    and can also choose when to open their values.

    We now rewrite this game slightly, to make the connection with what we're
    trying to simulate a bit clearer:

    \package{$\Gamma^1$}{
        &\set{x_1, \ldots, x_n, \tx{rush}_{m + 1}, \ldots, \tx{rush}_n}{\bot}\cr
        &\set{\tx{done}_1, \ldots, \tx{done}_n}{\txt{false}}\cr
        &\set{\tx{log}_j}{\tx{NewLog}()}\cr
        \cr
        &\begin{aligned}
            &\pfn{$(1)\tx{Rand}_i$}{}\cr
            \pind{1} \draw{x_i}{\mathbb{G}}\cr
            \pind{1} \pwait{\forall i.\ \tx{View}_i \neq \txt{empty}}\cr
            \pind{1} \set{\tx{done}_i}{\txt{true}}\cr
            \pind{1} \pwait{\forall i.\ \tx{View}_i = (\txt{open}, x_i)}\cr
            \pind{1} \preturn{\textstyle \sum_i x_i}\cr
            \cr
            &\pfn{$\tx{View}_i$}{}\cr
            \pind{1} \pif{\tx{Leak}() = (\txt{waiting}, s) \land i \in s}\cr
            \pind{2} \preturn{\txt{empty}}\cr
            \pind{1} \txbf{else } \pif{\tx{done}_i}\cr
            \pind{2} \pif{\tx{rush}_i \neq \bot}\cr
            \pind{3} \preturn{(\txt{open}, \tx{rush}_i)}\cr
            \pind{2} \txbf{assert } (\txt{done}, [y_i]) = \tx{Leak}()\cr
            \pind{2} \preturn{(\txt{open}, y_i)}\cr
            \pind{1} \preturn{\txt{set}}\cr
            \cr
            &\pfn{$\tx{log}_j$}{}\cr
            \pind{1} \set{\tx{log}'_j}{\tx{NewLog}()}\cr
            \pind{1} \set{\tx{log}'_j.\tx{Rand}_j}{\tx{log}_j.\tx{Rand}_j}\cr
            \pind{1} \set{\tx{log}'_j.\tx{Commit}_j}{\tx{log}_j.\tx{Add}_j}\cr
            \pind{1} \set{\tx{log}'_j.\tx{Open}_j}{\tx{log}_j.\tx{Sync}_j}\cr
            \pind{1} \preturn{\tx{log}'_j}\cr
        \end{aligned}
        \begin{aligned}
            &\pfn{$(1)\tx{Rand}_j$}{}\cr
            \pind{1} \tx{log}_j.\tx{Rand}_j.\tx{push}(\txt{input})\cr
            \pind{1} \draw{x_i}{\mathbb{G}}\cr
            \pind{1} \tx{log}_j.\tx{Add}_j.\tx{push}((\txt{call}, x_i))\cr
            \pind{1} \pwait{\forall i.\ \tx{View}_i \neq \txt{empty}}\cr
            \pind{1} \tx{log}_j.\tx{Sync}_j.\tx{push}(\txt{call})\cr
            \pind{1} \set{o_i}{\txt{true}}\cr
            \pind{1} \pwait{\forall i.\ \tx{View}_i = (\txt{open}, x_i)}\cr
            \pind{1} \preturn{\textstyle \sum_i x_i}\cr
            \cr
            &\pfn{$(1)\tx{Commit}_k$}{x}\cr
            \pind{1} \set{\tx{rush}_k}{x}\cr
            \pind{1} \set{x_k}{x}\cr
            \cr
            &\pfn{$(1)\tx{Open}_k$}{}\cr
            \pind{1} \txbf{assert } \tx{rush}_k \neq \bot\cr
            \pind{1} \set{\tx{done}_k}{\txt{true}}\cr
            \cr
            &\pfn{Leak}{}\cr
            \pind{1} \pif{\exists i.\ x_i = \bot}\cr
            \pind{2} \preturn{(\txt{waiting}, \{i \mid x_i = \bot\})}\cr
            \pind{1} \preturn{(\txt{done}, [x_i \mid i \in [n]])}\cr
        \end{aligned}
    }
    First of all, we've renamed several variables, like $o_i$ becoming
    $\tx{done}_i$, which has no effect on the game, of course.
    We've also introduced a secondary set of variables $\tx{rush}_k$
    to hold the values the malicious parties are committing to.
    We do this to stress the fact that the simulator will be able to see
    and capture these values.
    We also modify the logging in the semi-honest parties
    to use different names, reflecting what will happen in the eventual
    semi-honest party of $\mathscr{Q}$.
    This requires introducing a $\tx{log}_j$ function which will produce
    a simulated log by renaming these entries.

    Finally, the biggest change is in the $\tx{View}_i$ functions.
    We've rewritten the logic to be based on this $\tx{Leak}$ method
    we've introduced, which informs of us the status of the contributions.
    This gives us enough information to simulate the views accurately.
    For the honest parties, we know that they'll only open their values
    after everybody has already committed, so the assertion will always pass.
    This may not be the case for malicious parties, which may ``rush'',
    opening their values \emph{before} the other parties have finished committing.
    This is why it's important to keep track of their commitments separately,
    so that we can present them inside the view, if necessary.

    At this point, the next step is to realize that all of this logic
    can in fact work inside of a simulator, written as:

    $$
    \begin{matrix}
    \inlinepackage{$S$}{
        &\set{\tx{rush}_{m + 1}, \ldots, \tx{rush}_n}{\bot}\cr
        &\begin{aligned}
            &\pfn{$\tx{View}_i$}{}\cr
            \pind{1} \pif{\tx{Leak}() = (\txt{waiting}, s) \land i \in s}\cr
            \pind{2} \preturn{\txt{empty}}\cr
            \pind{1} \txbf{else } \pif{\tx{done}_i}\cr
            \pind{2} \pif{\tx{rush}_i \neq \bot}\cr
            \pind{3} \preturn{(\txt{open}, \tx{rush}_i)}\cr
            \pind{2} \txbf{assert } (\txt{done}, [y_i]) = \tx{Leak}()\cr
            \pind{2} \preturn{(\txt{open}, y_i)}\cr
            \pind{1} \preturn{\txt{set}}\cr
            \cr
        \end{aligned}
        \begin{aligned}
            &\pfn{$(1)\tx{Commit}_k$}{x}\cr
            \pind{1} \set{\tx{rush}_k}{x}\cr
            \pind{1} \tx{Add}_k(x)\cr
            \cr
            &\pfn{$(1)\tx{Open}_k$}{}\cr
            \pind{1} \txbf{assert } \tx{rush}_k \neq \bot\cr
            \pind{1} \tx{Sync}_k()\cr
            \cr
            &\pfn{$\tx{log}_j$}{}\cr
            \pind{1} \set{\tx{log}'_j}{\tx{NewLog}()}\cr
            \pind{1} \set{\tx{log}'_j.\tx{Rand}_j}{\super.\tx{log}_j.\tx{Rand}_j}\cr
            \pind{1} \set{\tx{log}'_j.\tx{Commit}_j}{\super.\tx{log}_j.\tx{Add}_j}\cr
            \pind{1} \set{\tx{log}'_j.\tx{Open}_j}{\super.\tx{log}_j.\tx{Sync}_j}\cr
            \pind{1} \preturn{\tx{log}'_j}\cr
        \end{aligned}
        \begin{aligned}

        \end{aligned}
    }
        \cr
        \otimes\cr
        1(\ldots)
    \end{matrix}
    $$

    And this concludes our proof, having shown that:
    $$
    \tx{Inst}_C(\mathscr{P}[\tx{Rand}]) = \tx{SimInst}_{S, C}(\mathscr{Q} \lhd (\mathscr{P}[\tx{IdealRand}] \circ F[\tx{Add}]))
    $$

    $\blacksquare$
\end{claim}
\section{Differences with UC Security}
\section{Conclusion}

In this work, we sought to develop a modular framework for analyzing
the security of protocols.
We did this by extending the standalone security
formalism of state-separable proofs \cite{AC:BDFKK18}.
The result is a framework for protocol security with similar
modular properties to those of state-separable proofs,
and with a strong connection with that formalism,
allowing for results in standalone security to be used
in showing the security of protocols.

While we believe our framework is already suitable
for proving the security of protocols, we expect
shortcomings to be discovered as the framework sees more use.
The novelty of the framework also provides a disadvantage
in that not many proofs have been written in it,
and many common UC idioms may not translate directly
either.
We hope that this disadvantage can diminish over time as more
more work is conducted using this or similar frameworks.

\bibliographystyle{alpha}
{\small \bibliography{cryptobib/abbrev3, cryptobib/crypto, bib}}
\clearpage
\appendix

\section{Additional Game Definitions}

In this section, we include explicit definitions of several games we
use throughout the rest of this work.
While we expect these notions to be familiar, we think the precise
details are worth spelling out here.

\subsection{Encryption}
\label{app:encryption}
A public key encryption scheme consists of types $\txbf{PK}, \txbf{PK}, \txbf{DK}, \txbf{M}, \txbf{C}$,
along with probabilistic functions $\tx{Enc} : \txbf{PK} \times \txbf{M} \xleftarrow{\$} \txbf{C}$ and $\tx{Dec} : \txbf{SK} \times \txbf{C} \to \txbf{M}$.
By $\txbf{M}(|m|)$ we denote the distribution of messages with the same
length as $m$.
We require that $\txbf{M}(|m|)$ is efficiently sampleable,
and we require that we can sample $(\txbf{SK}, \txbf{PK})$ via an algorithm
$\tx{Gen}$

The encryption scheme must satisfy a correctness property:
$$
\forall (\tx{sk}, \tx{pk}) \gets \tx{Gen}(),\ m \in \txbf{M}.\ P[\tx{Dec}(\tx{sk}, \tx{Enc}(\tx{pk}, m)) = m] = 1
$$
Encrypting and then decrypting a message should return that same message.

The security of an encryption scheme can be captured by the following game:
\package{$\tx{IND}_b$}{
\pind{0} \set{(\tx{sk}, \tx{pk})}{\tx{Gen}()}\cr
\cr
&\begin{aligned}
    &\pfn{Challenge}{m_0}\cr
    \pind{1} \draw{m_1}{\txbf{M}(|m|)}\cr
    \pind{1} \preturn{\tx{Enc}(\tx{pk}, m_b)}\cr
\end{aligned}
}

In essence, an adversary cannot distinguish between an encryption of a message
of their choice and that of a random message.
\end{document}